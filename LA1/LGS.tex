% -*- TeX-master: "../main" -*-

\chapter{Lineares Gleichungssystem}

\begin{example}
  \begin{align*}
    ax_1 + bx_2 &= c \\
    dx_1 + ex_2 &= f \\
    &\iff \operatorname{LGS}(x_1, x_2)
  \end{align*}
  D.h. Es ist eine Aussageform über $\mathbb{R} \times \mathbb{R}$

  Lösungen: $\mathscr{L} = \left\{ (x_1, x_2) \middle| \operatorname{LGS}(x_1, x_2) \right\}$
\end{example}

\begin{observation}[Geometrie]
  Es seien $x_1, x_2 \in \mathbb{R}$. Man definiert $x =
  \begin{pmatrix}
    x_1\\x_2
  \end{pmatrix}
  : \mathbb{E}^2 \to \mathbb{E}^2$
  . Diese abbildung heißt Vektor.
\end{observation}

\begin{definition}[Vektoraddition]
  $
  \begin{pmatrix}
    x_1\\x_2
  \end{pmatrix}
  \circ
  \begin{pmatrix}
    y_1\\y_2
  \end{pmatrix}
  :=
  \begin{pmatrix}
    x_1+y_1\\x_2+y_2
  \end{pmatrix}
  $
\end{definition}
\begin{definition}[Skalarmultiplikation]
  Sei $\lambda \in \mathbb{R}$.
  $\lambda
  \begin{pmatrix}
    x_1\\x_2
  \end{pmatrix}
  :=
  \begin{pmatrix}
    \lambda x_1 \\ \lambda x_2
  \end{pmatrix}
  $
\end{definition}
\begin{remark}
  Ist $O \in \mathbb{E}^2$ ein „Ursprung“, dann ist $I_0: \mathbb{R}^2 \to \mathbb{E}^2, v \mapsto v(0)$.

  Nach Wahl von $O$: $\mathbb{R}^2 \leftrightarrow \mathbb{E}^2$
\end{remark}
\begin{remark}
  $\mathbf{0} =
  \begin{pmatrix}
    0\\0
  \end{pmatrix}
  $ heißt Nullvektor.
\end{remark}

\begin{proposition}[Lösung eines LGS]
  $\Phi: \mathbb{R}^2 \to \mathbb{R}^2,
  \begin{pmatrix}
    x_1\\x_2
  \end{pmatrix}
  \mapsto
  \begin{pmatrix}
    ax_1 + bx_2\\ cx_1+dx_2
  \end{pmatrix}
  \implies \mathscr{L} = \Phi \left[
    \begin{pmatrix}
      e\\f
    \end{pmatrix}
  \right]
  $

  $\Phi$ hat folgende Eigenschaften:
  \begin{itemize}
  \item $\forall \lambda \in \mathbb{R}, x \in \mathbb{R}^2: \Phi(\lambda x) = \lambda \Phi(x)$
  \item $\forall x, y \in \mathbb{R}^2: \Phi(x + y) = \Phi(x) + \Phi(y)$
  \end{itemize}

  D.h. $\Phi$ ist linear.
\end{proposition}
\begin{remark}
  $\Phi \text{ ist linear} \implies \Phi(\lambda x + \mu y) = \lambda \Phi(x) + \mu \Phi (y), \lambda, \mu \in \mathbb{R}, x, y \in \mathbb{R}^2$
\end{remark}
\begin{notation}
  $Q:=
  \begin{pmatrix}
    a&b\\c&d
  \end{pmatrix}
  \in \mathbb{R}^{2 \times 2}$ ist eine Matrix.

  Zu jeder Matrix $Q$ gibt es eine Abbildung $\Phi_Q: \mathbb{R}^2 \to \mathbb{R}^2,
  x \mapsto
  \begin{pmatrix}
    ax_1 + bx_2 \\ cx_1+ dx_2
  \end{pmatrix}
  = Qx
  $
\end{notation}
\begin{theorem}
  Die Abbildung $\iota: \mathbb{R}^{2 \times 2} \to \operatorname{Lin}(\mathbb{R}^2, \mathbb{R}^2), Q \mapsto \Phi_Q$ ist bijektiv.
\end{theorem}
\begin{definition}[Standardbasis]
  Es seien $e_1 :=
  \begin{pmatrix}
    1\\0
  \end{pmatrix}
  , e_2 :=
  \begin{pmatrix}
    0\\1
  \end{pmatrix}
  $. Das Paar $(e_1, e_2)$ heißt die Standardbasis von $\mathbb{R}^2$
\end{definition}
\begin{remark}
  Es gilt: $Q_{\Phi}(e_1) =
  \begin{pmatrix}
    a\\c
  \end{pmatrix}
  , Q_{\Phi}(e_2) =
  \begin{pmatrix}
    b\\d
  \end{pmatrix}
  $.
  $Q$ heißt also die darstellende Matrix von $\Phi$
\end{remark}

\begin{definition}
  $\operatorname{Karn}(\Phi) = \Phi^{-1}[\mathbf{0}]$

  Ein LGS heißt homogen, falls $e = f = 0$. Seine Lösung ist dann $\operatorname{Kern}(\Phi)$
\end{definition}
\begin{proposition}
  Sei $W = \operatorname{Kern}(\Phi)$
  \begin{itemize}
  \item $\mathbf{0} \in W$, insbesondere ist $W \ne \emptyset$
  \item $\forall x, y \in W: x+y \in W$
  \item $\forall x \in W, \lambda \in \mathbb{R}: \lambda x \in W$
  \end{itemize}
  $W$ heißt Untervektorraum.
\end{proposition}
\begin{definition}[Ursprungsgerade]
  Ist $x \in \mathbb{R}^2$, dann heißt $\mathbb{R}x := \left\{ \lambda x \middle| \lambda \in \mathbb{R} \right\}$ die Ursprungsgerade durch $x$.
\end{definition}
\begin{theorem}
  Es gibt keine weitere Untervektorräume als
  \begin{itemize}
  \item $\left\{ \mathbf{0} \right\}$
  \item Unsprungsgeraden: $\left\{ \lambda x \middle| \lambda \in \mathbb{R}, x \in \mathbb{R}^2 \right\}$
  \item $\mathbb{R}^2$
  \end{itemize}
\end{theorem}