% -*- TeX-master: "../main" -*-

\chapter{Sprache der Mathematik}

Sprache der Aussagelogik und Mengenlehre: es geht um Aussagen.

\begin{definition}[Aussage\index{Aussage}]
  Korrektes sprachliches Gebilde mit eindeutigem Wahrheitswert: wahr (w) oder falsch (f)
\end{definition}

\section{Kalkül von Aussagen}
Es seien $A, B, C, D, \dots $ Aussagen.

\begin{itemize}
\item $\neg A$: Negation von $A$: „$A$ ist falsch“
\item $A \wedge B$: Kunjunktion: „$A$ und $B$ sind beide wahr“
\item $A \vee B$: Disjunktion: „Mindestens eine von $A$ und $B$ ist wahr“
\item $C :\iff (A \wedge B) \vee (\neg A \wedge \neg B)$: Definition/Zuweisung von (z.B.) $C$
\item $A \implies B :\iff (\neg A) \vee B$: Implikation: „$A$ impliziert  $B$“
\item $A \iff B :\iff (A \implies B) \wedge (B \implies A)$: Äquivalenz
\end{itemize}

\begin{tabularx}{1.0\linewidth}{|X|X|X|X|X|X|X|X|}
  \hline
  $A$&$B$&$\neg A$&$A \wedge B$&$A \vee B$&$C$&$A \implies B$&$A \iff B$\\\hline
  w & w & f & w & w & w & w & w \\
  w & f & f & f & w & f & f & f \\
  f & w & w & f & w & f & w & f \\
  f & f & w & f & f & w & w & w \\\hline
\end{tabularx}

\begin{remark}
  $(A \iff B) \iff C$
\end{remark}

\begin{theorem}
  Eigeschaften:
  \begin{itemize}
  \item $A \implies A$
  \item Transitivität: $(A \implies B) \wedge (B \implies C) \implies (A \implies C)$
  \item Assoziativ: $(A \wedge B) \wedge C \iff A \wedge (B \wedge C);\; (A \vee B) \vee C \iff A \vee (B \vee C)$
  \item Kommutativ: $A \wedge B \iff B \wedge A;\; A \vee B \iff B \vee A$
  \item Distributiv: $A \wedge (B \vee C) \iff (A \wedge B) \vee (A \wedge C);\; A \vee (B \wedge C) \iff (A \vee B) \wedge (A \vee C)$
  \item $\neg (A \wedge B) \iff (\neg A) \vee (\neg B);\; \neg (A \vee B) \iff (\neg A) \wedge (\neg B)$
  \item Doppelte Negation: $\neg (\neg A) \iff A$
  \end{itemize}
\end{theorem}

\begin{remark}Binderegeln:
  \begin{itemize}
  \item Klammern (wenn möglich) weglassen! $A \wedge B$ statt $(A \wedge B)$, $A \wedge B \wedge C$ statt $(A \wedge B) \wedge C$
  \item $\neg$ vor $\wedge, \vee$ vor $\implies, \iff$. $A \wedge B \wedge \neg C \implies A \wedge \neg C$ statt $(((A \wedge B) \wedge (\neg C)) \implies (A \wedge \neg C))$
  \end{itemize}
\end{remark}

\begin{corollary}[Kontraposition]
  $(A \implies B) \iff (\neg B \implies \neg A)$
\end{corollary}
\begin{proof}
  \begin{align*}
    (A \implies B) &\iff \neg A \vee B \\
    \, &\iff B \vee \neg A \\
    \, &\iff \neg (\neg B) \vee \neg A \\
    \, &\iff (\neg B \implies \neg A)
  \end{align*}
\end{proof}

Übung:
\begin{itemize}
\item $(A \implies B) \iff \neg(A \wedge \neg B)$ (z.B. Beweis durch Widerspruch)
\end{itemize}

\section{Sprache der Mengenlehre}
\begin{definition}[Mengenlehre\index{Mengenlehre}]
  Eine Menge $M$ ist eine Zusammenfassung von Objekten, so dass jedes „denkbare Objekt“ entweder in $M$ liegt oder nicht.
\end{definition}

\begin{itemize}
\item $x \in M$: „$x$ liegt in $M$“
\item $x \not\in M$: „ $x$ liegt nicht in $M$“
\item Definition von Mengen: $M := \left\{ 1, a, \alpha \right\}$
\end{itemize}

Zwei Mengen sind gleich, wenn sie dieselben Elemente haben: $\left\{ 1, a, \alpha \right\} = \left\{ a, 1, \alpha, a \right\}$

\begin{example}
  \begin{itemize}
  \item Natürliche Zahlen: $\mathbb{N} := \left\{ 1, 2, 3, 4, \dots \right\}$
  \item Ganze Zahlen: $\mathbb{Z} := \left\{ \dots, -2, -1, 0, 1, 2, \dots \right\}$
  \item „Menge aller Menschen“: $\left\{ x \middle| x \text{ ist Mensch} \right\}$
  \item „4 ist eine natürliche Zahl“: $4 \in \mathbb{N}$
  \item „Das Auto $A$ ist blau“: $A \in \left\{ x \middle| x \text{ ist blau} \right\}$
  \end{itemize}
\end{example}

\begin{definition}[Mengenoperationen]
  \begin{align*}
    M \cap N &:= \left\{ x \middle| x \in M \wedge x \in N \right\}\\
    M \cup N &:= \left\{ x \middle| x \in M \vee x \in N \right\}\\
    M \setminus N &:= \left\{ x \middle| x \in M \wedge x \not\in N \right\}\\
    M \subseteq N &\iff (x \in M \implies x \in N) \\
    M = N &\iff (x \in M \iff x \in N) \iff (M \subseteq N \wedge N \subseteq M) \\
    M \subsetneq N &\iff M \subseteq N \wedge M \neq N
  \end{align*}
\end{definition}

\begin{theorem}
  Für Mengen $M$, $N$ und $O$ gilt (vgl. logische Operatoren):
  \begin{align*}
    M &\subseteq M\\
    (M \subseteq N \wedge N \subseteq O) &\implies M \subseteq O\\
    (M \cap N) \cap O &= M \cap (N \cap O) \\
    M \cap N &= N \cap M\\
    M \cup N &= N \cup M\\
    M \cap (N \cup O) &= (M \cap N) \cup (M \cap O)\\
    M \cup (N \cap O) &= (M \cup N) \cap (M \cup O)\\
    N \subseteq O &\implies M \cap N \subseteq M \cap O \\
    O \setminus (M \cap N) &= (O \setminus M) \cup (O \setminus N) \\
    O \setminus (M \cup N) &= (O \setminus M) \cap (O \setminus N) \\
    O \setminus (O \setminus M) &= M \cap O
  \end{align*}
\end{theorem}

\begin{remark}
  \begin{itemize}
  \item Leere Menge: $\emptyset := \{\}$\index{Mengenlehre!$\emptyset$}
  \item Menge können andere Mengen als Elemente haben. $\left\{ \emptyset \right\}$ hat ein Element, $\left\{ \emptyset, \left\{ \emptyset \right\} \right\}$ hat zwei Elemente.
  \item Für $M:= \left\{ 1, \left\{ 2 \right\} \right\}$
    \begin{align*}
      1 &\in M & 2 &\not\in M \\
      \left\{ 1 \right\} &\subseteq M & \left\{ 2 \right\} &\not\subseteq M \\
      \left\{ 1 \right\} &\not\in M & \left\{ 2 \right\} &\in M \\
      \left\{ \left\{ 1 \right\} \right\} &\not\subseteq M & \left\{ \left\{ 2 \right\} \right\} &\subseteq M \\
    \end{align*}
  \end{itemize}
\end{remark}

\begin{definition}[Potenzmenge\index{Mengenlehre!Potenzmenge}]
  $\mathscr{P}(M) := \left\{ N \middle| N \subseteq M \right\}$ ist die Potenzmenge einer Menge $M$. $\mathscr{A} \subseteq \mathscr{P}(M)$ ist ein Mengensystem\index{Mengenlehre!Mengensystem} über $M$.
\end{definition}

\begin{remark}
Russell (1901): Es sei $R := \left\{ M \middle| M \text{ ist Menge} \wedge M \not\in M \right\}$. Ist $R$ in $R$?
\begin{align*}
  R \in R &\implies R \not\in R \\
  R \not\in R &\implies R \in R
\end{align*}

Axiomatische Mengenlehre (grob zusammengefasst):
\begin{itemize}
\item Klasse: Zusammenfassungen von Objekten, die durch Eigenschaften beschrieben werden.
\item Mengen: Klassen, die selbst Element einer Klasse sind.
\end{itemize}
\end{remark}