% -*- TeX-master: "../main" -*-
\chapter{Induktives Vorgehen}

\section{Vollständige Induktion}
\begin{definition}[Peano-Axiome]
  Folgende Aussagen charakterisieren die natürlichen Zahlen:
  \begin{itemize}
  \item $n \in \mathbb{N}_0$
  \item $\forall n \in \mathbb{N}_0: n+1 \in \mathbb{N}_0$
  \item $\forall n \in \mathbb{N}_0: n + 1 \ne 0$
  \item $\forall n, m \in \mathbb{N}_0: n+1 = m+1 \implies n = m$
  \item $\forall M \subseteq \mathbb{N}_0: (0 \in M \wedge \forall n \in M: n+1 \in M) \implies M = \mathbb{N}$
  \end{itemize}
\end{definition}

\begin{definition}[Struktur der vollständigen Induktion]
  Um $\forall n \in \mathbb{N}_0: A_n$ ($A_n$ heißt Induktionshypothese) zu zeigen, zeigt man
  \begin{description}
  \item[Induktionsanfang] $A_0$
  \item[Induktionsschritt] $A_n \implies A_{n+1}$
  \end{description}

  Im Induktionsschritt heißt $A_n$ Induktionsvorraussetzung und $A_{n+1}$ Induktionsbehauptung.

  \begin{prooftree}
    \AxiomC{$A_0$}
    \AxiomC{$\forall n \in \mathbb{N}_0: A_n \implies A_{n+1}$}
    \BinaryInfC{$\forall n \in \mathbb{N}_0: A_n$}
  \end{prooftree}
\end{definition}