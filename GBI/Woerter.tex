% -*- TeX-master: "../main" -*-

\chapter{Wörter}

\section{Definition von Wörtern}
\begin{definition}[Alphabet]
  Ein Alphabet $A$ ist eine nichtleere Menge von Zeichen oder Symbolen.
\end{definition}

\begin{definition}
  Es sei $\mathbb{Z}_n = \left\{ i \in \mathbb{N}_0 \middle| 0 \le i \wedge i < n \right\}$ (Achtung: $\mathbb{Z}_0 = \emptyset$). Ein Wort (über ein Alphabet $A$) ist eine Abbildung: $w: \mathbb{Z}_n \to A$. $|w| = n$ heißt die Länge des Wortes.
\end{definition}
\begin{remark}[Das leere Wort]
  $\varepsilon: \mathbb{Z}_0 \to A$ oder $\varepsilon: \emptyset \to A$. Als relation $R$ gilt: $R \subseteq \emptyset \times A = \emptyset$. Das leere Wort ist also (unabhängig vom Alphabet) eindeutig.
\end{remark}
\begin{definition}[Menge von Wörtern der Länge $n$]
  $A^n := \left\{ w: \mathbb{Z}_n \to A \right\}$
\end{definition}
\begin{definition}[Menge aller Wörter]
  $A^{*} := \bigcup_{n \in \mathbb{N}_0} A^n$
\end{definition}

\section{Konkatenation von Wörtern}

\begin{definition}[Konkatenation]
  Für $w_1: \mathbb{Z}_m \to A$ und $w_2: \mathbb{Z}_n \to A$ ist
  \[ w_1 \cdot w_2: \mathbb{Z}_{m+n} \to A, i \to
    \begin{cases}
      w_1(i) & 0 \le i < m \\
      w_2(i-m) & m \le i < m+n
    \end{cases}
  \]
\end{definition}
\begin{remark}
  $|w_1 \cdot w_2| = |w_1| + |w_2|$
\end{remark}
\begin{lemma}
  $\varepsilon$ ist das neutrale Element der Konkatenation: $\forall w \in A^{*}: \varepsilon \cdot w = w \wedge w \cdot \varepsilon = w$
\end{lemma}
\begin{remark}
  Eigenschaften der Konkatenation:
  \begin{itemize}
  \item Konkatenation ist nicht kommutativ.
  \item Konkatenation ist assoziativ.
  \end{itemize}
\end{remark}

\begin{definition}[Iterierte Konkatenation]
  Es sei $w$ ein Wort.
  \begin{align*}
    w^0 &:= \varepsilon \\
    \forall n \in \mathbb{N}_0: w^{n+1} &:= w^n \cdot w
  \end{align*}
\end{definition}
\begin{lemma}
  $|w^n| = n|w|$
\end{lemma}