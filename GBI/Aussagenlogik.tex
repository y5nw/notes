% -*- TeX-master: "../main" -*-
\chapter{Aussagenlogik}

\section{Syntax aussagenlogischer Formeln}
\begin{definition}[Alphabet aussagenlogischer Formeln]
  $A_{AL} = \{ \textstring{(}, \textstring{)}, \mathstring{\neg}, \mathstring{\wedge}, \mathstring{\vee}, \mathstring{\to}, \mathstring{\leftrightarrow} \} \cup \textit{Var}_{AL}$
\end{definition}

\begin{definition}
  Aussagenlogische Konnektive/Verknüpfung:
  \begin{description}
  \item[Negation] $\textstring{(} \mathstring{\neg} G \textstring{)}$
  \item[Konjunktion] $\textstring{(} G \mathstring{\wedge} H \textstring{)}$
  \item[Disjunktion] $\textstring{(} G \mathstring{\vee} H \textstring{)}$
  \item[Implikation] $\textstring{(} G \mathstring{\to} H \textstring{)}$
  \end{description}
\end{definition}
\begin{remark}[Konstruktionsabbildungen]
  Für jede Verknüpfung gibt es eine Abbildung. Z.B.: $f_{\mathstring{\wedge}}: A_{AL}^{*} \times A_{AL}^{*} \to A_{AL}^{*}: (G, H) \to \textstring{(} G \mathstring{\wedge} H \textstring{)}$
\end{remark}
\begin{remark}[Menge einmaliger Verknüpfungen]
  $\operatorname{Ver}(M) := \left\{ \textstring{(} \mathstring{\neg} G \textstring{)}, \textstring{(} G \mathstring{\wedge} H \textstring{)}, \textstring{(} G \mathstring{\vee} H \textstring{)}, \textstring{(} G \mathstring{\to} H \textstring{)} \middle| G, G \in \textit{Var}_{AL} \right\}$
\end{remark}

\begin{definition}[Aussagenlogische Variablen $\textit{Var}_{AL}$]
  Aussagenlogische Variablen (oder: Atome) werden oft mit $\textstring{P}_i, i \in \mathbb{N}_0$ oder $\textstring{P}, \textstring{Q}, \textstring{R}, \dots$ benannt.
\end{definition}

\begin{definition}[Formale Sprache $\textit{For}_{AL}$ der Aussagenlogik]
  \begin{align*}
    M_0 &:= \textit{Var}_{AL} \\
    \forall n \in \mathbb{N}_0: M_{n+1} &:= M_n \cup \operatorname{Ver}(M_n) \\
    For_{AL} &:= \bigcup_{n \in \mathbb{N}_0} M_n
  \end{align*}
\end{definition}

\begin{notation}
  Abkürzungen (Der „offiziele“ Syntax bleibt gleich):
  \begin{itemize}
  \item Man kann die äußersten Klammern weglassen
  \item mehrfach gleiches Konnektiv: „implizierte Linksklammerung“
  \item Bindestärke von Konnektiven: (stärksten) \mathstring{\neg}, \mathstring{\wedge}, \mathstring{\vee}, \mathstring{\to}, \mathstring{\leftrightarrow} (schwächsten)
  \item $\textstring{(} G \mathstring{\leftrightarrow} H \textstring{)}$ steht für $\textstring{(} \textstring{(} G \mathstring{\to} H \textstring{)} \mathstring{\wedge} \textstring{(} H \mathstring{\to} G \textstring{)} \textstring{)}$
  \end{itemize}
\end{notation}

\section{Semantik der Aussagenlogik}
Aussagen habe genau zwei Wahrheitswerte: „wahr“ (\consttrue) oder „falsch“ (\constfalse).
\begin{definition}
  $\mathbb{B} := \left\{ \consttrue, \constfalse \right\}$
\end{definition}
\begin{definition}[Interpretation]
  $I: V \to \mathbb{B}, V \subseteq \textit{Var}_{AL}$
\end{definition}
\begin{definition}[Auswertung von Formeln]
  \begin{align*}
    \forall G \in V: \operatorname{val}_I(G) &:= I(G) \\
    \operatorname{val}_I(\mathstring{\neg} G) &:=
                                             \begin{cases}
                                               \consttrue & I(G) = \constfalse \\
                                               \constfalse & \text{sonst}
                                             \end{cases} \\
    \operatorname{val}_I(G \mathstring{\wedge} H) &:=
                                               \begin{cases}
                                                 \consttrue & I(G) = \consttrue \text{ und } I(H) = \consttrue \\
                                                 \constfalse & \text{sonst}
                                               \end{cases} \\
    \operatorname{val}_I(G \mathstring{\vee} H) &:=
                                               \begin{cases}
                                                 \consttrue & I(G) = \consttrue \text{ oder } I(H) = \consttrue \\
                                                 \constfalse & \text{sonst}
                                               \end{cases} \\
    \operatorname{val}_I(G \mathstring{\to} H) &:=
                                               \begin{cases}
                                                 \consttrue & \text{wenn } I(G) = \consttrue \text{ dann } I(H) = \consttrue \\
                                                 \constfalse & I(G) = \consttrue \text{ und } I(H) = \constfalse
                                               \end{cases}
  \end{align*}
\end{definition}
\begin{remark}
  Das „Oder“ ist inklusiv.
\end{remark}

\begin{definition}
  Wenn zwei Aussagen $G$ und $H$ für alle Interpretationen den gleichen Wahrheitswerten annehmen, dann sind sie äquivalent. Man schreibt $G \equiv H$
\end{definition}

\begin{definition}[Modell]
  Ist $\operatorname{val}_I(G) = \consttrue$, dann nennt man $I$ ein Modell von $G$.

  Ist $I$ ein Modell jeder Formel $G \in \Gamma$, dann nennt man $I$ ein Modell von $\Gamma$.
\end{definition}

\begin{notation}
  Es sei $\Gamma$ eine Menge von Formeln und $G$ eine Formel. Ist jedes Modell von $\Gamma$ auch Modell von $G$, so schreibt man $\Gamma \models G$.

  Man schreibt $H \models G$ statt $\left\{ H \right\} \models G$ und $\models G$ statt $\emptyset \models G$.
\end{notation}
\begin{definition}
  Ist $\models G$, dann ist $G$ für alle Interpretationen wahr. $G$ heißt eine Tautologie oder eine allgemeingültige Formel.
\end{definition}
\begin{lemma}
  $G \equiv H$ gilt genau dann, wenn $G \leftrightarrow H$ Tautologie ist.
\end{lemma}
\begin{definition}
  $G$ heißt erfüllbar, wenn $G$ für mindestens eine Interpretation wahr ist.
\end{definition}

\section{Beweisbarkeit}
\begin{notation}[Logische Schlussregeln]
  Man schreibt
  \begin{prooftree}
    \AxiomC{$V_1$}
    \AxiomC{$V_2$}
    \AxiomC{$\cdots$}
    \AxiomC{$V_n$}
    \QuaternaryInfC{$C$}
  \end{prooftree}
  $V_i$ heißen Vorraussetzungen, $C$ heißt Folgerung („conclusio“).

  Eine Regel heißt korrekt, wenn $\left\{ V_1, V_2, \dots, V_n \right\} \models C$
\end{notation}
\begin{example}
  \AxiomC{$A \wedge B$}
  \UnaryInfC{$A$}
  \DisplayProof
\end{example}
\begin{example}[Modus ponens]
  \AxiomC{$A$}
  \AxiomC{$A \to B$}
  \BinaryInfC{$B$}
  \DisplayProof
\end{example}
\begin{example}[Aussagen mit Bedingungen]
  \AxiomC{$\Gamma_1 \models V_1$}
  \AxiomC{$\Gamma_2 \models V_2$}
  \AxiomC{$\cdots$}
  \AxiomC{$\Gamma_n \models V_n$}
  \QuaternaryInfC{$\scriptstyle \left( \bigcup_{i=1}^n \Gamma_i \right) \models C$}
  \DisplayProof
\end{example}

\begin{example}
  Zu zeigen: $\models (P \vee \neg P)$
  \begin{prooftree}
    \AxiomC{}
    \UnaryInfC{$\neg (P \vee \neg P) \models \neg (P \vee \neg P)$}
    \AxiomC{}
    \UnaryInfC{$\neg (P \vee \neg P) \models \neg (P \vee \neg P)$}
    \AxiomC{}
    \UnaryInf$P \ \fCenter \models P$
    \UnaryInf$P \ \fCenter \models P \vee \neg P$
    \BinaryInf$\neg (P \vee \neg P), P \ \fCenter \models \constfalse$
    \UnaryInf$\neg (P \vee \neg P) \ \fCenter \models \neg P$
    \UnaryInf$\neg (P \vee \neg P) \ \fCenter \models P \vee \neg P$
    \BinaryInfC{$\neg (P \vee \neg P) \ \fCenter \models \constfalse$}
    \UnaryInfC{$P \vee \neg P$}
  \end{prooftree}
\end{example}