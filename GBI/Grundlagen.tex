% -*- TeX-master: "../main" -*-
\chapter{Mengen und Abbildungen}

\section{Mengen}

Eine Menge\index{Menge} ist ein „Behälter“ von „Objekten“ (Elementen).

\begin{definition}
  \begin{align*}
    \emptyset &:= \{\} \tag{leere Menge\index{Menge!Leere Menge}} \\
    x \in A \cup B &:\iff x \in A \vee x \in B \tag{Vereinigung\index{Menge!Vereinigung}}\\
    x \in A \cap B &:\iff x \in A \wedge x \in B \tag{Durchschnitt\index{Menge!Durchschnitt}}\\
    A \subseteq B &:\iff x \in A \implies x \in B \tag{Teilmenge und Obermenge\index{Menge!Teilmenge}\index{Menge!Obermenge}}\\
    A = B &:\iff A \subseteq B \wedge B \subseteq A \tag{Gleichheit\index{Menge!Gleichheit}} \\
    A \subsetneq B &:\iff A \subseteq B \wedge A \ne B \tag{Echte Teilmenge\index{Menge!Echte Teilmenge}}
  \end{align*}
  Ist $A \cap B = \emptyset$, dann sind $A$ und $B$ disjunkte Mengen\index{Menge!Disjunkte Mengen}.
  \index[sym]{$\emptyset$}\index[sym]{$\cup$}\index[sym]{$\cap$}\index[sym]{$\subseteq$}\index[sym]{$\subsetneq$}\index[sym]{$=$}
\end{definition}
\begin{lemma}
  \begin{align*}
    A \cup A &= A & A \cap A &= A \tag{Idempotenzgesetz\index{Menge!Idempotenzgesetz}} \\
    A \cup B &= B \cup A & A \cap B &= B \cap A \tag{Kommutativgesetz\index{Menge!Kommutativgesetz}} \\
    (A \cup B) \cup C &= A \cup (B \cup C) & (A \cap B) \cap C &= A \cap (B \cap C) \tag{Assoziativgesetz\index{Menge!Assoziativgesetz}} \\
    A \cup (B \cap C) &= (A \cup B) \cap (A \cup C) & A \cap (B \cup C) &= (A \cap B) \cup (A \cap C) \tag{Distributivgesetz\index{Menge!Distributivgesetz}}
  \end{align*}
\end{lemma}

\begin{notation}[Allquantor]
  $\forall$: „für alle“
\end{notation}
\begin{notation}[Existenzquantor]
  $\exists$: „es gibt“
\end{notation}

\begin{definition}[Kartesisches Produkt]
  $A \times B := \left\{ (a, b) | a \in A, b \in B \right\}$
\end{definition}
\begin{remark}
  $M^2 = M \times M, M^3 = M \times M \times M, \dots$
\end{remark}

\section{Relationen und Abbildungen}

\begin{definition}[Relation]
  $R \subseteq A \times B$ heißt binäre Relation von $A$ in $B$.
\end{definition}
\begin{notation}
  Für $(a, b) \in R$ sagt man: „$a$ steht in Relation $R$ zu $b$“ und schreibt $a R b$ (z.B. für $\le \subseteq \mathbb{N} \times \mathbb{N}$: $23 \le 42$ statt $(23,42) \in \le$).
\end{notation}

\begin{definition}
  $R$ heißt
  \begin{itemize}
  \item linkstotal, wenn $\forall a \in A\: \exists b \in B: (a, b) \in R$
  \item linkseindeutig, wenn $\forall (a_1, b_1), (a_2, b_2) \in R: a_1 \ne a_2 \implies b_1 \ne b_2$
  \item rechtstotal, wenn $\forall b \in B\: \exists a \in A: (a, b) \in R$
  \item rechtseindeutig, wenn $\forall a \in A\: \forall b_1, b_2 \in B: (a, b_1) \in R \wedge (a, b_2) \in R \implies b_1 = b_2$
  \end{itemize}
\end{definition}
\begin{definition}[Funktion]
  Relationen, die linkstotal und rechtseindeutig sind, heißen (totale) Funktionen (oder Abbildungen). $A$ heißt der Definitionsbereich, $B$ heißt der Zielbereich.

  Man schreibt: $R(a) = b$. $b$ heißt Funktionswert an der Stelle $a$.

  Partielle Funktionen sind rechtseindeutig aber nicht linkstotal.

  Eine linkseindeutige Funktion heißt injektiv. Eine rechtstotale Funktion heißt surjektiv. Eine injektiv und surjektive Funktion heißt bijektiv.
\end{definition}

\begin{notation}[Angabe vom Definitions- und Zielbereich]
  $f: A \to B$
\end{notation}
\begin{notation}[Spezifikation und Funktionswerten]
  $f: a \mapsto b$
\end{notation}
\begin{remark}
  Zu einer Abbildung (und auch einer Relation) gehören
  \begin{itemize}
  \item der Definitionsbereich,
  \item der Zielbereich und
  \item die Abbildungsvorschrift
  \end{itemize}
\end{remark}

\section{Mehr zu Mengen}

\begin{notation}[Mengen von Funktionen]
  $B^A := \left\{ f \middle| f \text{ ist eine Funktion mit } f: A \to B \right\}$
\end{notation}
\begin{remark}
  Für endliche Mengen gilt: $|B^A| = |B|^{|A|}$
\end{remark}

\begin{remark}[Konstruktion natürlicher Zahlen durch Mengen]
  \begin{align*}
    \emptyset &= \left\{  \right\} \\
    \not{1} &:= \left\{ \emptyset \right\} \\
    \not2 &:= \left\{ \emptyset, \not1 \right\} \\
    \not3 &:= \left\{ \emptyset, \not1, \not2 \right\} \\
  \end{align*}
  usw.
\end{remark}

\begin{definition}[Potenzmenge]
  $\mathscr{P}(M) = \left\{ A \middle| A \subseteq M \right\}$
\end{definition}
\begin{notation}
  Für die Potenzmenge $\mathscr{P}(M)$ schreibt man auch $2^M$
\end{notation}
\begin{remark}
  $A \in 2^M \iff A \subseteq M$
\end{remark}
\begin{notation}
  \begin{align*}
    \bigcup_{i \in I} M_i &= \left\{ x \middle| \exists i \in I: x \in M_i \right\} \\
    \bigcap_{i \in I} M_i &= \left\{ x \middle| \forall i \in I: x \in M_i \right\}
  \end{align*}
\end{notation}