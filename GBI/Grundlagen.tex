% -*- TeX-master: "../main" -*-
\chapter{Mengen, Alphabete, Abbildungen}

Eine Menge\index{Menge} ist ein „Behälter“ von „Objekten“ (Elementen).

\begin{definition}
  \begin{align*}
    \emptyset &:= \{\} \tag{leere Menge\index{Menge!Leere Menge}} \\
    x \in A \cup B &:\iff x \in A \vee x \in B \tag{Vereinigung\index{Menge!Vereinigung}}\\
    x \in A \cap B &:\iff x \in A \wedge x \in B \tag{Durchschnitt\index{Menge!Durchschnitt}}\\
    A \subseteq B &:\iff x \in A \implies x \in B \tag{Teilmenge und Obermenge\index{Menge!Teilmenge}\index{Menge!Obermenge}}\\
    A = B &:\iff A \subseteq B \wedge B \subseteq A \tag{Gleichheit\index{Menge!Gleichheit}} \\
    A \subsetneq B &:\iff A \subseteq B \wedge A \ne B \tag{Echte Teilmenge\index{Menge!Echte Teilmenge}}
  \end{align*}
  Ist $A \cap B = \emptyset$, dann sind $A$ und $B$ disjunkte Mengen\index{Menge!Disjunkte Mengen}.
  \index[sym]{$\emptyset$}\index[sym]{$\cup$}\index[sym]{$\cap$}\index[sym]{$\subseteq$}\index[sym]{$\subsetneq$}\index[sym]{$=$}
\end{definition}
\begin{lemma}
  \begin{align*}
    A \cup A &= A & A \cap A &= A \tag{Idempotenzgesetz\index{Menge!Idempotenzgesetz}} \\
    A \cup B &= B \cup A & A \cap B &= B \cap A \tag{Kommutativgesetz\index{Menge!Kommutativgesetz}} \\
    (A \cup B) \cup C &= A \cup (B \cup C) & (A \cap B) \cap C &= A \cap (B \cap C) \tag{Assoziativgesetz\index{Menge!Assoziativgesetz}} \\
    A \cup (B \cap C) &= (A \cup B) \cap (A \cup C) & A \cap (B \cup C) &= (A \cap B) \cup (A \cap C) \tag{Distributivgesetz\index{Menge!Distributivgesetz}}
  \end{align*}
\end{lemma}