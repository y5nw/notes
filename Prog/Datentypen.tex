% -*- TeX-master: "../main" -*-
\chapter{Datentypen}

\section{Elementare Datentypen und Operationen}
\begin{definition}
  Elementare Datentypen:
  \begin{description}
  \item[\texttt{boolean}] Wahrheitswerte: \texttt{true}, \texttt{false}\index[java]{boolean@\texttt{boolean}}\index[java]{true@\texttt{true}}\index[java]{false@\texttt{false}}
  \item[\texttt{char}] 16-Bit-Unicode-Zeichen\index[java]{char@\texttt{char}}
  \item[\texttt{byte}] 8-Bit-Ganzzahl\index[java]{byte@\texttt{byte}}
  \item[\texttt{short}] 16-Bit-Ganzzahl\index[java]{short@\texttt{short}}
  \item[\texttt{int}] 32-Bit-Ganzzahl\index[java]{int@\texttt{int}}
  \item[\texttt{long}] 64-Bit-Ganzzahl (beachte: \texttt{12L} statt \texttt{12})\index[java]{long@\texttt{long}}
  \item[\texttt{float}] 32-Bit-Gleitpunktzahl (beachte: \texttt{9.81F} statt \texttt{9.8})\index[java]{float@\texttt{float}}
  \item[\texttt{double}] 64-Bit-Gleitpunktzahl\index[java]{double@\texttt{double}}
  \end{description}
\end{definition}
\begin{definition}
  Operationen auf elementaren Datentypen:
  \begin{enumerate}
  \item \texttt{+x}, \texttt{-x}, \texttt{\^{}x}, \texttt{!x}
  \item \texttt{x*y}, \texttt{x/y}, \texttt{x\%y}
  \item \texttt{x+y}, \texttt{x-y}
  \item \texttt{x<{}<y}, \texttt{x>{}>y}, \texttt{x>{}>{}>y}
  \item \texttt{<}, \texttt{<=}, \texttt{>}, \texttt{>=}
  \item \texttt{x==y}, \texttt{x!=y}
  \item \texttt{x\&y}
  \item \texttt{x\^{}y}
  \item \texttt{x|y}
  \item \texttt{x\&\&y}
  \item \texttt{x||y}
  \end{enumerate}
\end{definition}
\begin{remark}[Gleitkommazahlen]
  Die Zahlen sind nicht gleichmäßig dicht dargestellt. Dies können zu Rundungsfehlern führen.
\end{remark}

\section{Der Datentyp \texttt{String}}
Man verwendet \texttt{String}\index[java]{String@\texttt{String}} für Zeichenketten.
\begin{lstlisting}[language=Java]
String text = "Hallo";
String address = "Am Fasanengarten 5\n76131 Karlsruhe";
\end{lstlisting}
\begin{remark}
  Konkatenation von \texttt{String}s ist durch \texttt{+} möglich.
\end{remark}

\section{\texttt{enum}}
\index[java]{enum@\texttt{enum}}
\begin{lstlisting}[language=Java]
enum Color {RED, GREEN, BLUE}
\end{lstlisting}
\begin{remark}
  \texttt{enum}-Operationen müssen selbst definiert werden.
\end{remark}

\section{Objekt-Variablen}
Objekt-Variablen sind Referenzen, d.h., Speicheradressen. \texttt{null}\index[java]{null@\texttt{null}} steht für „kein Objekt“.

\section{Konstanten}
\begin{lstlisting}[language=Java]
static final float PI = 3.14159265f;
\end{lstlisting}