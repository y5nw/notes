% -*- TeX-master: "../main" -*-
\chapter{Kontrollstrukturen}

\section{Ausdrücke und Anweisungen}
\begin{itemize}
\item Ein Ausdrück beschreibt eine vorzunehmende Auswertung.
\item Eine Anweisung ist eine Einheit der Ausführung.
\item Bestimmte Ausdrücke können als Anweisungen verwendet werden.
\end{itemize}

\section{Eingaben über die Konsole}
\lstinputlisting[language=Java]{Prog/ScannerExample.java}

\section{\texttt{if}-Anweisung}
\lstinputlisting[language=Java]{Prog/IfExample.java}

\section{\texttt{switch}-Anweisung}
\lstinputlisting[language=Java]{Prog/SwitchExample.java}

\section{\texttt{while}-Schleife}
\lstinputlisting[language=Java]{Prog/WhileExample.java}

\section{\texttt{do-while}-Schleife}
\begin{lstlisting}[language=Java]
while (Bedingung) {Anweisungen}
do {Anweisungen} while (Bedingung);
\end{lstlisting}

In der \texttt{do-while}-Schleife wird die Bedingung erst nach der Ausführung der Anweisungen geprüft!

\section{\texttt{for}-Schleife}
\lstinputlisting[language=Java]{Prog/ForExample.java}

\section{\texttt{break}-Anweisung}
\texttt{break} veranlasst das sofortige Verlassen der innersten Schleife.

\section{\texttt{continue}-Anweisung}
\texttt{continue} bricht die aktuelle Iteration ab und springt zur nächsten Iteration.