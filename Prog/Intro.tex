% -*- TeX-master: "../main" -*-
\chapter{Einführung}

\section{Einfache Programme}

\begin{example}[Algorithmus für Summation]
  \( \sum_1^n = \frac{n(n+1)}{2} \)
  \lstinputlisting[language=Java]{Prog/SimpleProgram/SimpleProgram.java}
\end{example}

\texttt{javac} Übersetzt Quellcode in Klassen. \texttt{java} führt das Programm aus. Dabei wird in der Klasse die Methode \texttt{main}\index[java]{main@\texttt{main}} gestartet.

Compiler\index{Compiler}: Sie transformieren die Programm in maschinennahe Programme (oder Bytecode), die später ausgeführt werden. Der Quellcode wird in das Zielprogramm übersetzt.

Interpreter\index{Interpreter}: Sie übersetzen die Anweisungen und führen sie unmittelbar aus.

Java ist (anders als z.B. C) vom Plattform unabhängig: Programme werden in (vom Plattform unabhängigen) Java-Bytecode übersetzt, der vom Interpreter ausgeführt.

JIT (Just-In-Time) Compilation\index{Compiler!JIT-Compilation}: Findet während der Ausführung des Codes statt; Optimierungen

Wenn man den Quellcode ändern, muss man das Programm wieder compilieren.

\section{Objekten und Klassen}

Jedes Objekt der Realität hat ein virtuelles Gegenstück. Objekte\index{Java!Objekt} kooperieren durch Datenaustausch (z.B. „$A$ ruft eine Methode von $B$ auf“). Objekten werden durch drei Aspekte charakterisiert:
\begin{itemize}
\item Identität: Objektnamen (oder: Speicheradresse)
\item Zustand: Attribute\index{Java!Objekt!Attribut}
\item Verhalten: Methoden\index{Java!Objekt!Methode}
\end{itemize}

Was modelliert man bei einem Billiard-Spiel?
\begin{itemize}
\item Tisch: Größe
\item Bälle: Farbe, Gewicht, Position
\end{itemize}

Klasse\index{Java!Klasse}\index[java]{class@\texttt{class}}: „Bauplan“ von Objekten. Sie legen fest, welche Attribute und Methoden die Objekt-Instanzen\index{Java!Objekt!Instanz} der Klassen haben können.

Operatoren werden mit dem \texttt{new}-Operator\index[java]{new@\texttt{new}} erzeugt.

\begin{example}
  Man implementiert eine \texttt{main}-Methode der Klasse \texttt{Vector2D}
  \lstinputlisting[language=Java]{Prog/OOP\_Intro/Vector2D.java}
\end{example}