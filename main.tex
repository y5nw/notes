\documentclass[openany]{book}
\usepackage[T1]{fontenc}
\usepackage[a4paper,margin=1in]{geometry}
\usepackage{stix2}
\usepackage[ngerman]{babel}
\usepackage{xcolor}
\usepackage{units}
\usepackage{amsmath,amssymb,amsthm}
\usepackage{mathtools}
\usepackage{listings}
\usepackage{hyperref}
\usepackage{imakeidx}
\usepackage{cancel,ulem}
\usepackage{menukeys}
\usepackage{tabularx}
\usepackage{tikz}
\usetikzlibrary{datavisualization,datavisualization.formats.functions}

\lstset{
  basicstyle=\small\tt,
  columns=flexible,
  identifierstyle=\color{green!50!black},
  keywordstyle=\color{blue},
  numbers=left,
  numberstyle=\tiny,
  tabsize=4,
}

\title{Notizen}
\author{}
\date{}

\everymath{\displaystyle}

\def\phantomequal{\phantom{{}={}}}
\def\bar#1{\overline{#1}}
\def\hooktwoheadrightarrow{\hookrightarrow\mathrel{\mspace{-13mu}}\rightarrow}

\makeindex[intoc]
\makeindex[name=sym,title=Index Mathematischer Symbole,intoc]
\makeindex[name=java,title=Java-Index,intoc]
\begin{document}
\newtheorem{theorem}{Satz}[section]
\newtheorem{axiom}{Axiom}[section]
\newtheorem{lemma}{Lemma}[section]
\newtheorem{corollary}{Korollar}[section]
\theoremstyle{definition}
\newtheorem{example}{Beispiel}[section]
\newtheorem{definition}{Definition}[section]
\theoremstyle{remark}
\newtheorem*{remark}{Bemerkung}

\maketitle
\tableofcontents

\part{Grundbegriffe der Informatik}
% -*- TeX-master: "../main" -*-
\chapter{Mengen, Alphabete, Abbildungen}

Eine Menge\index{Menge} ist ein „Behälter“ von „Objekten“ (Elementen).

\begin{definition}
  \begin{align*}
    \emptyset &:= \{\} \tag{leere Menge\index{Menge!Leere Menge}} \\
    x \in A \cup B &:\iff x \in A \vee x \in B \tag{Vereinigung\index{Menge!Vereinigung}}\\
    x \in A \cap B &:\iff x \in A \wedge x \in B \tag{Durchschnitt\index{Menge!Durchschnitt}}\\
    A \subseteq B &:\iff x \in A \implies x \in B \tag{Teilmenge und Obermenge\index{Menge!Teilmenge}\index{Menge!Obermenge}}\\
    A = B &:\iff A \subseteq B \wedge B \subseteq A \tag{Gleichheit\index{Menge!Gleichheit}} \\
    A \subsetneq B &:\iff A \subseteq B \wedge A \ne B \tag{Echte Teilmenge\index{Menge!Echte Teilmenge}}
  \end{align*}
  Ist $A \cap B = \emptyset$, dann sind $A$ und $B$ disjunkte Mengen\index{Menge!Disjunkte Mengen}.
  \index[sym]{$\emptyset$}\index[sym]{$\cup$}\index[sym]{$\cap$}\index[sym]{$\subseteq$}\index[sym]{$\subsetneq$}\index[sym]{$=$}
\end{definition}
\begin{lemma}
  \begin{align*}
    A \cup A &= A & A \cap A &= A \tag{Idempotenzgesetz\index{Menge!Idempotenzgesetz}} \\
    A \cup B &= B \cup A & A \cap B &= B \cap A \tag{Kommutativgesetz\index{Menge!Kommutativgesetz}} \\
    (A \cup B) \cup C &= A \cup (B \cup C) & (A \cap B) \cap C &= A \cap (B \cap C) \tag{Assoziativgesetz\index{Menge!Assoziativgesetz}} \\
    A \cup (B \cap C) &= (A \cup B) \cap (A \cup C) & A \cap (B \cup C) &= (A \cap B) \cup (A \cap C) \tag{Distributivgesetz\index{Menge!Distributivgesetz}}
  \end{align*}
\end{lemma}

\part{Programmieren}
% -*- TeX-master: "../main" -*-
\chapter{Einführung}

\section{Einfache Programme}

\begin{example}[Algorithmus für Summation]
  \( \sum_1^n = \frac{n(n+1)}{2} \)
  \lstinputlisting[language=Java]{Prog/SimpleProgram/SimpleProgram.java}
\end{example}

\texttt{javac} Übersetzt Quellcode in Klassen. \texttt{java} führt das Programm aus. Dabei wird in der Klasse die Methode \texttt{main}\index{\texttt{main}} gestartet.

Compiler\index{Compiler}: Sie transformieren die Programm in maschinennahe Programme (oder Bytecode), die später ausgeführt werden. Der Quellcode wird in das Zielprogramm übersetzt.

Interpreter\index{Interpreter}: Sie übersetzen die Anweisungen und führen sie unmittelbar aus.

Java ist (anders als z.B. C) vom Plattform unabhängig: Programme werden in (vom Plattform unabhängigen) Java-Bytecode übersetzt, der vom Interpreter ausgeführt.

JIT (Just-In-Time) Compilation\index{Compiler!JIT-Compilation}: Findet während der Ausführung des Codes statt; Optimierungen

Wenn man den Quellcode ändern, muss man das Programm wieder compilieren.

\section{Objekten und Klassen}

Jedes Objekt der Realität hat ein virtuelles Gegenstück. Objekte kooperieren durch Datenaustausch (z.B. „$A$ ruft eine Methode von $B$ auf“). Objekten werden durch drei Aspekte charakterisiert:
\begin{itemize}
\item Identität: Objektnamen (oder: Speicheradresse)
\item Zustand: Attribute
\item Verhalten: Methoden
\end{itemize}

Was modelliert man bei einem Billiard-Spiel?
\begin{itemize}
\item Tisch: Größe
\item Bälle: Farbe, Gewicht, Position
\end{itemize}

Klasse: „Bauplan“ von Objekten. Sie legen fest, welche Attribute und Methoden die Objekt-Instanzen der Klassen haben können.

Operatoren werden mit dem \texttt{new}-Operator erzeugt.

\begin{example}
  Man implementiert eine \texttt{main}-Methode der Klasse \texttt{Vector2D}
  \lstinputlisting[language=Java]{Prog/OOP\_Intro/Vector2D.java}
\end{example}

\part{Lineare Algebra I}
% -*- TeX-master: "../main" -*-

\chapter{Sprache der Mathematik}

Sprache der Aussagelogik und Mengenlehre: es geht um Aussagen.

\begin{definition}[Aussage\index{Aussage}]
  Korrektes sprachliches Gebilde mit eindeutigem Wahrheitswert: wahr (w) oder falsch (f)
\end{definition}

\section{Kalkül von Aussagen}
Es seien $A, B, C, D, \dots $ Aussagen.

\begin{itemize}
\item $\neg A$: Negation von $A$: „$A$ ist falsch“\index{Aussage!Negation}\index[sym]{$\neg$}
\item $A \wedge B$: Kunjunktion: „$A$ und $B$ sind beide wahr“\index{Aussage!Konjunktion}\index[sym]{$\wedge$}
\item $A \vee B$: Disjunktion: „Mindestens eine von $A$ und $B$ ist wahr“\index{Aussage!Disjunktion}\index[sym]{$\vee$}
\item $C :\iff (A \wedge B) \vee (\neg A \wedge \neg B)$: Definition/Zuweisung von (z.B.) $C$
\item $A \implies B :\iff (\neg A) \vee B$: Implikation: „$A$ impliziert $B$“\index{Aussage!Implikation}\index[sym]{$\implies$}
\item $A \iff B :\iff (A \implies B) \wedge (B \implies A)$: Äquivalenz\index{Aussage!Äquivalenz}\index[sym]{$\iff$}
\end{itemize}

\begin{tabularx}{1.0\linewidth}{|X|X|X|X|X|X|X|X|}
  \hline
  $A$&$B$&$\neg A$&$A \wedge B$&$A \vee B$&$C$&$A \implies B$&$A \iff B$\\\hline
  w & w & f & w & w & w & w & w \\
  w & f & f & f & w & f & f & f \\
  f & w & w & f & w & f & w & f \\
  f & f & w & f & f & w & w & w \\\hline
\end{tabularx}

\begin{remark}
  $(A \iff B) \iff C$
\end{remark}

\begin{theorem}
  Eigeschaften:
  \begin{itemize}
  \item $A \implies A$
  \item Transitivität\index{Aussage!Transitivität}: $(A \implies B) \wedge (B \implies C) \implies (A \implies C)$
  \item Assoziativ\index{Aussage!Assoziativgesetz}: $(A \wedge B) \wedge C \iff A \wedge (B \wedge C);\; (A \vee B) \vee C \iff A \vee (B \vee C)$
  \item Kommutativ\index{Aussage!Kommutativgesetz}: $A \wedge B \iff B \wedge A;\; A \vee B \iff B \vee A$
  \item Distributiv\index{Aussage!Distributivgesetz}: $A \wedge (B \vee C) \iff (A \wedge B) \vee (A \wedge C);\; A \vee (B \wedge C) \iff (A \vee B) \wedge (A \vee C)$
  \item $\neg (A \wedge B) \iff (\neg A) \vee (\neg B);\; \neg (A \vee B) \iff (\neg A) \wedge (\neg B)$
  \item Doppelte Negation\index{Aussage!Doppelte Negation}: $\neg (\neg A) \iff A$
  \end{itemize}
\end{theorem}

\begin{remark}Binderegeln:
  \begin{itemize}
  \item Klammern (wenn möglich) weglassen! $A \wedge B$ statt $(A \wedge B)$, $A \wedge B \wedge C$ statt $(A \wedge B) \wedge C$
  \item $\neg$ vor $\wedge, \vee$ vor $\implies, \iff$. $A \wedge B \wedge \neg C \implies A \wedge \neg C$ statt $(((A \wedge B) \wedge (\neg C)) \implies (A \wedge \neg C))$
  \end{itemize}
\end{remark}

\begin{corollary}[Kontraposition\index{Aussage!Kontraposition}]
  $(A \implies B) \iff (\neg B \implies \neg A)$
\end{corollary}
\begin{proof}
  \begin{align*}
    (A \implies B) &\iff \neg A \vee B \\
    \, &\iff B \vee \neg A \\
    \, &\iff \neg (\neg B) \vee \neg A \\
    \, &\iff (\neg B \implies \neg A)
  \end{align*}
\end{proof}

Übung:
\begin{itemize}
\item $(A \implies B) \iff \neg(A \wedge \neg B)$ (z.B. Beweis durch Widerspruch)
\end{itemize}

\section{Sprache der Mengenlehre}
\begin{definition}[Mengenlehre\index{Menge}]
  Eine Menge $M$ ist eine Zusammenfassung von Objekten, so dass jedes „denkbare Objekt“ entweder in $M$ liegt oder nicht.
\end{definition}

\begin{itemize}
\item $x \in M$: „$x$ liegt in $M$“\index[sym]{$\in$}
\item $x \not\in M$: „ $x$ liegt nicht in $M$“
\item Definition von Mengen: $M := \left\{ 1, a, \alpha \right\}$
\end{itemize}

Zwei Mengen sind gleich, wenn sie dieselben Elemente haben: $\left\{ 1, a, \alpha \right\} = \left\{ a, 1, \alpha, a \right\}$

\begin{example}
  \begin{itemize}
  \item Natürliche Zahlen: $\mathbb{N} := \left\{ 1, 2, 3, 4, \dots \right\}$
  \item Ganze Zahlen: $\mathbb{Z} := \left\{ \dots, -2, -1, 0, 1, 2, \dots \right\}$
  \item „Menge aller Menschen“: $\left\{ x \middle| x \text{ ist Mensch} \right\}$
  \item „4 ist eine natürliche Zahl“: $4 \in \mathbb{N}$
  \item „Das Auto $A$ ist blau“: $A \in \left\{ x \middle| x \text{ ist blau} \right\}$
  \end{itemize}
\end{example}

\begin{definition}[Mengenoperationen]
  \begin{description}
  \item[Durchschnitt\index{Menge!Durchschnitt}] $M \cap N := \left\{ x \middle| x \in M \wedge x \in N \right\}$\index[sym]{$\cap$}
  \item[Vereinigung\index{Menge!Vereinigung}] $M \cup N := \left\{ x \middle| x \in M \vee x \in N \right\}$\index[sym]{$\cup$}
  \item[Differenz\index{Menge!Differenz}] $M \setminus N := \left\{ x \middle| x \in M \wedge x \not\in N \right\}$\index[sym]{$\setminus$}
  \item[Teilmenge\index{Menge!Teilmenge}] $M \subseteq N \iff (x \in M \implies x \in N)$\index[sym]{$\subseteq$}
  \item[Gleichheit\index{Menge!Gleichheit}] $M = N \iff (x \in M \iff x \in N) \iff (M \subseteq N \wedge N \subseteq M)$\index[sym]{$=$}
  \item[Echte Teilmenge\index{Menge!Echte Teilmenge}] $M \subsetneq N \iff M \subseteq N \wedge M \neq N$\index[sym]{$\subsetneq$}
  \end{description}
\end{definition}

\begin{theorem}
  Für Mengen $M$, $N$ und $O$ gilt (vgl. logische Operatoren):
  \begin{align*}
    M &\subseteq M\\
    (M \subseteq N \wedge N \subseteq O) &\implies M \subseteq O \tag{Transitivität\index{Menge!Transitivität}}\\
    (M \cap N) \cap O &= M \cap (N \cap O) & (M \cup N) \cup O &= M \cup (N \cup O)  \tag{Assoziativgesetz\index{Menge!Assoziativgesetz}} \\
    M \cap N &= N \cap M & M \cup N &= N \cup M \tag{Kommutativgesetz\index{Menge!Kommutativgesetz}} \\
    M \cap (N \cup O) &= (M \cap N) \cup (M \cap O) & M \cup (N \cap O) &= (M \cup N) \cap (M \cup O) \tag{Distributivgesetz\index{Menge!Distributivgesetz}} \\
    N \subseteq O &\implies M \cap N \subseteq M \cap O \\
    O \setminus (M \cap N) &= (O \setminus M) \cup (O \setminus N) & O \setminus (M \cup N) &= (O \setminus M) \cap (O \setminus N) \\
    O \setminus (O \setminus M) &= M \cap O
  \end{align*}
\end{theorem}

\begin{remark}
  \begin{itemize}
  \item Leere Menge: $\emptyset := \{\}$\index{Menge!Leere Menge}\index[sym]{$\emptyset$}
  \item Menge können andere Mengen als Elemente haben. $\left\{ \emptyset \right\}$ hat ein Element, $\left\{ \emptyset, \left\{ \emptyset \right\} \right\}$ hat zwei Elemente.
  \item Für $M:= \left\{ 1, \left\{ 2 \right\} \right\}$
    \begin{align*}
      1 &\in M & 2 &\not\in M \\
      \left\{ 1 \right\} &\subseteq M & \left\{ 2 \right\} &\not\subseteq M \\
      \left\{ 1 \right\} &\not\in M & \left\{ 2 \right\} &\in M \\
      \left\{ \left\{ 1 \right\} \right\} &\not\subseteq M & \left\{ \left\{ 2 \right\} \right\} &\subseteq M \\
    \end{align*}
  \end{itemize}
\end{remark}

\begin{definition}[Potenzmenge]
  $\mathscr{P}(M) := \left\{ N \middle| N \subseteq M \right\}$ ist die Potenzmenge\index{Menge!Potenzmenge}\index[sym]{P(M)@$\mathscr{P}(M)$} einer Menge $M$. $\mathscr{A} \subseteq \mathscr{P}(M)$ ist ein Mengensystem\index{Menge!Mengensystem}\index[sym]{A@$\mathscr{A}$} über $M$.
\end{definition}

\begin{remark}
Russell (1901): Es sei $R := \left\{ M \middle| M \text{ ist Menge} \wedge M \not\in M \right\}$. Ist $R$ in $R$?\index{Menge!Russels Paradox}
\begin{align*}
  R \in R &\implies R \not\in R \\
  R \not\in R &\implies R \in R
\end{align*}

Axiomatische Mengenlehre (grob zusammengefasst):
\begin{itemize}
\item Klasse: Zusammenfassungen von Objekten, die durch Eigenschaften beschrieben werden.
\item Mengen: Klassen, die selbst Element einer Klasse sind.
\end{itemize}
\end{remark}

\begin{definition}
  Ist $A(x)$ ein Ausdruck, für die man für $x$ Elemente von $M$ einsetzen kann, so heißt $A(x)$ eine Aussageform\index{Menge!Aussageform} über $M$. $x$ ist die freie Variable.
\end{definition}
\begin{remark}
  $\left\{ x \in M \middle| A(x) \right\} := \left\{ x \middle| x \in M \wedge A(x) \right\}$
\end{remark}
\begin{definition}
  Quantoren\index{Menge!Quantor}
  \begin{itemize}
  \item $\forall x \in M: A(x)$: Für alle $x \in M$ ist $A(x)$ wahr.\index[sym]{$\forall$}
  \item $\exists x \in M: A(x)$: Für mindestens ein $x \in M$ ist $A(x)$ wahr.\index[sym]{$\exists$}
  \item $\exists! x \in M: A(x)$: Für genau ein $x \in M$ ist $A(x)$ wahr.\index[sym]{$\exists"!"$}
  \end{itemize}
\end{definition}
\begin{remark}
  Negation von Quantoren:
  \begin{align*}
    \neg (\forall x: A(x)) &\iff \exists x: \neg A(x) \\
    \neg (\exists x: A(x)) &\iff \forall x: \neg A(x) \\
  \end{align*}
\end{remark}
\begin{definition}[Äquivalenz von Aussageformen\index{Aussage!Äquivalenz}]
  $(A(x) \iff B(x)) \iff (\forall x \in M: A(x) \iff B(x))$
\end{definition}
\begin{remark}
  Quantoren werden von links nach rechts gelesen. Reihenfolge beachten!
\end{remark}
\begin{definition}
  \begin{align*}
    \bigcap \mathscr{A} &:= \left\{ x \in M \middle| \forall A in \mathscr{A}: x \in A \right\} \\
    \bigcup \mathscr{A} &:= \left\{ x \in M \middle| \exists A in \mathscr{A}: x \in A \right\}
  \end{align*}
\end{definition}
\begin{definition}
  Sei $M$ eine endliche Menge, $|M|$ ist die Anzahl der Elemente von M.\index[sym]{$|M|$}
\end{definition}
\begin{remark}
  \begin{align*}
    |\mathscr{P}(M)| &= 2^{|M|} \\
    |M \cup N| &= |M| + |N| - |M \cap N| \\
  \end{align*}
\end{remark}

\section{Relationen und Funktionen}
\begin{definition}[Geordnetes Paar\index{Menge!Geordnetes Paar}]
  $(m, n) := \left\{ m, \left\{ m, n \right\} \right\}$
\end{definition}
\begin{definition}[Kartesisches Produkt\index{Menge!Kartesisches Produkt}]
  $M \times N := \left\{ (m, n) \middle| m \in M \wedge N \in N \right\}$\index[sym]{$\times$}
\end{definition}
\begin{definition}
  Eine Relation\index{Relation} $R = (M, N, G_R)$ besteht aus
  \begin{itemize}
  \item einer Definitionsmenge\index{Relation!Definitionsmenge} $M$
  \item einer Zielmenge\index{Relation!Zielmenge} $N$ und
  \item einer Menge $G_R \subset M \times N$ („Graph“)\index{Relation!Graph}
  \end{itemize}

  Relationen sind gleich, wenn die die gleiche Definitionsmengen, Zielmengen und Graphen haben
  
  Schreib- und Sprechweisen:
  \begin{itemize}
  \item „$m$ steht in $R$-Beziehung zu $n$“: $mRn \iff (m, n) \in G_R$
  \item „$R$ ist Relation von $M$ nach $N$“
  \item Ist $M=N$, so nennt man es eine Relation auf $M$.
  \end{itemize}
\end{definition}

\begin{definition}
  Eine Relation $R = (M, N, G_R)$ heißt eine Funktion\index{Funktion}, falls
  \[ \forall m \in M\, \exists! n \in N: (m, n) \in G_R \]
  Ist $m \in M$ und $(m, n) = G_R$, so schreibt man $n = R(m)$
  $n$ heißt das Bild\index{Funktion!Bild} von $m$ unter $R$.
\end{definition}
\begin{remark}
  Ist $f = (M, N, G_f)$ eine Funktion, so schreibt man $M \stackrel{f}{=} N$ oder $f: M \to N$ oder $m \mapsto f(m)$.\index[sym]{f(m)@$f(m)$}\index[sym]{$\to$}\index[sym]{$\mapsto$}

  Funktion ist bestimmt durch $M$, $N$ und die Zuordnung.
\end{remark}
\begin{definition}
  Es sei $f: M \to N$ eine Funktion.

  Ist $A \subseteq M$, dann ist $f[A] = \left\{ f(m) \middle| m \in A \right\}$ das Bild von $A$.\index{Funktion!Bild}\index[sym]{f[A]@$f[A]$}

  Ist $B \subseteq N$, dann ist $f^{-1}[B] = \left\{ m \in M \middle| f(m) \in B \right\}$ das Urbild von $B$.\index{Funktion!Urbild}\index[sym]{f-1[B]@$f^{-1}[B]$}

  Ist $n \in N$, dann ist $f^{-1}[n] = f[{n}]$ die Faser\index{Funktion!Faser} über $n$. Die Elemente von $f^{-1}[n]$ heißen Urbilder von $n$.\index[sym]{f-1[n]@$f^{-1}[n]$}
\end{definition}
\begin{definition}
  $f: M \to N, m \mapsto n$ heißt
  \begin{itemize}
  \item injektiv, wenn $\forall m_1, m_2 \in M: f(m_1) = f(m_2) \iff m_1 = m_2$ ($M \hookrightarrow N \implies \forall n \in N: |f^{-1}[n]| \le 1$)\index{Funktion!Injektiv}
  \item surjektiv, wenn $f[M] = N$ ($M \twoheadrightarrow N \implies \forall n \in N: |f^{-1}[n]| \ge 1$)\index{Funktion!Surjektiv}
  \item bijektiv, wenn $f$ injektiv und surjektiv ist. ($M \hooktwoheadrightarrow N \implies \forall n \in N: |f^{-1}[n]| = 1$)\index{Funktion!Bijektiv}
  \end{itemize}
\end{definition}
\begin{definition}
  Sei $A \subseteq M$, dann ist $f|_A: A \to N, a \mapsto f(a)$ Restriktion.\index{Funktion!Restriktion}\index[sym]{f"|"A@$f"|"_A$}

  Sei $f[M] \subseteq B \subseteq N$, dann ist $f|^B: M \to B, m \mapsto f(m)$ Korrestriktion.\index{Funktion!Korestriktion}\index[sym]{f"|"B@$f"|"^B$}
\end{definition}
\begin{remark}
  Ist $M \to N$ bijektiv und $n \in N$, dann $\exists! m \in M: f(m) = n$. Man nennt $f^{-1}: N \to M, n \mapsto m$ Umkehrfunktion von $f$.\index{Funktion!Umkehrfunktion}\index[sym]{f-1@$f^{-1}$}
\end{remark}

\begin{definition}
  Es sei $f: M \to N, g: N \to O$, dann ist $g \circ f: M \to O, m \mapsto g(f(m))$ eine Komposition\index{Funktion!Komposition}\index[sym]{$\circ$}. Man sagt „$g$ nach $f$“
\end{definition}
\begin{lemma}[Assoziativität der Komposition]
  Es seien $f: M \to N, g: N \to O, h: O \to P$. $h \circ (g \circ f) = (h \circ g) \circ f$
\end{lemma}
\begin{proof}
  Beides sind Funktionen von $M$ nach $P$ und
  \begin{align*}
    h \circ (g \circ f) (m) &= h (g \circ f (m)) \\
    \, &= h(g(f(m))) \\
    \, &= (h \circ g)(f(m)) \\
    \, &= (h \circ g) \circ f (m)
  \end{align*}
\end{proof}

\begin{remark}
  $\operatorname{Id}_M: M \to M, m \mapsto m$ heißt Identität\index{Funktion!Identität}\index[sym]{Id@$\operatorname{Id}_M$}. Dann gilt für $f: M \to N$:
  \begin{align*}
    f \circ \operatorname{Id}_M &= f \\
    \operatorname{Id}_M \circ f &= f
  \end{align*}
\end{remark}

\begin{theorem}
  Es sei $f: M \to N$ eine Funktion und $M \ne \emptyset$.
  \begin{itemize}
  \item Ist $f$ injektiv, dann $\exists g: N \to M: g \circ f = \operatorname{Id}_M$
  \item Ist $f$ surjektiv, dann $\exists h: N \to M: f \circ h = \operatorname{Id}_N$
  \item Ist $f$ bijektiv, dann $\exists g, h$ wie oben und $g = h = f^{-1}$ und $f \circ f = \operatorname{Id}_M$
  \end{itemize}
\end{theorem}
\begin{proof}
  Man nimmt an, $f$ ist injektiv.

  Man wählt $m_0 \in M$ (beachte: $M \ne \emptyset$). Ist $n \not\in \operatorname{Bild}(f)$, dann wählt man $g(n) = m$. Ist $n \in \operatorname{Bild}(f)$, wählt man $g(n) = m$.

  Dann ist $(g \circ f)(m) = g(f(m)) = m = \operatorname{Id}_M(m)$

  Sei $g$ wie oben. Für $m_1, m_2 \in M$ gilt
  \begin{align*}
    f(m_1) &= f(m_2) \\
    \implies m_1 = \operatorname{Id}_M(m_1) &= (g \circ f)(m_1) \\
    \, &= g(\underbrace{f(m_2)}_{= f(m_{1})}) = (g \circ f)(m_2) \\
    \, &= \operatorname{Id}_M(m_2) = m_2 \\
    \implies m_1 &= m_2
  \end{align*}

  $f: M \to N$

  Man nimmt $f(m) = n$ und $h(n) = m$. Dann ist
  \begin{align*}
    (f \circ h)(n) = \operatorname(Id)_N(n)
  \end{align*}
\end{proof}
\begin{remark}
  Seien $g$ („Linksinverse“) und $h$ („Rechtsinverse“) wie oben, dann ist:
  \begin{align*}
    \underbrace{(g \circ f)}_{= \operatorname{Id}_M} \circ h &= g \circ \underbrace{(f \circ h)}_{= \operatorname{Id}_N} \\
    \implies h &= g = f^{-1}
  \end{align*}
\end{remark}

\begin{definition}
  Es sei $I$ eine Menge und alle $X_i$ für $i \in I$ eine Menge.
  \[ \prod_{i \in I} X_i = \left\{ f: I \to \bigcup \left\{ X_i \middle| i \in I \right\} \middle| f(i) \in X \right\} \]
  heißt Produkt von $X_i$.
\end{definition}

\section{Partitionen}
\begin{definition}
  Es seien $M, N, \dots$ Mengen. $\mathscr{P} \subseteq \mathscr{P}(M)$

  $\mathscr{P}$ heißt paarweise disjunkt, wenn $\forall A, B \in P: A \cap B = \emptyset$. In dem Fall schreibt man $A \sqcup B$ statt $A \cup B$.

  $\mathscr{P}$ heißt Partition, wenn $\emptyset \not\in \mathscr{P}$ und $\bigsqcup \mathscr{P} = M$
\end{definition}
\begin{theorem}
  $\forall x \in M\: \exists! A \in \mathscr{P}: x \in A$
\end{theorem}
\begin{proof}
  Es ist $M = \bigsqcup P$. Man nimmt an, $x \in A \wedge x \in B$. Dann ist
  $A \cap B \ne \emptyset \implies$ Widerspruch.
\end{proof}
\begin{remark}
  Man schreibt $A = [x]_{\mathscr{P}}$
\end{remark}

\begin{definition}
  $\pi: M \to \mathscr{P}, x \mapsto [x]_{\mathscr{P}}$ heißt kanonische Projektion.

  Für $x, y \in M$ definiert man die $\mathscr{P}$-Äquivalenz $R_{\mathscr{P}}$ mit $x R_{\mathscr{P}} y :\iff [x]_{\mathscr{P}} = [y]_{\mathscr{P}}$
\end{definition}
\begin{remark}
  Es ist $p: M \twoheadrightarrow N$. Dann ist $\mathscr{P}_p = \left\{ p^{-1}[n] \middle| n \in N \right\}$ eine Faserpartition von $p$.

  Es gilt: $x R_p y \iff p(x) = p(y)$

  Ist $\pi: M \to \mathscr{P}$ die Partition, dann ist $\pi^{-1}[A] = A$, also $\mathscr{P} = \mathscr{P}_{\pi}$
\end{remark}
\begin{remark}
  $\pi: M \to \mathscr{P}$ ist surjektiv, dann ist $A \ne \emptyset$, also $\exists!x: x \in A$ und $\pi(x) = A$.

  Es gibt also eine Rechtsinverse $\sigma: \mathscr{P} \to M$ mit $\forall A \in \mathscr{P}: \sigma(A) = A$. Ist $R = \operatorname{Bild}(\sigma)$, dann ist $\forall A \in \mathscr{P}: R \cap A = \left\{ \sigma(A) \right\} \implies |R \cap A| = 1$
\end{remark}
\begin{remark}
  Sei $\mathscr{P}$ eine Partition und $R = R_{\mathscr{P}}$ die zugehörige Äquivalenzrelation, dann legt $R$ die Partition eindeutig fest:
  \begin{itemize}
  \item $\forall x \in M: [x]_{\mathscr{P}} = \left\{ y \middle| x R y \right\}$
  \item $\mathscr{P} = \left\{ [x]_{\mathscr{P}} \middle| x \in M \right\}$
  \end{itemize}
\end{remark}
\begin{proof}
  \begin{align*}
    y &\in [x]_{\mathscr{P}} \\
    \implies y &\in [x]_{\mathscr{P}} \cap [y]_{\mathscr{P}} \\
    \implies [x]_{\mathscr{P}} \cap [y]_{\mathscr{P}} &= \emptyset \\
    \implies [x]_{\mathscr{P}} &= [y]_{\mathscr{P}} \\
    \implies x &R y \\
    x &R y \\
    \implies [x]_{\mathscr{P}} &= y_{\mathscr{P}} \\
    \implies y &\in [x]_{\mathscr{P}}
  \end{align*}
  $\pi: M \to \mathscr{P}, x \mapsto [x]_{\mathscr{P}}$ ist surjektiv.
\end{proof}
\begin{remark}
  Notation:

  Es sei $R$ eine Relation auf $M$, dann ist $\forall x \in M: [x]_R = \left\{ y \middle| x R y \right\}$. Man schreibt

  $M / R = \left\{ [x]_R \middle| x \in M \right\}$ „$M$ modulo $B$“.

  Falls $R = R_{\mathscr{P}}$, dann ist $M/R = \mathscr{P}$.
\end{remark}
\begin{theorem}
  $M/R = \mathscr{P}$ genau dann, wenn $\forall x, y, z \in M$
  \begin{itemize}
  \item Reflexiv: $x R x$
  \item Symmetrisch: $x R y \implies y R x$
  \item Transitiv: $x R y \wedge y R z \implies x R z$
  \end{itemize}
  In dem Fall ist $R = R_{M/R}$
\end{theorem}
\begin{proof}
  $\Rightarrow$: Ist $M/R$ eine Partition, dann ist $\pi: M \to M/R, x \mapsto [x]_R$. Dann gilt $x R y \iff \pi(x) = \pi(y)$. Dann sind die o.g. Eigenschaften durch die Gleichheiterfüllt.

  $\Leftarrow$: Aus Reflexivität folge $x \in [x]_R$. Dann ist $\emptyset \not\in \mathscr{P}$ und $\bigcup \mathscr{P} \in M$.

  Angenommen, $z \in [x]_{\mathscr{P}} \cap [y]_{\mathscr{P}}$. Dann ist
  \begin{align*}
    (x R z) \wedge (y R z) &\implies (x R z) \wedge (z R y) \\
    \, &\implies x R y \\
    \, &\implies y \in [x]_{R} \\
    w \in [y]_R &\implies (y R w) \wedge (x R y) \\
    \, &\implies x R w \\
    \, &\implies w \in [x]_{\mathscr{P}}
  \end{align*}
  Also: $[y]_R \subseteq [x]_R, [x]_R \subseteq [y]_{R}$
\end{proof}

\begin{theorem}
  Jede Funktion $f: M \to N$ lässt sich kanonisch in eine surjektive, bijektive und injektive Abbildung zerlegen.
\end{theorem}

\part{Analysis I}
% -*- TeX-master: "../main" -*-

\chapter{Grundlagen}
\setcounter{section}{-1}

\section{Beispiele von logischen Schlussfolgerungen}

Natürliche Zahlen: \( \mathbb{N} = \left\{ 1, 2, \dots \right\} \)

$n$ ist gerade, wenn $k \in \mathbb{N}$ existiert mit $n = 2k$.

$n$ ist ungerade, wenn $k \in \mathbb{N}$ existiert mit $n = 2k-1$.

\begin{theorem}
  Sei $n \in \mathbb{N}$. $n$ ist genau dann gerade, wenn $n^2$ gerade ist.
\end{theorem}

\begin{proof}
  a. $n$ gerade $\Rightarrow$ $n^2$ gerade
  \begin{align*}
    n &= 2k \\
    n^2 &= (2k)^2 \\
    \, &= 2\underbrace{(2k^2)}_{\in \mathbb{N}} \\
    \, &\in \mathbb{N}
  \end{align*}

  b. $n^2$ gerade $\Rightarrow$ $n$ gerade: schwer zu beweisen

  c. $n$ ungerade $\Rightarrow$ $n^2$ ungerade
  \begin{align*}
    n &= 2k-1 \\
    n^2 &= (2k-1)^2 \\
    \, &= 4k^2-4k+1 \\
    \, &= 2\underbrace{(2k^2-2k+1}_{\in \mathbb{N}})-1 \\
    \, &\in \mathbb{N}
  \end{align*}

  Was hat c mit b zu tun?

  $p$: „$n$ ist gerade“; $q$: „$n^2$ ist gerade“.

  b. $p \Rightarrow q$; c. $\neg p \Rightarrow \neg q$ $\leftarrow$ Kontraposition zu b

  D.h. b ist genau dann wahr, wenn c wahr ist.

  c ist wahr $\Rightarrow$ b ist auch wahr.
\end{proof}

\begin{example}[Beweis durch Widerspruch]
  $\sqrt{2}$ ist irrational.

  \begin{proof}
    Annahme: $\sqrt{2}$ ist rational.

    Es sei \( A = \left\{ n \in \mathbb{N} \middle| \exists m \in \mathbb{Z}: \sqrt{2} = \frac{m}{n} \right\} \)

    $\sqrt{2}$ ist rational, wenn $A$ nicht leer ist. $A \subseteq \mathbb{N}$.

    Es sei $n_{*}: \forall n \in A: n \ge n_{*}$. Dann ist $\sqrt{2} = \frac{m}{n_{*}}$.

    \begin{align*}
      m-n_{*} &= \sqrt{2}n_{*} = \underbrace{\left( \sqrt{2}-1 \right)}_{< 1}n_{*} \\
      \, &< n_{*} \\
      m-n_{*} &\in \mathbb{N} \\
      \sqrt{2} &= \frac{m}{n_{*}} = \frac{m(m-n_{*})}{n(m-n_{*})} = \frac{m^2-mn_{*}}{n(m-n_{*})} \\
      \, &= \frac{2n_{*}^2-mn_{*}}{n(m-n_{*})} = \frac{2n_{*}-m}{m-n_{*}} \\
      \, &= \frac{\tilde{m}}{m-n_{*}}, \tilde{m} \in \mathbb{Z} \\
    \end{align*}
    Widerspruch: $m-n_{*} < n$
  \end{proof}

  Man kann damit auch zeigen, dass $\sqrt{3}$ irrational ist.

  Was ist mit $\sqrt{k}, k \in \mathbb{N}$?
\end{example}

\begin{theorem}
  Sei $k \in \mathbb{N}$. Dann ist entweder $\sqrt{k} \in \mathbb{N}$ oder $\sqrt{k}$ irrational.
\end{theorem}
\begin{proof}
  Annahme: $k \in \mathbb{Q} \setminus \mathbb{N}$

  Es sei \( A = \left\{ n \in \mathbb{N} \middle| \exists m \in \mathbb{Z}: \sqrt{k} = \frac{m}{n} \right\} \) (vgl. oben)

  $A$ hat ein kleinstes Element $n_{*}$ (vgl. oben)

  $k > 1 \implies \exists l \in \mathbb{N}: l < \sqrt{k} < l+1$

  \begin{align*}
    \sqrt{k} &= \frac{m}{n_{*}} \\
    \smash{\underbrace{m - ln_{*}}_{\in \mathbb{Z}}} &= \sqrt{k} n_{*} - ln_{*} \\
    \, &= \underbrace{(\sqrt{k} - l)}_{> 0} n_{*} > 0 \\
    \implies m - ln_{*} &\in \mathbb{N} \\
    m-ln_{*} &= \underbrace{(\sqrt{k} - l)}_{< 1} n < n \\
    \sqrt{k} = \frac{m}{n} &= \frac{m(m-ln_{*})}{n_{*}(m-ln_{*})} \\
    \, &= \frac{m^2 - lmn_*}{n_{*}(m-ln_{*})} \\
    \, &= \frac{kn^2 - lmn_{*}}{n_{*}(m-ln_{*}} \\
    \, &= \frac{\overbrace{kn - lm}^{\in \mathbb{Z}}}{m-ln_{*}} \\
    \implies m-ln_{*} &\in \mathbb{A}
  \end{align*}
  Widerspruch: $m-ln_{*} < n_{*}$
\end{proof}

\section{Aussagenlogik}

\begin{definition}
  Eine Aussage\index{Aussage} ist eine Behauptung, welche sprachlich oder durch eine Formel formuliert ist. Diese kann entweder wahr oder falsch sein. (Prinzip von ausgeschlossenen dritten)
\end{definition}
\begin{remark}
  Ein Beispiel beweist niemals eine Aussage. Ein Gegenbeispiel beweist hingegen, dass die Aussage falsch ist.
\end{remark}

\begin{definition}
  Es seien $p, q$ Aussagen.
  \begin{description}
  \item[Konjunktion] $p \wedge q$
  \item[Disjunktion] $p \vee q$
  \item[Implikation] $p \implies q$
  \item[Äquivalenz] $p \iff q$
  \item[Exklusives Oder] $(p \vee q) \wedge (\neg p \vee \neg q)$
  \end{description}
\end{definition}

\begin{definition}
  Aussagenform $H(x)$: Aussage mit Variable
\end{definition}
\begin{example}
  \begin{align*}
    H_1(x) &:\iff (x^2 - 3x + 2 = 0) \\
    H_2(x) &:\iff (x = 1 \vee x = 2) \\
    H_1(x) &\iff H_2(x)
  \end{align*}
\end{example}

\subsection{Beweisstruktur}
Es sei $p \implies q$. $p$ heißt die vorraussetzung, $q$ heißt die Behauptung.

Ein Beweis hat die Struktur von $p \implies r_1 \implies r_2 \implies \dots \implies r_n \implies q$. $r_1, \dots, r_n$ sind bereits bekannte wahre Aussagen oder Axiome.

\begin{theorem}
  Regeln der Aussagenlogik:
  \begin{align*}
    A &\implies A \\
    (A \implies B) \wedge (B \implies C) &\implies (A \implies C) \tag{Transitivität} \\
    (A \wedge B) \wedge C &\iff A \wedge B \wedge C & (A \vee B) \vee C &\iff A \vee B \vee C \tag{Assoziativität} \\
    A \wedge B &\iff B \wedge A & A \vee B &\iff B \vee A \tag{Kommutativität} \\
    A \wedge (B \vee C) &\iff (A \wedge B) \vee (A \wedge C) & A \vee (B \wedge C) &\iff (A \vee B) \wedge (A \vee C) \tag{Distributivität} \\
    (B \implies C) &\implies (A \wedge B \implies A \wedge C) \tag{Monotonie} \\
    \neg (A \wedge B) &\iff \neg A \vee \neg B & \neg (A \vee B) &\iff \neg A \wedge \neg B \tag{Morgan'sche Regeln} \\
    \neg (\neg A) &\iff A \tag{Doppelte Negation}
  \end{align*}
\end{theorem}

\subsection{Mengen}
Nach Cantor ist eine Menge $M$ eine Zusammenfassung bestimmter, wohlunterschiedener Objekte unserer Anschauung oder unseres Denkens (welche die Elemente von $M$ genannt werden) zu einem Ganzen.

\begin{example}
  \begin{align*}
    A &:= {M, A, T, H, E, M, A, T, I, K} \\
    \, &= {M, A, H, T, E, A, I, K} \\
    \, &= {T, H, E, M, A, T, I, K} \\
  \end{align*}
\end{example}

Man schreibt $x \in A$, wenn $A$ eine Menge ist und $x$ ein Element von $A$ ist. Ist $x$ kein Element von $A$, so schreibt man $x \not\in A$.

Ist $H(x)$ eine Aussage, die von einer Variable $x$ abhängig ist, dann gibt es eine Menge $A := \{x|H(x)\}$. $x \in A \iff H(x)$.

\begin{definition}
  \begin{description}
  \item[Gleichheit] Zwei Mengen $A$ und $B$ sind gleich, wenn sie dieselben Elemente enthalten.
  \item[Leere Menge] Die leere Menge $\emptyset := \{\}$ ist die eindeutige Menge, welche kein Element enthält.
  \item[Teilmenge] $A \subseteq B \iff \forall x \in A: x \in B$
  \item[Echte Teilmenge] $A \subsetneq B \iff A \subseteq B \wedge A \ne B$
  \item[Disjunkte Mengen] $\forall x \in A: x \not\in B$
  \end{description}
\end{definition}
\begin{remark}
  $A = B \iff A \subseteq B \wedge B \subseteq A$
\end{remark}

\begin{definition}
  Operationen mit Mengen:
  \begin{description}
  \item[Durchschnitt] $A \cap B := \left\{ x \middle| x \in A \wedge x \in B \right\}$
  \item[Vereinigung] $A \cup B := \left\{ x \middle| x \in A \vee x \in B \right\}$
  \item[Differenz] $A \setminus B := \left\{ x \in A \wedge x \not\in B \right\}$
  \item[Komplement] Für $A \subseteq M: A^C = A^C_M = M \setminus A$
  \end{description}
\end{definition}

\backmatter
\part{Appendices}
\printindex
\printindex[sym]
\printindex[java]

\end{document}