\documentclass[openany]{book}
\usepackage[T1]{fontenc}
\usepackage[a4paper,margin=1in]{geometry}
\usepackage{stix2}
\usepackage[ngerman]{babel}
\usepackage{xcolor}
\usepackage{units}
\usepackage{amsmath,amssymb,amsthm}
\usepackage{mathtools}
\usepackage{bussproofs}
\usepackage{listings}
\usepackage{imakeidx}
\usepackage{hyperref}
\usepackage{cancel,ulem}
\usepackage{menukeys}
\usepackage{tabularx}
\usepackage{tikz}
\usetikzlibrary{datavisualization,datavisualization.formats.functions}

\lstset{
  basicstyle=\small\tt,
  columns=flexible,
  identifierstyle=\color{green!50!black},
  keywordstyle=\color{blue},
  numbers=left,
  numberstyle=\tiny,
  tabsize=4,
}

\title{Notizen}
\author{}
\date{}

\everymath{\displaystyle}

\def\phantomequal{\phantom{{}={}}}
\def\bar#1{\overline{#1}}
\def\hooktwoheadrightarrow{\hookrightarrow\mathrel{\mspace{-13mu}}\rightarrow}
\def\consttrue{\textsc{wahr}}
\def\constfalse{\textsc{falsch}}
\def\textstring#1{\texttt{\color{blue}#1}}
\def\mathstring#1{\ensuremath{\mathtt{\color{blue}#1}}}

\makeindex[intoc]
\makeindex[name=sym,title=Index Mathematischer Symbole,intoc]
\makeindex[name=java,title=Java-Index,intoc]
\begin{document}
\newtheorem{theorem}{Satz}[section]
\newtheorem{axiom}[theorem]{Axiom}
\newtheorem{lemma}[theorem]{Lemma}
\newtheorem{corollary}[theorem]{Korollar}
\newtheorem{proposition}[theorem]{Proposition}
\theoremstyle{definition}
\newtheorem*{example}{Beispiel}
\newtheorem*{problem}{Problem}
\newtheorem{definition}[theorem]{Definition}
\newtheorem*{notation}{Schreibweise}
\theoremstyle{remark}
\newtheorem*{remark}{Bemerkung}
\newtheorem*{observation}{Beobachtung}

\maketitle
\tableofcontents

\part{Grundbegriffe der Informatik}
% -*- TeX-master: "../main" -*-
\chapter{Mengen, Alphabete, Abbildungen}

Eine Menge\index{Menge} ist ein „Behälter“ von „Objekten“ (Elementen).

\begin{definition}
  \begin{align*}
    \emptyset &:= \{\} \tag{leere Menge\index{Menge!Leere Menge}} \\
    x \in A \cup B &:\iff x \in A \vee x \in B \tag{Vereinigung\index{Menge!Vereinigung}}\\
    x \in A \cap B &:\iff x \in A \wedge x \in B \tag{Durchschnitt\index{Menge!Durchschnitt}}\\
    A \subseteq B &:\iff x \in A \implies x \in B \tag{Teilmenge und Obermenge\index{Menge!Teilmenge}\index{Menge!Obermenge}}\\
    A = B &:\iff A \subseteq B \wedge B \subseteq A \tag{Gleichheit\index{Menge!Gleichheit}} \\
    A \subsetneq B &:\iff A \subseteq B \wedge A \ne B \tag{Echte Teilmenge\index{Menge!Echte Teilmenge}}
  \end{align*}
  Ist $A \cap B = \emptyset$, dann sind $A$ und $B$ disjunkte Mengen\index{Menge!Disjunkte Mengen}.
  \index[sym]{$\emptyset$}\index[sym]{$\cup$}\index[sym]{$\cap$}\index[sym]{$\subseteq$}\index[sym]{$\subsetneq$}\index[sym]{$=$}
\end{definition}
\begin{lemma}
  \begin{align*}
    A \cup A &= A & A \cap A &= A \tag{Idempotenzgesetz\index{Menge!Idempotenzgesetz}} \\
    A \cup B &= B \cup A & A \cap B &= B \cap A \tag{Kommutativgesetz\index{Menge!Kommutativgesetz}} \\
    (A \cup B) \cup C &= A \cup (B \cup C) & (A \cap B) \cap C &= A \cap (B \cap C) \tag{Assoziativgesetz\index{Menge!Assoziativgesetz}} \\
    A \cup (B \cap C) &= (A \cup B) \cap (A \cup C) & A \cap (B \cup C) &= (A \cap B) \cup (A \cap C) \tag{Distributivgesetz\index{Menge!Distributivgesetz}}
  \end{align*}
\end{lemma}
% -*- TeX-master: "../main" -*-

\chapter{Wörter}

\section{Definition von Wörtern}
\begin{definition}[Alphabet]
  Ein Alphabet $A$ ist eine nichtleere Menge von Zeichen oder Symbolen.
\end{definition}

\begin{definition}
  Es sei $\mathbb{Z}_n = \left\{ i \in \mathbb{N}_0 \middle| 0 \le i \wedge i < n \right\}$ (Achtung: $\mathbb{Z}_0 = \emptyset$). Ein Wort (über ein Alphabet $A$) ist eine Abbildung: $w: \mathbb{Z}_n \to A$. $|w| = n$ heißt die Länge des Wortes.
\end{definition}
\begin{remark}[Das leere Wort]
  $\varepsilon: \mathbb{Z}_0 \to A$ oder $\varepsilon: \emptyset \to A$. Als relation $R$ gilt: $R \subseteq \emptyset \times A = \emptyset$. Das leere Wort ist also (unabhängig vom Alphabet) eindeutig.
\end{remark}
% -*- TeX-master: "../main" -*-
\chapter{Aussagenlogik}

\section{Syntax aussagenlogischer Formeln}
\begin{definition}[Alphabet aussagenlogischer Formeln]
  $A_{AL} = \{ \textstring{(}, \textstring{)}, \mathstring{\neg}, \mathstring{\wedge}, \mathstring{\vee}, \mathstring{\to}, \mathstring{\leftrightarrow} \} \cup \textit{Var}_{AL}$
\end{definition}

\begin{definition}
  Aussagenlogische Konnektive/Verknüpfung:
  \begin{description}
  \item[Negation] $\textstring{(} \mathstring{\neg} G \textstring{)}$
  \item[Konjunktion] $\textstring{(} G \mathstring{\wedge} H \textstring{)}$
  \item[Disjunktion] $\textstring{(} G \mathstring{\vee} H \textstring{)}$
  \item[Implikation] $\textstring{(} G \mathstring{\to} H \textstring{)}$
  \end{description}
\end{definition}
\begin{remark}[Konstruktionsabbildungen]
  Für jede Verknüpfung gibt es eine Abbildung. Z.B.: $f_{\mathstring{\wedge}}: A_{AL}^{*} \times A_{AL}^{*} \to A_{AL}^{*}: (G, H) \to \textstring{(} G \mathstring{\wedge} H \textstring{)}$
\end{remark}
\begin{remark}[Menge einmaliger Verknüpfungen]
  $\operatorname{Ver}(M) := \left\{ \textstring{(} \mathstring{\neg} G \textstring{)}, \textstring{(} G \mathstring{\wedge} H \textstring{)}, \textstring{(} G \mathstring{\vee} H \textstring{)}, \textstring{(} G \mathstring{\to} H \textstring{)} \middle| G, G \in \textit{Var}_{AL} \right\}$
\end{remark}

\begin{definition}[Aussagenlogische Variablen $\textit{Var}_{AL}$]
  Aussagenlogische Variablen (oder: Atome) werden oft mit $\textstring{P}_i, i \in \mathbb{N}_0$ oder $\textstring{P}, \textstring{Q}, \textstring{R}, \dots$ benannt.
\end{definition}

\begin{definition}[Formale Sprache $\textit{For}_{AL}$ der Aussagenlogik]
  \begin{align*}
    M_0 &:= \textit{Var}_{AL} \\
    \forall n \in \mathbb{N}_0: M_{n+1} &:= M_n \cup \operatorname{Ver}(M_n) \\
    For_{AL} &:= \bigcup_{n \in \mathbb{N}_0} M_n
  \end{align*}
\end{definition}

\begin{notation}
  Abkürzungen (Der „offiziele“ Syntax bleibt gleich):
  \begin{itemize}
  \item Man kann die äußersten Klammern weglassen
  \item mehrfach gleiches Konnektiv: „implizierte Linksklammerung“
  \item Bindestärke von Konnektiven: (stärksten) \mathstring{\neg}, \mathstring{\wedge}, \mathstring{\vee}, \mathstring{\to}, \mathstring{\leftrightarrow} (schwächsten)
  \item $\textstring{(} G \mathstring{\leftrightarrow} H \textstring{)}$ steht für $\textstring{(} \textstring{(} G \mathstring{\to} H \textstring{)} \mathstring{\wedge} \textstring{(} H \mathstring{\to} G \textstring{)} \textstring{)}$
  \end{itemize}
\end{notation}

\section{Semantik der Aussagenlogik}
Aussagen habe genau zwei Wahrheitswerte: „wahr“ (\consttrue) oder „falsch“ (\constfalse).
\begin{definition}
  $\mathbb{B} := \left\{ \consttrue, \constfalse \right\}$
\end{definition}
\begin{definition}[Interpretation]
  $I: V \to \mathbb{B}, V \subseteq \textit{Var}_{AL}$
\end{definition}
\begin{definition}[Auswertung von Formeln]
  \begin{align*}
    \forall G \in V: \operatorname{val}_I(G) &:= I(G) \\
    \operatorname{val}_I(\mathstring{\neg} G) &:=
                                             \begin{cases}
                                               \consttrue & I(G) = \constfalse \\
                                               \constfalse & \text{sonst}
                                             \end{cases} \\
    \operatorname{val}_I(G \mathstring{\wedge} H) &:=
                                               \begin{cases}
                                                 \consttrue & I(G) = \consttrue \text{ und } I(H) = \consttrue \\
                                                 \constfalse & \text{sonst}
                                               \end{cases} \\
    \operatorname{val}_I(G \mathstring{\vee} H) &:=
                                               \begin{cases}
                                                 \consttrue & I(G) = \consttrue \text{ oder } I(H) = \consttrue \\
                                                 \constfalse & \text{sonst}
                                               \end{cases} \\
    \operatorname{val}_I(G \mathstring{\to} H) &:=
                                               \begin{cases}
                                                 \consttrue & \text{wenn } I(G) = \consttrue \text{ dann } I(H) = \consttrue \\
                                                 \constfalse & I(G) = \consttrue \text{ und } I(H) = \constfalse
                                               \end{cases}
  \end{align*}
\end{definition}
\begin{remark}
  Das „Oder“ ist inklusiv.
\end{remark}

\begin{definition}
  Wenn zwei Aussagen $G$ und $H$ für alle Interpretationen den gleichen Wahrheitswerten annehmen, dann sind sie äquivalent. Man schreibt $G \equiv H$
\end{definition}

\begin{definition}[Modell]
  Ist $\operatorname{val}_I(G) = \consttrue$, dann nennt man $I$ ein Modell von $G$.

  Ist $I$ ein Modell jeder Formel $G \in \Gamma$, dann nennt man $I$ ein Modell von $\Gamma$.
\end{definition}

\begin{notation}
  Es sei $\Gamma$ eine Menge von Formeln und $G$ eine Formel. Ist jedes Modell von $\Gamma$ auch Modell von $G$, so schreibt man $\Gamma \models G$.

  Man schreibt $H \models G$ statt $\left\{ H \right\} \models G$ und $\models G$ statt $\emptyset \models G$.
\end{notation}
\begin{definition}
  Ist $\models G$, dann ist $G$ für alle Interpretationen wahr. $G$ heißt eine Tautologie oder eine allgemeingültige Formel.
\end{definition}
\begin{lemma}
  $G \equiv H$ gilt genau dann, wenn $G \leftrightarrow H$ Tautologie ist.
\end{lemma}
\begin{definition}
  $G$ heißt erfüllbar, wenn $G$ für mindestens eine Interpretation wahr ist.
\end{definition}

\section{Beweisbarkeit}
\begin{notation}[Logische Schlussregeln]
  Man schreibt
  \begin{prooftree}
    \AxiomC{$V_1$}
    \AxiomC{$V_2$}
    \AxiomC{$\cdots$}
    \AxiomC{$V_n$}
    \QuaternaryInfC{$C$}
  \end{prooftree}
  $V_i$ heißen Vorraussetzungen, $C$ heißt Folgerung („conclusio“).

  Eine Regel heißt korrekt, wenn $\left\{ V_1, V_2, \dots, V_n \right\} \models C$
\end{notation}
\begin{example}
  \AxiomC{$A \wedge B$}
  \UnaryInfC{$A$}
  \DisplayProof
\end{example}
\begin{example}[Modus ponens]
  \AxiomC{$A$}
  \AxiomC{$A \to B$}
  \BinaryInfC{$B$}
  \DisplayProof
\end{example}
\begin{example}[Aussagen mit Bedingungen]
  \AxiomC{$\Gamma_1 \models V_1$}
  \AxiomC{$\Gamma_2 \models V_2$}
  \AxiomC{$\cdots$}
  \AxiomC{$\Gamma_n \models V_n$}
  \QuaternaryInfC{$\scriptstyle \left( \bigcup_{i=1}^n \Gamma_i \right) \models C$}
  \DisplayProof
\end{example}

\begin{example}
  Zu zeigen: $\models (P \vee \neg P)$
  \begin{prooftree}
    \AxiomC{}
    \UnaryInfC{$\neg (P \vee \neg P) \models \neg (P \vee \neg P)$}
    \AxiomC{}
    \UnaryInfC{$\neg (P \vee \neg P) \models \neg (P \vee \neg P)$}
    \AxiomC{}
    \UnaryInf$P \ \fCenter \models P$
    \UnaryInf$P \ \fCenter \models P \vee \neg P$
    \BinaryInf$\neg (P \vee \neg P), P \ \fCenter \models \constfalse$
    \UnaryInf$\neg (P \vee \neg P) \ \fCenter \models \neg P$
    \UnaryInf$\neg (P \vee \neg P) \ \fCenter \models P \vee \neg P$
    \BinaryInfC{$\neg (P \vee \neg P) \ \fCenter \models \constfalse$}
    \UnaryInfC{$P \vee \neg P$}
  \end{prooftree}
\end{example}

\part{Programmieren}
% -*- TeX-master: "../main" -*-
\chapter{Einführung}

\section{Einfache Programme}

\begin{example}[Algorithmus für Summation]
  \( \sum_1^n = \frac{n(n+1)}{2} \)
  \lstinputlisting[language=Java]{Prog/SimpleProgram/SimpleProgram.java}
\end{example}

\texttt{javac} Übersetzt Quellcode in Klassen. \texttt{java} führt das Programm aus. Dabei wird in der Klasse die Methode \texttt{main}\index{\texttt{main}} gestartet.

Compiler\index{Compiler}: Sie transformieren die Programm in maschinennahe Programme (oder Bytecode), die später ausgeführt werden. Der Quellcode wird in das Zielprogramm übersetzt.

Interpreter\index{Interpreter}: Sie übersetzen die Anweisungen und führen sie unmittelbar aus.

Java ist (anders als z.B. C) vom Plattform unabhängig: Programme werden in (vom Plattform unabhängigen) Java-Bytecode übersetzt, der vom Interpreter ausgeführt.

JIT (Just-In-Time) Compilation\index{JIT-Compilation}: Findet während der Ausführung des Codes statt; Optimierungen

Wenn man den Quellcode ändern, muss man das Programm wieder compilieren.

\section{Objekten und Klassen}

Jedes Objekt der Realität hat ein virtuelles Gegenstück. Objekte kooperieren durch Datenaustausch (z.B. „$A$ ruft eine Methode von $B$ auf“). Objekten werden durch drei Aspekte charakterisiert:
\begin{itemize}
\item Identität: Objektnamen (oder: Speicheradresse)
\item Zustand: Attribute
\item Verhalten: Methoden
\end{itemize}

Was modelliert man bei einem Billiard-Spiel?
\begin{itemize}
\item Tisch: Größe
\item Bälle: Farbe, Gewicht, Position
\end{itemize}

Klasse: „Bauplan“ von Objekten. Sie legen fest, welche Attribute und Methoden die Objekt-Instanzen der Klassen haben können.

Operatoren werden mit dem \texttt{new}-Operator erzeugt.

\begin{example}
  Man implementiert eine \texttt{main}-Methode der Klasse \texttt{Vector2D}
  \lstinputlisting[language=Java]{Prog/OOP\_Intro/Vector2D.java}
\end{example}
% -*- TeX-master: "../main" -*-
\chapter{Datentypen}

\section{Elementare Datentypen und Operationen}
\begin{definition}
  Elementare Datentypen:
  \begin{description}
  \item[\texttt{boolean}] Wahrheitswerte: \texttt{true}, \texttt{false}\index[java]{boolean@\texttt{boolean}}\index[java]{true@\texttt{true}}\index[java]{false@\texttt{false}}
  \item[\texttt{char}] 16-Bit-Unicode-Zeichen\index[java]{char@\texttt{char}}
  \item[\texttt{byte}] 8-Bit-Ganzzahl\index[java]{byte@\texttt{byte}}
  \item[\texttt{short}] 16-Bit-Ganzzahl\index[java]{short@\texttt{short}}
  \item[\texttt{int}] 32-Bit-Ganzzahl\index[java]{int@\texttt{int}}
  \item[\texttt{long}] 64-Bit-Ganzzahl (beachte: \texttt{12L} statt \texttt{12})\index[java]{long@\texttt{long}}
  \item[\texttt{float}] 32-Bit-Gleitpunktzahl (beachte: \texttt{9.81F} statt \texttt{9.8})\index[java]{float@\texttt{float}}
  \item[\texttt{double}] 64-Bit-Gleitpunktzahl\index[java]{double@\texttt{double}}
  \end{description}
\end{definition}
\begin{definition}
  Operationen auf elementaren Datentypen:
  \begin{enumerate}
  \item \texttt{+x}, \texttt{-x}, \texttt{\^{}x}, \texttt{!x}
  \item \texttt{x*y}, \texttt{x/y}, \texttt{x\%y}
  \item \texttt{x+y}, \texttt{x-y}
  \item \texttt{x<{}<y}, \texttt{x>{}>y}, \texttt{x>{}>{}>y}
  \item \texttt{<}, \texttt{<=}, \texttt{>}, \texttt{>=}
  \item \texttt{x==y}, \texttt{x!=y}
  \item \texttt{x\&y}
  \item \texttt{x\^{}y}
  \item \texttt{x|y}
  \item \texttt{x\&\&y}
  \item \texttt{x||y}
  \end{enumerate}
\end{definition}
\begin{remark}[Gleitkommazahlen]
  Die Zahlen sind nicht gleichmäßig dicht dargestellt. Dies können zu Rundungsfehlern führen.
\end{remark}

\section{Der Datentyp \texttt{String}}
Man verwendet \texttt{String}\index[java]{String@\texttt{String}} für Zeichenketten.
\begin{lstlisting}[language=Java]
String text = "Hallo";
String address = "Am Fasanengarten 5\n76131 Karlsruhe";
\end{lstlisting}
\begin{remark}
  Konkatenation von \texttt{String}s ist durch \texttt{+} möglich.
\end{remark}

\section{\texttt{enum}}
\index[java]{enum@\texttt{enum}}
\begin{lstlisting}[language=Java]
enum Color {RED, GREEN, BLUE}
\end{lstlisting}
\begin{remark}
  \texttt{enum}-Operationen müssen selbst definiert werden.
\end{remark}

\section{Objekt-Variablen}
Objekt-Variablen sind Referenzen, d.h., Speicheradressen. \texttt{null}\index[java]{null@\texttt{null}} steht für „kein Objekt“.

\section{Konstanten}
\begin{lstlisting}[language=Java]
static final float PI = 3.14159265f;
\end{lstlisting}

\part{Lineare Algebra I}
% -*- TeX-master: "../main" -*-

\chapter{Sprache der Mathematik}

Sprache der Aussagelogik und Mengenlehre: es geht um Aussagen.

\begin{definition}[Aussage\index{Aussage}]
  Korrektes sprachliches Gebilde mit eindeutigem Wahrheitswert: wahr (w) oder falsch (f)
\end{definition}

\section{Kalkül von Aussagen}
Es seien $A, B, C, D, \dots $ Aussagen.

\begin{itemize}
\item $\neg A$: Negation von $A$: „$A$ ist falsch“\index{Aussage!Negation}\index[sym]{$\neg$}
\item $A \wedge B$: Kunjunktion: „$A$ und $B$ sind beide wahr“\index{Aussage!Konjunktion}\index[sym]{$\wedge$}
\item $A \vee B$: Disjunktion: „Mindestens eine von $A$ und $B$ ist wahr“\index{Aussage!Disjunktion}\index[sym]{$\vee$}
\item $C :\iff (A \wedge B) \vee (\neg A \wedge \neg B)$: Definition/Zuweisung von (z.B.) $C$
\item $A \implies B :\iff (\neg A) \vee B$: Implikation: „$A$ impliziert $B$“\index{Aussage!Implikation}\index[sym]{$\implies$}
\item $A \iff B :\iff (A \implies B) \wedge (B \implies A)$: Äquivalenz\index{Aussage!Äquivalenz}\index[sym]{$\iff$}
\end{itemize}

\begin{tabularx}{1.0\linewidth}{|X|X|X|X|X|X|X|X|}
  \hline
  $A$&$B$&$\neg A$&$A \wedge B$&$A \vee B$&$C$&$A \implies B$&$A \iff B$\\\hline
  w & w & f & w & w & w & w & w \\
  w & f & f & f & w & f & f & f \\
  f & w & w & f & w & f & w & f \\
  f & f & w & f & f & w & w & w \\\hline
\end{tabularx}

\begin{remark}
  $(A \iff B) \iff C$
\end{remark}

\begin{theorem}
  Eigeschaften:
  \begin{itemize}
  \item $A \implies A$
  \item Transitivität\index{Aussage!Transitivität}: $(A \implies B) \wedge (B \implies C) \implies (A \implies C)$
  \item Assoziativ\index{Aussage!Assoziativgesetz}: $(A \wedge B) \wedge C \iff A \wedge (B \wedge C);\; (A \vee B) \vee C \iff A \vee (B \vee C)$
  \item Kommutativ\index{Aussage!Kommutativgesetz}: $A \wedge B \iff B \wedge A;\; A \vee B \iff B \vee A$
  \item Distributiv\index{Aussage!Distributivgesetz}: $A \wedge (B \vee C) \iff (A \wedge B) \vee (A \wedge C);\; A \vee (B \wedge C) \iff (A \vee B) \wedge (A \vee C)$
  \item $\neg (A \wedge B) \iff (\neg A) \vee (\neg B);\; \neg (A \vee B) \iff (\neg A) \wedge (\neg B)$
  \item Doppelte Negation\index{Aussage!Doppelte Negation}: $\neg (\neg A) \iff A$
  \end{itemize}
\end{theorem}

\begin{remark}Binderegeln:
  \begin{itemize}
  \item Klammern (wenn möglich) weglassen! $A \wedge B$ statt $(A \wedge B)$, $A \wedge B \wedge C$ statt $(A \wedge B) \wedge C$
  \item $\neg$ vor $\wedge, \vee$ vor $\implies, \iff$. $A \wedge B \wedge \neg C \implies A \wedge \neg C$ statt $(((A \wedge B) \wedge (\neg C)) \implies (A \wedge \neg C))$
  \end{itemize}
\end{remark}

\begin{corollary}[Kontraposition\index{Aussage!Kontraposition}]
  $(A \implies B) \iff (\neg B \implies \neg A)$
\end{corollary}
\begin{proof}
  \begin{align*}
    (A \implies B) &\iff \neg A \vee B \\
    \, &\iff B \vee \neg A \\
    \, &\iff \neg (\neg B) \vee \neg A \\
    \, &\iff (\neg B \implies \neg A)
  \end{align*}
\end{proof}

Übung:
\begin{itemize}
\item $(A \implies B) \iff \neg(A \wedge \neg B)$ (z.B. Beweis durch Widerspruch)
\end{itemize}

\section{Sprache der Mengenlehre}
\begin{definition}[Mengenlehre\index{Menge}]
  Eine Menge $M$ ist eine Zusammenfassung von Objekten, so dass jedes „denkbare Objekt“ entweder in $M$ liegt oder nicht.
\end{definition}

\begin{itemize}
\item $x \in M$: „$x$ liegt in $M$“\index[sym]{$\in$}
\item $x \not\in M$: „ $x$ liegt nicht in $M$“
\item Definition von Mengen: $M := \left\{ 1, a, \alpha \right\}$
\end{itemize}

Zwei Mengen sind gleich, wenn sie dieselben Elemente haben: $\left\{ 1, a, \alpha \right\} = \left\{ a, 1, \alpha, a \right\}$

\begin{example}
  \begin{itemize}
  \item Natürliche Zahlen: $\mathbb{N} := \left\{ 1, 2, 3, 4, \dots \right\}$
  \item Ganze Zahlen: $\mathbb{Z} := \left\{ \dots, -2, -1, 0, 1, 2, \dots \right\}$
  \item „Menge aller Menschen“: $\left\{ x \middle| x \text{ ist Mensch} \right\}$
  \item „4 ist eine natürliche Zahl“: $4 \in \mathbb{N}$
  \item „Das Auto $A$ ist blau“: $A \in \left\{ x \middle| x \text{ ist blau} \right\}$
  \end{itemize}
\end{example}

\begin{definition}[Mengenoperationen]
  \begin{description}
  \item[Durchschnitt\index{Menge!Durchschnitt}] $M \cap N := \left\{ x \middle| x \in M \wedge x \in N \right\}$\index[sym]{$\cap$}
  \item[Vereinigung\index{Menge!Vereinigung}] $M \cup N := \left\{ x \middle| x \in M \vee x \in N \right\}$\index[sym]{$\cup$}
  \item[Differenz\index{Menge!Differenz}] $M \setminus N := \left\{ x \middle| x \in M \wedge x \not\in N \right\}$\index[sym]{$\setminus$}
  \item[Teilmenge\index{Menge!Teilmenge}] $M \subseteq N \iff (x \in M \implies x \in N)$\index[sym]{$\subseteq$}
  \item[Gleichheit\index{Menge!Gleichheit}] $M = N \iff (x \in M \iff x \in N) \iff (M \subseteq N \wedge N \subseteq M)$\index[sym]{$=$}
  \item[Echte Teilmenge\index{Menge!Echte Teilmenge}] $M \subsetneq N \iff M \subseteq N \wedge M \neq N$\index[sym]{$\subsetneq$}
  \end{description}
\end{definition}

\begin{theorem}
  Für Mengen $M$, $N$ und $O$ gilt (vgl. logische Operatoren):
  \begin{align*}
    M &\subseteq M\\
    (M \subseteq N \wedge N \subseteq O) &\implies M \subseteq O \tag{Transitivität\index{Menge!Transitivität}}\\
    (M \cap N) \cap O &= M \cap (N \cap O) & (M \cup N) \cup O &= M \cup (N \cup O)  \tag{Assoziativgesetz\index{Menge!Assoziativgesetz}} \\
    M \cap N &= N \cap M & M \cup N &= N \cup M \tag{Kommutativgesetz\index{Menge!Kommutativgesetz}} \\
    M \cap (N \cup O) &= (M \cap N) \cup (M \cap O) & M \cup (N \cap O) &= (M \cup N) \cap (M \cup O) \tag{Distributivgesetz\index{Menge!Distributivgesetz}} \\
    N \subseteq O &\implies M \cap N \subseteq M \cap O \\
    O \setminus (M \cap N) &= (O \setminus M) \cup (O \setminus N) & O \setminus (M \cup N) &= (O \setminus M) \cap (O \setminus N) \\
    O \setminus (O \setminus M) &= M \cap O
  \end{align*}
\end{theorem}

\begin{remark}
  \begin{itemize}
  \item Leere Menge: $\emptyset := \{\}$\index{Menge!Leere Menge}\index[sym]{$\emptyset$}
  \item Menge können andere Mengen als Elemente haben. $\left\{ \emptyset \right\}$ hat ein Element, $\left\{ \emptyset, \left\{ \emptyset \right\} \right\}$ hat zwei Elemente.
  \item Für $M:= \left\{ 1, \left\{ 2 \right\} \right\}$
    \begin{align*}
      1 &\in M & 2 &\not\in M \\
      \left\{ 1 \right\} &\subseteq M & \left\{ 2 \right\} &\not\subseteq M \\
      \left\{ 1 \right\} &\not\in M & \left\{ 2 \right\} &\in M \\
      \left\{ \left\{ 1 \right\} \right\} &\not\subseteq M & \left\{ \left\{ 2 \right\} \right\} &\subseteq M \\
    \end{align*}
  \end{itemize}
\end{remark}

\begin{definition}[Potenzmenge]
  $\mathscr{P}(M) := \left\{ N \middle| N \subseteq M \right\}$ ist die Potenzmenge\index{Menge!Potenzmenge}\index[sym]{P(M)@$\mathscr{P}(M)$} einer Menge $M$. $\mathscr{A} \subseteq \mathscr{P}(M)$ ist ein Mengensystem\index{Menge!Mengensystem}\index[sym]{A@$\mathscr{A}$} über $M$.
\end{definition}

\begin{remark}
Russell (1901): Es sei $R := \left\{ M \middle| M \text{ ist Menge} \wedge M \not\in M \right\}$. Ist $R$ in $R$?\index{Menge!Russels Paradox}
\begin{align*}
  R \in R &\implies R \not\in R \\
  R \not\in R &\implies R \in R
\end{align*}

Axiomatische Mengenlehre (grob zusammengefasst):
\begin{itemize}
\item Klasse: Zusammenfassungen von Objekten, die durch Eigenschaften beschrieben werden.
\item Mengen: Klassen, die selbst Element einer Klasse sind.
\end{itemize}
\end{remark}

\begin{definition}
  Ist $A(x)$ ein Ausdruck, für die man für $x$ Elemente von $M$ einsetzen kann, so heißt $A(x)$ eine Aussageform\index{Menge!Aussageform} über $M$. $x$ ist die freie Variable.
\end{definition}
\begin{remark}
  $\left\{ x \in M \middle| A(x) \right\} := \left\{ x \middle| x \in M \wedge A(x) \right\}$
\end{remark}
\begin{definition}
  Quantoren\index{Menge!Quantor}
  \begin{itemize}
  \item $\forall x \in M: A(x)$: Für alle $x \in M$ ist $A(x)$ wahr.\index[sym]{$\forall$}
  \item $\exists x \in M: A(x)$: Für mindestens ein $x \in M$ ist $A(x)$ wahr.\index[sym]{$\exists$}
  \item $\exists! x \in M: A(x)$: Für genau ein $x \in M$ ist $A(x)$ wahr.\index[sym]{$\exists"!"$}
  \end{itemize}
\end{definition}
\begin{remark}
  Negation von Quantoren:
  \begin{align*}
    \neg (\forall x: A(x)) &\iff \exists x: \neg A(x) \\
    \neg (\exists x: A(x)) &\iff \forall x: \neg A(x) \\
  \end{align*}
\end{remark}
\begin{definition}[Äquivalenz von Aussageformen\index{Aussage!Äquivalenz}]
  $(A(x) \iff B(x)) \iff (\forall x \in M: A(x) \iff B(x))$
\end{definition}
\begin{remark}
  Quantoren werden von links nach rechts gelesen. Reihenfolge beachten!
\end{remark}
\begin{definition}
  \begin{align*}
    \bigcap \mathscr{A} &:= \left\{ x \in M \middle| \forall A in \mathscr{A}: x \in A \right\} \\
    \bigcup \mathscr{A} &:= \left\{ x \in M \middle| \exists A in \mathscr{A}: x \in A \right\}
  \end{align*}
\end{definition}
\begin{definition}
  Sei $M$ eine endliche Menge, $|M|$ ist die Anzahl der Elemente von M.\index[sym]{$|M|$}
\end{definition}
\begin{remark}
  \begin{align*}
    |\mathscr{P}(M)| &= 2^{|M|} \\
    |M \cup N| &= |M| + |N| - |M \cap N| \\
  \end{align*}
\end{remark}

\section{Relationen und Funktionen}
\begin{definition}[Geordnetes Paar\index{Menge!Geordnetes Paar}]
  $(m, n) := \left\{ m, \left\{ m, n \right\} \right\}$
\end{definition}
\begin{definition}[Kartesisches Produkt\index{Menge!Kartesisches Produkt}]
  $M \times N := \left\{ (m, n) \middle| m \in M \wedge N \in N \right\}$\index[sym]{$\times$}
\end{definition}
\begin{definition}
  Eine Relation\index{Relation} $R = (M, N, G_R)$ besteht aus
  \begin{itemize}
  \item einer Definitionsmenge\index{Relation!Definitionsmenge} $M$
  \item einer Zielmenge\index{Relation!Zielmenge} $N$ und
  \item einer Menge $G_R \subset M \times N$ („Graph“)\index{Relation!Graph}
  \end{itemize}

  Relationen sind gleich, wenn die die gleiche Definitionsmengen, Zielmengen und Graphen haben
  
  Schreib- und Sprechweisen:
  \begin{itemize}
  \item „$m$ steht in $R$-Beziehung zu $n$“: $mRn \iff (m, n) \in G_R$
  \item „$R$ ist Relation von $M$ nach $N$“
  \item Ist $M=N$, so nennt man es eine Relation auf $M$.
  \end{itemize}
\end{definition}

\begin{definition}
  Eine Relation $R = (M, N, G_R)$ heißt eine Funktion\index{Funktion}, falls
  \[ \forall m \in M\, \exists! n \in N: (m, n) \in G_R \]
  Ist $m \in M$ und $(m, n) = G_R$, so schreibt man $n = R(m)$
  $n$ heißt das Bild\index{Funktion!Bild} von $m$ unter $R$.
\end{definition}
\begin{remark}
  Ist $f = (M, N, G_f)$ eine Funktion, so schreibt man $M \stackrel{f}{=} N$ oder $f: M \to N$ oder $m \mapsto f(m)$.\index[sym]{f(m)@$f(m)$}\index[sym]{$\to$}\index[sym]{$\mapsto$}

  Funktion ist bestimmt durch $M$, $N$ und die Zuordnung.
\end{remark}
\begin{definition}
  Es sei $f: M \to N$ eine Funktion.

  Ist $A \subseteq M$, dann ist $f[A] = \left\{ f(m) \middle| m \in A \right\}$ das Bild von $A$.\index{Funktion!Bild}\index[sym]{f[A]@$f[A]$}

  Ist $B \subseteq N$, dann ist $f^{-1}[B] = \left\{ m \in M \middle| f(m) \in B \right\}$ das Urbild von $B$.\index{Funktion!Urbild}\index[sym]{f-1[B]@$f^{-1}[B]$}

  Ist $n \in N$, dann ist $f^{-1}[n] = f[{n}]$ die Faser\index{Funktion!Faser} über $n$. Die Elemente von $f^{-1}[n]$ heißen Urbilder von $n$.\index[sym]{f-1[n]@$f^{-1}[n]$}
\end{definition}
\begin{definition}
  $f: M \to N, m \mapsto n$ heißt
  \begin{itemize}
  \item injektiv, wenn $\forall m_1, m_2 \in M: f(m_1) = f(m_2) \iff m_1 = m_2$ ($M \hookrightarrow N \implies \forall n \in N: |f^{-1}[n]| \le 1$)\index{Funktion!Injektiv}
  \item surjektiv, wenn $f[M] = N$ ($M \twoheadrightarrow N \implies \forall n \in N: |f^{-1}[n]| \ge 1$)\index{Funktion!Surjektiv}
  \item bijektiv, wenn $f$ injektiv und surjektiv ist. ($M \hooktwoheadrightarrow N \implies \forall n \in N: |f^{-1}[n]| = 1$)\index{Funktion!Bijektiv}
  \end{itemize}
\end{definition}
\begin{definition}
  Sei $A \subseteq M$, dann ist $f|_A: A \to N, a \mapsto f(a)$ Restriktion.\index{Funktion!Restriktion}\index[sym]{f"|"A@$f"|"_A$}

  Sei $f[M] \subseteq B \subseteq N$, dann ist $f|^B: M \to B, m \mapsto f(m)$ Korrektion.\index{Funktion!Korrektion}\index[sym]{f"|"B@$f"|"^B$}
\end{definition}
\begin{remark}
  Ist $M \to N$ bijektiv und $n \in N$, dann $\exists! m \in M: f(m) = n$. Man nennt $f^{-1}: N \to M, n \mapsto m$ Umkehrfunktion von $f$.\index{Funktion!Umkehrfunktion}\index[sym]{f-1@$f^{-1}$}
\end{remark}

\begin{definition}
  Es sei $f: M \to N, g: N \to O$, dann ist $g \circ f: M \to O, m \mapsto g(f(m))$ eine Komposition\index{Funktion!Komposition}\index[sym]{$\circ$}. Man sagt „$g$ nach $f$“
\end{definition}
\begin{lemma}[Assoziativität der Komposition]
  Es seien $f: M \to N, g: N \to O, h: O \to P$. $h \circ (g \circ f) = (h \circ g) \circ f$
\end{lemma}
\begin{proof}
  Beides sind Funktionen von $M$ nach $P$ und
  \begin{align*}
    h \circ (g \circ f) (m) &= h (g \circ f (m)) \\
    \, &= h(g(f(m))) \\
    \, &= (h \circ g)(f(m)) \\
    \, &= (h \circ g) \circ f (m)
  \end{align*}
\end{proof}

\begin{remark}
  $\operatorname{Id}_M: M \to M, m \mapsto m$ heißt Identität\index{Funktion!Identität}\index[sym]{Id@$\operatorname{Id}_M$}. Dann gilt für $f: M \to N$:
  \begin{align*}
    f \circ \operatorname{Id}_M &= f \\
    \operatorname{Id}_M \circ f &= f
  \end{align*}
\end{remark}

\begin{theorem}
  Es sei $f: M \to N$ eine Funktion und $M \ne \emptyset$.
  \begin{itemize}
  \item Ist $f$ injektiv, dann $\exists g: N \to M: g \circ f = \operatorname{Id}_M$
  \item Ist $f$ surjektiv, dann $\exists h: N \to M: f \circ h = \operatorname{Id}_N$
  \item Ist $f$ bijektiv, dann $\exists g, h$ wie oben und $g = h = f^{-1}$ und $f \circ f = \operatorname{Id}_M$
  \end{itemize}
\end{theorem}
\begin{proof}
  Man nimmt an, $f$ ist injektiv.

  Man wählt $m_0 \in M$ (beachte: $M \ne \emptyset$). Ist $n \not\in \operatorname{Bild}(f)$, dann wählt man $g(n) = m$. Ist $n \in \operatorname{Bild}(f)$, wählt man $g(n) = m$.

  Dann ist $(g \circ f)(m) = g(f(m)) = m = \operatorname{Id}_M(m)$

  Sei $g$ wie oben. Für $m_1, m_2 \in M$ gilt
  \begin{align*}
    f(m_1) &= f(m_2) \\
    \implies m_1 = \operatorname{Id}_M(m_1) &= (g \circ f)(m_1) \\
    \, &= g(\underbrace{f(m_2)}_{= f(m_{1})}) = (g \circ f)(m_2) \\
    \, &= \operatorname{Id}_M(m_2) = m_2 \\
    \implies m_1 &= m_2
  \end{align*}
\end{proof}
% -*- TeX-master: "../main" -*-

\chapter{Lineares Gleichungssystem}

\begin{example}
  \begin{align*}
    ax_1 + bx_2 &= c \\
    dx_1 + ex_2 &= f \\
    &\iff \operatorname{LGS}(x_1, x_2)
  \end{align*}
  D.h. Es ist eine Aussageform über $\mathbb{R} \times \mathbb{R}$

  Lösungen: $\mathscr{L} = \left\{ (x_1, x_2) \middle| \operatorname{LGS}(x_1, x_2) \right\}$
\end{example}

\begin{observation}[Geometrie]
  Es seien $x_1, x_2 \in \mathbb{R}$. Man definiert $x =
  \begin{pmatrix}
    x_1\\x_2
  \end{pmatrix}
  : \mathbb{E}^2 \to \mathbb{E}^2$
  . Diese abbildung heißt Vektor.
\end{observation}

\begin{definition}[Vektoraddition]
  $
  \begin{pmatrix}
    x_1\\x_2
  \end{pmatrix}
  \circ
  \begin{pmatrix}
    y_1\\y_2
  \end{pmatrix}
  :=
  \begin{pmatrix}
    x_1+y_1\\x_2+y_2
  \end{pmatrix}
  $
\end{definition}
\begin{definition}[Skalarmultiplikation]
  Sei $\lambda \in \mathbb{R}$.
  $\lambda
  \begin{pmatrix}
    x_1\\x_2
  \end{pmatrix}
  :=
  \begin{pmatrix}
    \lambda x_1 \\ \lambda x_2
  \end{pmatrix}
  $
\end{definition}
\begin{remark}
  Ist $O \in \mathbb{E}^2$ ein „Ursprung“, dann ist $I_0: \mathbb{R}^2 \to \mathbb{E}^2, v \mapsto v(0)$.

  Nach Wahl von $O$: $\mathbb{R}^2 \leftrightarrow \mathbb{E}^2$
\end{remark}
\begin{remark}
  $\mathbf{0} =
  \begin{pmatrix}
    0\\0
  \end{pmatrix}
  $ heißt Nullvektor.
\end{remark}

\begin{proposition}[Lösung eines LGS]
  $\Phi: \mathbb{R}^2 \to \mathbb{R}^2,
  \begin{pmatrix}
    x_1\\x_2
  \end{pmatrix}
  \mapsto
  \begin{pmatrix}
    ax_1 + bx_2\\ cx_1+dx_2
  \end{pmatrix}
  \implies \mathscr{L} = \Phi \left[
    \begin{pmatrix}
      e\\f
    \end{pmatrix}
  \right]
  $

  $\Phi$ hat folgende Eigenschaften:
  \begin{itemize}
  \item $\forall \lambda \in \mathbb{R}, x \in \mathbb{R}^2: \Phi(\lambda x) = \lambda \Phi(x)$
  \item $\forall x, y \in \mathbb{R}^2: \Phi(x + y) = \Phi(x) + \Phi(y)$
  \end{itemize}

  D.h. $\Phi$ ist linear.
\end{proposition}
\begin{remark}
  $\Phi \text{ ist linear} \implies \Phi(\lambda x + \mu y) = \lambda \Phi(x) + \mu \Phi (y), \lambda, \mu \in \mathbb{R}, x, y \in \mathbb{R}^2$
\end{remark}
\begin{notation}
  $Q:=
  \begin{pmatrix}
    a&b\\c&d
  \end{pmatrix}
  \in \mathbb{R}^{2 \times 2}$ ist eine Matrix.

  Zu jeder Matrix $Q$ gibt es eine Abbildung $\Phi_Q: \mathbb{R}^2 \to \mathbb{R}^2,
  x \mapsto
  \begin{pmatrix}
    ax_1 + bx_2 \\ cx_1+ dx_2
  \end{pmatrix}
  = Qx
  $
\end{notation}
\begin{theorem}
  Die Abbildung $\iota: \mathbb{R}^{2 \times 2} \to \operatorname{Lin}(\mathbb{R}^2, \mathbb{R}^2), Q \mapsto \Phi_Q$ ist bijektiv.
\end{theorem}
\begin{definition}[Standardbasis]
  Es seien $e_1 :=
  \begin{pmatrix}
    1\\0
  \end{pmatrix}
  , e_2 :=
  \begin{pmatrix}
    0\\1
  \end{pmatrix}
  $. Das Paar $(e_1, e_2)$ heißt die Standardbasis von $\mathbb{R}^2$
\end{definition}
\begin{remark}
  Es gilt: $Q_{\Phi}(e_1) =
  \begin{pmatrix}
    a\\c
  \end{pmatrix}
  , Q_{\Phi}(e_2) =
  \begin{pmatrix}
    b\\d
  \end{pmatrix}
  $.
  $Q$ heißt also die darstellende Matrix von $\Phi$
\end{remark}

\begin{definition}
  $\operatorname{Karn}(\Phi) = \Phi^{-1}[\mathbf{0}]$

  Ein LGS heißt homogen, falls $e = f = 0$. Seine Lösung ist dann $\operatorname{Kern}(\Phi)$
\end{definition}
\begin{proposition}
  Sei $W = \operatorname{Kern}(\Phi)$
  \begin{itemize}
  \item $\mathbf{0} \in W$, insbesondere ist $W \ne \emptyset$
  \item $\forall x, y \in W: x+y \in W$
  \item $\forall x \in W, \lambda \in \mathbb{R}: \lambda x \in W$
  \end{itemize}
  $W$ heißt Untervektorraum.
\end{proposition}
\begin{definition}[Ursprungsgerade]
  Ist $x \in \mathbb{R}^2$, dann heißt $\mathbb{R}x := \left\{ \lambda x \middle| \lambda \in \mathbb{R} \right\}$ die Ursprungsgerade durch $x$.
\end{definition}
\begin{theorem}
  Es gibt keine weitere Untervektorräume als
  \begin{itemize}
  \item $\left\{ \mathbf{0} \right\}$
  \item Unsprungsgeraden: $\left\{ \lambda x \middle| \lambda \in \mathbb{R}, x \in \mathbb{R}^2 \right\}$
  \item $\mathbb{R}^2$
  \end{itemize}
\end{theorem}

% NOTES FORM 08.11. NEED TO BE COMPLEMENTED

\begin{remark}
  Besondere Matrizen:
  Einheitsmatrix: $\mathbf{1} =
    \begin{pmatrix}
      1 0 \\ 0 1
    \end{pmatrix}$
\end{remark}
\begin{definition}
  $A = D(\Phi)$ heißt invertierbar, falls $\Phi$ bijektiv ist, $A^{-1} = D(\Phi^{-1})$ heißt inversion.
\end{definition}

% NOTES FROM 10.11. NEED TO BE COMPLEMENTED

\part{Analysis I}
% -*- TeX-master: "../main" -*-

\chapter{Grundlagen}
\setcounter{section}{-1}

\section{Beispiele von logischen Schlussfolgerungen}

Natürliche Zahlen: \( \mathbb{N} = \left\{ 1, 2, \dots \right\} \)

$n$ ist gerade, wenn $k \in \mathbb{N}$ existiert mit $n = 2k$.

$n$ ist ungerade, wenn $k \in \mathbb{N}$ existiert mit $n = 2k-1$.

\begin{theorem}
  Sei $n \in \mathbb{N}$. $n$ ist genau dann gerade, wenn $n^2$ gerade ist.
\end{theorem}

\begin{proof}
  a. $n$ gerade $\Rightarrow$ $n^2$ gerade
  \begin{align*}
    n &= 2k \\
    n^2 &= (2k)^2 \\
    \, &= 2\underbrace{(2k^2)}_{\in \mathbb{N}} \\
    \, &\in \mathbb{N}
  \end{align*}

  b. $n^2$ gerade $\Rightarrow$ $n$ gerade: schwer zu beweisen

  c. $n$ ungerade $\Rightarrow$ $n^2$ ungerade
  \begin{align*}
    n &= 2k-1 \\
    n^2 &= (2k-1)^2 \\
    \, &= 4k^2-4k+1 \\
    \, &= 2\underbrace{(2k^2-2k+1}_{\in \mathbb{N}})-1 \\
    \, &\in \mathbb{N}
  \end{align*}

  Was hat c mit b zu tun?

  $p$: „$n$ ist gerade“; $q$: „$n^2$ ist gerade“.

  b. $p \Rightarrow q$; c. $\neg p \Rightarrow \neg q$ $\leftarrow$ Kontraposition zu b

  D.h. b ist genau dann wahr, wenn c wahr ist.

  c ist wahr $\Rightarrow$ b ist auch wahr.
\end{proof}

\begin{example}[Beweis durch Widerspruch]
  $\sqrt{2}$ ist irrational.

  \begin{proof}
    Annahme: $\sqrt{2}$ ist rational.

    Es sei \( A = \left\{ n \in \mathbb{N} \middle| \exists m \in \mathbb{Z}: \sqrt{2} = \frac{m}{n} \right\} \)

    $\sqrt{2}$ ist rational, wenn $A$ nicht leer ist. $A \subseteq \mathbb{N}$.

    Es sei $n_{*}: \forall n \in A: n \ge n_{*}$. Dann ist $\sqrt{2} = \frac{m}{n_{*}}$.

    \begin{align*}
      m-n_{*} &= \sqrt{2}n_{*} = \underbrace{\left( \sqrt{2}-1 \right)}_{< 1}n_{*} \\
      \, &< n_{*} \\
      m-n_{*} &\in \mathbb{N} \\
      \sqrt{2} &= \frac{m}{n_{*}} = \frac{m(m-n_{*})}{n(m-n_{*})} = \frac{m^2-mn_{*}}{n(m-n_{*})} \\
      \, &= \frac{2n_{*}^2-mn_{*}}{n(m-n_{*})} = \frac{2n_{*}-m}{m-n_{*}} \\
      \, &= \frac{\tilde{m}}{m-n_{*}}, \tilde{m} \in \mathbb{Z} \\
    \end{align*}
    Widerspruch: $m-n_{*} < n$
  \end{proof}

  Man kann damit auch zeigen, dass $\sqrt{3}$ irrational ist.

  Was ist mit $\sqrt{k}, k \in \mathbb{N}$?
\end{example}

\begin{theorem}
  Sei $k \in \mathbb{N}$. Dann ist entweder $\sqrt{k} \in \mathbb{N}$ oder $\sqrt{k}$ irrational.
\end{theorem}
\begin{proof}
  Annahme: $k \in \mathbb{Q} \setminus \mathbb{N}$

  Es sei \( A = \left\{ n \in \mathbb{N} \middle| \exists m \in \mathbb{Z}: \sqrt{k} = \frac{m}{n} \right\} \) (vgl. oben)

  $A$ hat ein kleinstes Element $n_{*}$ (vgl. oben)

  Womit soll man dies erweitern? ($m - l \cdot n_{*}$)
\end{proof}
% -*- TeX-master: "../main" -*-
\chapter{Die Axiome der reellen Zahlen}
Axiome der reelle Zahlen: Es gibt eine Menge $\mathbb{R}$, genannt „reelle Zahlen“, die 3 Gruppen von Axiomen erfüllt:
\begin{itemize}
\item algebraische Axiome
\item Anordnungsaxiome
\item Vollständigkeitsaxiom (!)
\end{itemize}

\section{Algebraische Axiome}
In $\mathbb{R}$ gibt es 2 Operationen:
\begin{itemize}
\item Addition: $a+b \in \mathbb{R}$
\item Multiplikation: $a \cdot b \in \mathbb{R}$
\end{itemize}

Es gilt folgende Axiome:
\begin{enumerate}
\item Assoziativität der Addition: $(a + b) + c = a + (b + c)$
\item Kommutativität der Addition: $a + b = b + a$
\item Neutrales Element der Addition: Es gibt genau eine Zahl, genannt „null“ und geschrieben 0, mit $\forall a \in \mathbb{R}: a + 0 = a$
\item Inverse der Addition: $\forall a \in \mathbb{R}\: \exists! b \in R: a + b = 0$, geschrieben $b = -a$
\item Assoziativität der Multiplikation: $(a \cdot b) \cdot c = a \cdot (b \cdot c)$
\item Kommutativität der Multiplikation: $a \cdot b = b \cdot a$
\item Neutrales Element der Multiplikation: Es gibt genau eine Zahl, genannt „eins“ und geschrieben 1, die von null verschieden ist, mit $\forall a \in \mathbb{R}, a \cdot 1 = a$
\item Inverse der Multiplikation: $\forall a \in \mathbb{R}\: \exists! b \in R: a \cdot b = 1$, geschrieben $b = a^{-1}$
\item Distributivgesetz: $a \cdot (b + c) = a \cdot b + a \cdot c$
\end{enumerate}

Jede Menge $\mathbb{K}$, die die o.g. Axiome erfüllt, heißt Körper.

\begin{definition}
  Notation:
  \begin{itemize}
  \item Negation: $a - b := a + (-b)$
  \item Quotient: $\frac{a}{b} := a \cdot b^{-1}$
  \end{itemize}
\end{definition}

\begin{theorem}
  Es gilt:
  \begin{itemize}
  \item $-(-a) = a$
  \item $(-a)+(-b) = -(a+b)$
  \item $(a^{-1})^{-1} = a$
  \item $a^{-1} b^{-1} = (ab)^{-1}$
  \item $a \cdot 0 = 0$
  \item $a(-b) = -(ab)$
  \item $a(b-c) = ab-ac$
  \item $ab = 0 \implies a = 0 \vee b = 0$
  \end{itemize}
\end{theorem}
\begin{proof}[zu $a \cdot 0 = 0$]
  \begin{align*}
    a \cdot 0 + a \cdot 0 &= a \cdot (0+0) \\
    \,&= a \cdot 0 \\
    \implies (a \cdot 0 + a \cdot 0) + (-a \cdot 0) &= a \cdot 0 + (-a \cdot 0) \\
    \,&= 0 \\
    (a \cdot 0 + a \cdot 0) + (-a \cdot 0) &= a \cdot 0 + 0 \\
    \,&= a \cdot 0 \\
    \implies a \cdot 0 &= 0
  \end{align*}
\end{proof}
\begin{proof}[zu $ab = 0 \implies a = 0 \vee b = 0$]
  \begin{align*}
    a \ne 0 \implies b &= 1 \cdot b \\
    \,&= (a^{-1}a)b \\
    \,&= a^{-1}(ab) \\
    \,&= a \cdot 0 \\
    \,&= 0
  \end{align*}
\end{proof}

\begin{theorem}
  \begin{align*}
    \frac{a}{b} + \frac{c}{d} &= \frac{ad+cb}{bd} \\
    \frac{a}{b} \cdot \frac{c}{d} &= \frac{ac}{bd} \\
    \frac{\frac{a}{b}}{\frac{c}{d}} &= \frac{ad}{bc}
  \end{align*}
\end{theorem}

\begin{proof}[Zur Eindeutigkeit von 0 und 1]
  Es seien 0 und $0'$ neutrale Elemente der Addition. Dann ist $0 = 0 + 0' = 0'$
\end{proof}
\begin{proof}[Zur Eindeutigkeit der Inverse]
  Es seien $a + b = 0, a + b' = 0$. Dann ist
  \begin{align*}
    b' &= b' + 0 \\
    \,&= b' + (a + b) \\
    \,&= (a + b') + b \\
    \,&= b
  \end{align*}
\end{proof}

\section{Die Anordnungsaxiome}
\[ \forall a, b \in \mathbb{R}: a = b \vee a \ne b \]

Ist $a \ne b$, so besteht eine Anordnung $<$, sodass genau eine der Relationen $a < b$ oder $b < a$ gilt.

D.h., für alle $a, b \in \mathbb{R}$ gilt genau eine der Aussagen $a < b, a = b, a > b$.

Diese Anordnung genügt den Axiomen
\begin{itemize}
\item Transitivität: $a < b \wedge b < c \implies a < c$
\item $a < b, c = \mathbb{R} \implies a + c < b + c$
\item $a < b, c > 0 \implies ac < bc$
\end{itemize}

\begin{definition}
  Notation:
  \begin{itemize}
  \item $a < b$: $a$ ist echt kleiner als $b$
  \item $a \le b \iff a < b \vee a = b$
  \item $a > b$: $a$ ist größer als $b$
  \item $a \ge b \iff a > b \vee a = b$
  \item $a$ ist positiv, wenn $a > 0$; $a$ ist negativ, wenn $a < 0$
  \item $a$ ist nicht negativ, wenn $a \ge 0$; $a$ ist nicht positiv, wenn $a \le 0$
  \end{itemize}
\end{definition}

\begin{example}[Beweis für $a < b \iff b - a > 0$]
  \begin{align*}
    a < b &\implies a-a < b-a \\
    \,&\implies 0 < b-a \\
    b-a > 0 &\implies b-a+a > a \\
    \,&\implies b > a
  \end{align*}
\end{example}

\begin{theorem}
  \begin{align*}
    a < b &\iff b-a > 0 \\
    a < 0 &\iff -a > 0 \\
    a > 0 &\iff -a < 0 \\
    a < b &\iff -b < -a \\
    a < b \wedge c < d &= a+c < b+d \\
    ab > 0 &\iff (a > 0 \wedge b > 0) \vee (a < 0 \wedge b < 0) \\
    ab < 0 &\iff (a > 0 \wedge b < 0) \vee (a < 0 \wedge b > 0) \\
    a \ne 0 &\iff a^2 > 0 \\
    a < b \wedge c < 0 &\iff ac > bc \\
    a > 0 &\iff \frac{1}{a} > 0 \\
    a^2 < b^2 \wedge a > 0 \wedge b > 0 &\implies a < b
  \end{align*}
\end{theorem}
\begin{proof}[Beweis für $a < 0 \iff b-a > 0$, etc.]
  Sei $a < b$, dann ist
  \begin{align*}
    a + (-a) &< b + (-a) \\
    0 &< b - a
  \end{align*}

  Man setzt oben $b = 0$ ein. Dann ist
  \begin{align*}
    0 &< 0 - a \\
    0 &< -a
  \end{align*}
\end{proof}
\begin{proof}[Beweis für $a < b \wedge c < d \iff a+c < b+d$]
  \begin{align*}
    a < b \wedge c + d &\iff a + c < b + c \wedge b + c < b + d \\
    \,&\iff a + c < b + d
  \end{align*}
\end{proof}
\begin{proof}[Beweis für $a, b > 0 \vee a, b < 0 \implies ab > 0$]
  \begin{align*}
    a, b > 0 &\iff ab > 0b \\
    a, b < 0 &\iff (-a), (-b) > 0 \\
    \,&\implies (-a)(-b) > 0 \\
    \,&\iff ab > 0
  \end{align*}
\end{proof}

\section{Das Vollständigkeitsaxiom}
\begin{axiom}[Vollständigkeitsaxiom]
  Jede nichtleere Teilmenge $M \subseteq \mathbb{R}$, welche nach oben beschränkt ist, besitzt eine kleinste obere Schrankt, ganannt das Supremum vom $M$, geschrieben $\operatorname{sup} M$.
\end{axiom}

\begin{definition}
  Sei $M \subseteq \mathbb{R}$, $M \ne \emptyset$
  \begin{itemize}
  \item $M$ heißt nach oben beschränkt, falls $k \in \mathbb{R}$ existiert mit $\forall x \in M: x \le k$. Jede solche Zahl heißt obere Schranke von $M$.
  \item $M$ heißt nach unten beschränkt, falls $k \in \mathbb{R}$ existiert mit $\forall x \in M: x \ge k$. Jede solche Zahl heißt untere Schranke von $M$.
  \item $M$ heißt beschränkt, falls $k \ge 0$ existiert mit $\forall x \in M: -k \le x \le k$.
  \end{itemize}
\end{definition}
\begin{definition}
  Eine Zahl $k \in \mathbb{R}$ heißt kleinste obere (bzw. größte untere) Schranke, falls
  \begin{itemize}
  \item Es eine obere (bzw. untere) Schranke ist und
  \item Es gibt keine kleinere obere (bzw. größere untere) Schranke für $M$
  \end{itemize}
\end{definition}
\begin{remark}
  $x \le k \iff -k \ge -x$. D.h.,
  \begin{itemize}
  \item $k$ ist eine obere Schranke für $M$ $\iff$ $-k$ ist eine untere Schranke für $-M = \left\{ x \middle| x \in M \right\}$.
  \item $k$ ist die kleinste obere Schranke für $M$ $\iff$ $-k$ ist die größte untere Schranke für $-M$.
  \end{itemize}

  Jede nichtleere nach unten beschränkte Menge $M \subseteq \mathbb{R}$ hat eine größte untere Schranke, genannt Infimum von $M$, geschrieben $\operatorname{inf} M$
\end{remark}
\begin{example}
  Es seien $M = [0, 1] = \left\{ x \middle| 0 \le x \le 1 \right\}, A = (0, 1) = \left\{ x \middle| 0 < x < 1 \right\}$.
  \begin{align*}
    \operatorname{sup} M &= 1 & \operatorname{inf} M &= 0 \\
    \operatorname{sup} A &= 1 & \operatorname{inf} A &= 0
  \end{align*}
\end{example}
\begin{notation}
  Sei $M \subseteq R, M \ne \emptyset$. Man schreibt
  \begin{itemize}
  \item $\operatorname{sup} M < \infty$ bzw. $\operatorname{inf} M > -\infty$, falls $M$ nach oben (bzw. unten) beschränkt ist
  \item $\operatorname{sup} M = \infty$ bzw. $\operatorname{inf} M = \infty$, falls $M$ nicht nach oben (bzw. unten) beschränkt ist
  \end{itemize}
\end{notation}
\begin{theorem}
  Sei $M \subseteq \mathbb{R}, M \ne \emptyset$.
  \begin{itemize}
  \item Ist $\operatorname{sup} M < \infty$, folgt $\forall \epsilon > 0\: \exists x \in M: (\operatorname{sup} M)-\varepsilon < x$.
  \item Ist $\operatorname{sup} M = \infty$, so gilt $\forall k \ge 0\: \exists x \in M: x \ge k$
  \end{itemize}
\end{theorem}
\begin{theorem}
  Sei $M \subseteq \mathbb{R}, M \ne \emptyset$.
  \begin{itemize}
  \item Ist $\operatorname{inf} M > -\infty$, folgt $\forall \varepsilon > 0\: \exists x \in M: x < (\operatorname{inf} M)+\varepsilon$
  \item Ist $\operatorname{inf} M = -\infty$, folgt $\forall k \ge 0\: \exists x \in M: x < -k$
  \end{itemize}
\end{theorem}
\begin{theorem}
  Sei $M \subseteq \mathbb{R}, M \ne \emptyset$.
  \begin{itemize}
  \item $m \in M$ heißt größtes Element (Maximum) von $M$, geschrieben $\operatorname{max} M$, wenn $\forall x \in M: x \le m$
  \item $m \in M$ heißt kleinstes Element (Minimum) von $M$, geschrieben $\operatorname{min} M$, wenn $\forall x \in M: x \ge m$
  \end{itemize}
\end{theorem}
\begin{example}
  Sei $M = \left[0, 1\right)$. Dann ist $\operatorname{sup} M = 1, \operatorname{inf} M = 0$
  \begin{itemize}
  \item $M$ hat kein Maximum.
  \item $M$ hat ein Minimum.
  \end{itemize}
\end{example}
% -*- TeX-master: "../main" -*-
\chapter{Die natürliche Zahlen $\mathbb{N}$}
\begin{definition}\label{Ana1:N:Induktiv}
  Eine Teilmenge $A \in \mathbb{R}$ heißt induktiv, falls
  \begin{itemize}
  \item $1 \in A$
  \item $x \in A \implies x+1 \in A$
  \end{itemize}
\end{definition}
\begin{observation}
  Sei $J$ eine Indexmenge und $A_j, j \in J$ induktive Teilmengen von $\mathbb{R}$. Dann ist $A := \bigcap_{j \in J} A_j$ induktiv.
  \begin{proof}
    \begin{align*}
      x \in A &\iff \forall j \in J: x \in A_j \\
      \,&\implies \forall j \in J: x+1 \in A_j \\
      \,&\implies x+1 \in A
    \end{align*}
  \end{proof}
\end{observation}
\begin{definition}\label{Ana1:N}
  Sei $\mathfrak{F} := \left\{ A \subseteq R \middle| A \text{ ist induktiv} \right\}$. Man definiert $\mathbb{N}$ als die kleinster induktive Teilmenge von $\mathbb{R}$, d.h. $\mathbb{N} := \bigcap_{A \in \mathfrak{F}} A$
\end{definition}

\begin{theorem}[Induktionsprinzip]\label{Ana1:N:Induktionsprinzip}
  Ist $M \subseteq \mathbb{N}$ induktiv, dann ist $M = \mathbb{N}$.
\end{theorem}
\begin{proof}
  Nach Vorraussetzung ist $M \subseteq \mathbb{N}$. Aus \ref{Ana1:N} ist $\mathbb{N} \subseteq M \implies M = \mathbb{N}$.
\end{proof}
\begin{theorem}[Induktionsbeweis]
  Für jedes $n \in \mathbb{N}$ sei eine Aussage $B_n$ gegeben - derart, dass folgendes gilt:
  \begin{description}
  \item[Induktionsanfang] $B_1 \iff w$
  \item[Induktionsschluss] $\forall n \in \mathbb{N}: (B_n \implies B_{n+1})$
  \end{description}
  Dann ist $\forall n \in \mathbb{N}: B_n$
\end{theorem}
\begin{remark}
  Im Induktionsschluss heißt $B_n$ Induktionsannahme.
\end{remark}
\begin{proof}
  Man definiert $M := \left\{ n \in \mathbb{N} | B_n \iff w \right\}$. Zu zeigen ist: $M = \mathbb{N}$.
  \begin{itemize}
  \item $(B_1 \iff w) \implies 1 \in M$
  \item $B_n \implies B_{n+1} \implies n+1 \in M$
  \end{itemize}
  Nach \ref{Ana1:N:Induktiv} und \ref{Ana1:N:Induktionsprinzip} ist $M = \mathbb{N}$.
\end{proof}
\begin{remark}[Variante eines Induktionsbeweis]
  $B_1$ ist wahr und $\left( \bigwedge_{k=1}^n B_k \implies B_{n+1} \right)$.
\end{remark}

\begin{theorem}[Eigenschaften von $\mathbb{N}$]
  Es gilt:
  \begin{itemize}
  \item $\forall n \in \mathbb{N}: n \ge 1$
  \item $\forall n, m \in \mathbb{N}: n+m \in \mathbb{N}$
  \item $\forall n, m \in \mathbb{N}: n \cdot m \in \mathbb{N}$
  \item $\forall n, m \in \mathbb{N}: \text{entweder } n = 1 \text{ oder } n-1 \in \mathbb{N}$
  \item $\forall n, m \in \mathbb{N}: m < n \implies n - m \in \mathbb{N}$
  \end{itemize}
\end{theorem}

\begin{corollary}
  $\not\exists n \in \mathbb{N}: 0 < n < 1$. Ferner gilt: $\forall n \in \mathbb{N}\: \not\exists m \in \mathbb{N}: n < m < n+1$
\end{corollary}

\begin{definition}[Rationale Zahlen]
  $\mathbb{Q} := \left\{ \frac{p}{q} \middle| p \in \mathbb{Z}, q \in \mathbb{N} \right\}$
\end{definition}
\begin{definition}[Irrationale Zahlen]
  $\mathbb{R} \setminus \mathbb{Q}$
\end{definition}
\begin{remark}
  Sei für $B_n$ für $n \in \mathbb{Z}$ eine Aussage, $n_0 \in \mathbb{N}$. Dann ist $(\forall n \ge n_0: B_n) \iff (B_n \wedge (B_n \wedge n \ge n_0 \implies B_{n+1}))$
\end{remark}
\begin{theorem}
  $a, b \in \mathbb{Z} \implies a + b \in Z \wedge a \cdot b \in \mathbb{Z}$
\end{theorem}

\begin{theorem}[Satz von Archimedes]
  $\forall x \in \mathbb{R}\: \exists n \in \mathbb{N}: x < n$ (D.h. $\mathbb{N}$ ist eine nach oben unbeschränkte Teilmenge von $\mathbb{R}$)
\end{theorem}
\begin{proof}
  Angenommen: die Aussage ist falsch, d.h., $\exists x \in \mathbb{R}\: \forall n \in \mathbb{N}: n \le x$.
  Nach dem Vollständigkeitsaxiom: $\exists a = \operatorname{sup} \mathbb{N}$

  $a$ ist die kleinste obere Schranke: $\exists n \in \mathbb{N}: a-1 < n$

  Daraus folgt aber: $a < n + 1$. Widerspruch!
\end{proof}

\begin{corollary}
  $\forall x \in \mathbb{R}\: \exists n \in \mathbb{N}: -n < x$
\end{corollary}

\begin{theorem}[wohlordnungsprinzip für $\mathbb{N}$]
  Jede nichtleere Menge natürlicher Zahlen hat ein kleinstes Element.
\end{theorem}
\begin{proof}
  Sei $M \subseteq \mathbb{N}, M \ne \emptyset$.

  $\operatorname{inf} \mathbb{N} = 1 \implies a = \operatorname{inf} M \ge 1 > \infty$. Zu zeigen: $a \in M$

  Angenommen: $a \not\in M$
  \begin{align*}
    \,&\implies \forall m \in M: a < m \\
    \varepsilon := 1 &\implies \exists m \in M: m < a+1 \\
    \varepsilon := m - a &\implies \exists m' \in M: m' < a - \varepsilon = m \\
    \,&\implies \exists m', m \in M: 0 < m - m' < 1
  \end{align*}
  Widerspruch, denn $m - m' \not\in \mathbb{N}$
\end{proof}

\backmatter
\part{Appendices}
\printindex
\printindex[sym]
\printindex[java]

\end{document}