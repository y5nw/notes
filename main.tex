\documentclass[openany]{book}
\usepackage[T1]{fontenc}
\usepackage[a4paper,margin=1in]{geometry}
\usepackage{stix2}
\usepackage[ngerman]{babel}
\usepackage{units}
\usepackage{amsmath,amssymb,amsthm}
\usepackage{mathtools}
\usepackage{hyperref}
\usepackage{cancel,ulem}
\usepackage{menukeys}
\usepackage{tabularx}
\usepackage{tikz}
\usetikzlibrary{datavisualization,datavisualization.formats.functions}

\title{Notizen}
\author{}
\date{}

\everymath{\displaystyle}

\def\phantomequal{\phantom{{}={}}}
\def\bar#1{\overline{#1}}

\begin{document}
\newtheorem{theorem}{Satz}[section]
\newtheorem{axiom}{Axiom}[section]
\newtheorem{lemma}{Lemma}[section]
\theoremstyle{definition}
\newtheorem{example}{Beispiel}[section]
\newtheorem{definition}{Definition}[section]
\theoremstyle{remark}
\newtheorem*{remark}{Bemerkung}

\maketitle
\tableofcontents

\part{Analysis I}
% -*- TeX-master: "../main" -*-

\chapter{Grundlagen}
\setcounter{section}{-1}

\section{Beispiele von logischen Schlussfolgerungen}

Natürliche Zahlen: \( \mathbb{N} = \left\{ 1, 2, \dots \right\} \)

$n$ ist gerade, wenn $k \in \mathbb{N}$ existiert mit $n = 2k$.

$n$ ist ungerade, wenn $k \in \mathbb{N}$ existiert mit $n = 2k-1$.

\begin{theorem}
  Sei $n \in \mathbb{N}$. $n$ ist genau dann gerade, wenn $n^2$ gerade ist.
\end{theorem}

\begin{proof}
  a. $n$ gerade $\Rightarrow$ $n^2$ gerade
  \begin{align*}
    n &= 2k \\
    n^2 &= (2k)^2 \\
    \, &= 2\underbrace{(2k^2)}_{\in \mathbb{N}} \\
    \, &\in \mathbb{N}
  \end{align*}

  b. $n^2$ gerade $\Rightarrow$ $n$ gerade: schwer zu beweisen

  c. $n$ ungerade $\Rightarrow$ $n^2$ ungerade
  \begin{align*}
    n &= 2k-1 \\
    n^2 &= (2k-1)^2 \\
    \, &= 4k^2-4k+1 \\
    \, &= 2\underbrace{(2k^2-2k+1}_{\in \mathbb{N}})-1 \\
    \, &\in \mathbb{N}
  \end{align*}

  Was hat c mit b zu tun?

  $p$: „$n$ ist gerade“; $q$: „$n^2$ ist gerade“.

  b. $p \Rightarrow q$; c. $\neg p \Rightarrow \neg q$ $\leftarrow$ Kontraposition zu b

  D.h. b ist genau dann wahr, wenn c wahr ist.

  c ist wahr $\Rightarrow$ b ist auch wahr.
\end{proof}

\begin{example}[Beweis durch Widerspruch]
  $\sqrt{2}$ ist irrational.

  \begin{proof}
    Annahme: $\sqrt{2}$ ist rational.

    Es sei \( A = \left\{ n \in \mathbb{N} \middle| \exists m \in \mathbb{Z}: \sqrt{2} = \frac{m}{n} \right\} \)

    $\sqrt{2}$ ist rational, wenn $A$ nicht leer ist. $A \subseteq \mathbb{N}$.

    Es sei $n_{*}: \forall n \in A: n \ge n_{*}$. Dann ist $\sqrt{2} = \frac{m}{n_{*}}$.

    \begin{align*}
      m-n_{*} &= \sqrt{2}n_{*} = \underbrace{\left( \sqrt{2}-1 \right)}_{< 1}n_{*} \\
      \, &< n_{*} \\
      m-n_{*} &\in \mathbb{N} \\
      \sqrt{2} &= \frac{m}{n_{*}} = \frac{m(m-n_{*})}{n(m-n_{*})} = \frac{m^2-mn_{*}}{n(m-n_{*})} \\
      \, &= \frac{2n_{*}^2-mn_{*}}{n(m-n_{*})} = \frac{2n_{*}-m}{m-n_{*}} \\
      \, &= \frac{\tilde{m}}{m-n_{*}}, \tilde{m} \in \mathbb{Z} \\
    \end{align*}
    Widerspruch: $m-n_{*} < n$
  \end{proof}

  Man kann damit auch zeigen, dass $\sqrt{3}$ irrational ist.

  Was ist mit $\sqrt{k}, k \in \mathbb{N}$?
\end{example}

\begin{theorem}
  Sei $k \in \mathbb{N}$. Dann ist entweder $\sqrt{k} \in \mathbb{N}$ oder $\sqrt{k}$ irrational.
\end{theorem}
\begin{proof}
  Annahme: $k \in \mathbb{Q} \setminus \mathbb{N}$

  Es sei \( A = \left\{ n \in \mathbb{N} \middle| \exists m \in \mathbb{Z}: \sqrt{k} = \frac{m}{n} \right\} \) (vgl. oben)

  $A$ hat ein kleinstes Element $n_{*}$ (vgl. oben)

  Womit soll man dies erweitern? ($m - l \cdot n_{*}$)
\end{proof}

\end{document}