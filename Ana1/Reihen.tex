% -*- TeX-master: "../main" -*-
\chapter{Reihen}

\begin{definition}
  Sei $(a_n)_n$ eine Folge.
  \begin{align*}
    s_n &:= \sum_{k=1}^n a_k \\
    \sum_{n=1}^{\infty} a_n &:= \lim_{n \to \infty} s_n
  \end{align*}

  $s_n$ heißt Reihe.
\end{definition}

\begin{theorem}
  Sei $(a_n)_n$ eine reelle Folge mit $a_n \ge 0$. $s_n$ konvergiert $\iff$ $s_n$ ist beschränkt.
\end{theorem}

\begin{corollary}
  Sei $(a_n)_n$ eine Folge mit $a_n \ge 0$. Dann gilt entweder $\sum_{n=0}^{\infty} a_n = \infty$ oder $\sum_{n=0}^{\infty} a_n < \infty$
\end{corollary}

\begin{lemma}[Geometrische Summe und Reihe]
  \begin{align*}
    q \ne 1 \implies \sum_{k=0}^n q^k &= \frac{1-q^{k+1}}{1-q} \\\
    |q| < 1 \implies \sum_{k=0}^{\infty} q^n &= \frac{1}{1-q}
  \end{align*}
\end{lemma}

\begin{theorem}
  Seien $\sum_{n=1}^{\infty} a_n, \sum_{n=1}^{\infty} b_n$ konvergente Reihen, dann ist $\sum_{n=1}^n (\lambda a_n + \mu b_n) = \lambda \sum_{n=1}^{\infty} a_n + \mu \sum_{n=1}^{\infty} b_n$ für $\lambda, \mu \in \mathbb{R}$.
\end{theorem}

\begin{theorem}[Majorantemkriterium]
  Es sei $0 \le a_n \le b_n$. Konvergiert $\sum_{n=0}^{\infty} b_n$, so gilt: $0 \le \sum_{n=0}^{\infty} a_n \le \sum_{n=0}^{\infty} b_n$.
\end{theorem}

\begin{theorem}[Minorantemkriterium]
  Es sei $0 \le b_n \le a_n$. Divergiert $\sum_{n=0}^n b_n$, so divergiert auch $\sum_{n=0}^n a_n$.
\end{theorem}

\begin{theorem}
  Sei $(a_n)$ eine monoton fallende Nullfolge. Ist $\sum_{n=0}^{\infty}$ konvergent, so ist $\sum_{n=0}^n 2^n a_{2^n}$ auch konvergent.
\end{theorem}

\begin{theorem}
  Sei $a_n$ eine Folge reeller Zahlen. $\sum_{n=0}^{\infty} a_n$ konvergiert $\iff$ $\forall \varepsilon > 0\: \exists N_{\varepsilon} \in \mathbb{N}\: \forall m \ge n \ge N_{\varepsilon}: \left| \sum_{j=n+1}^m \right| < \varepsilon$
\end{theorem}

\begin{corollary}
  Ist $\sum_{n=0}^{\infty} a_n$ konvergent, so ist $a_n$ eine Nullfolge.
\end{corollary}

\section{Absolut konvergente Reihen}

\begin{definition}
  $\sum_{n=0}^{\infty} a_n$ heißt absolut konvergent, wenn $\sum_{n=0}^{\infty} |a_n|$ konvergiert.
\end{definition}

\begin{theorem}
  Ist $\sum_{n=0}^{\infty} a_n$ absolut konvergent, so ist sie auch konvergent und $\left| \sum_{n=1}^{\infty} a_n \right| \le \sum_{n=1}^{\infty} |a_n|$.
\end{theorem}

\begin{definition}[Majorante]
  Die Reihe $\sum_{n=0}^{\infty} c_n$ mit $\forall n \in \mathbb{N}: c_n > 0$ heißt Majorante der Reihe $\sum_{n=1}^{\infty} a_n$, falls $|a_n| \le c_n$ für fast alle $n \in \mathbb{N}$.
\end{definition}

\begin{theorem}[Majorantenkriterium]
  Hat $\sum_{n=1}^{\infty} a_n$ eine konvergente Majorante $\sum_{n=1}^{\infty} c_n$, so ist $\sum_{n=1}^{\infty} a_n$ absolut konvergent.
\end{theorem}

\begin{theorem}[Quotientenkriterium]
  Sei $\sum_{n=1}^{\infty}$ Reihe mit $a_n \ne 0$. Ferner gebe es $0 \le q \le 1$ mit $\frac{|a_{n+1}|}{|a_n|} \le q$ für fast alle $n$, dann konvergiert $\sum_{n=1}^{\infty} a_n$ absolut.
\end{theorem}

\begin{remark}
  $\iff \limsup_{n \to \infty} \frac{|a_{n+1}|}{|a_n|} < 1$.

  Ist $\liminf_{n \to \infty} \frac{|a_{n+1}|}{|a_n|} > 1 \implies$ Divergenz.
\end{remark}

\begin{theorem}
  $e = \sum_{n=0}^{\infty} \frac{1}{n!} = \lim_{n \to \infty} \left( 1 + \frac{1}{n} \right)^n$ ist irrational.
\end{theorem}

\begin{theorem}[Wurzelkriterium]
  Sei $a_n$ reelle Folge mit $\limsup_{n \to \infty} \sqrt[n]{|a_n|} < 1$, dann konvergiert $\sum_{n=0}^{\infty} a_n$ absolut. Ist $\limsup_{n \to \infty} \sqrt[n]{|a_n|} > 1$, so divergiert $\sum_{n=0}^{\infty} a_n$. Sonst ist keine Aussage möglich.
\end{theorem}

\begin{definition}
  Seien $\sum_{n=0}^{\infty} a_n, \sum_{n=0}^{\infty} b_n$ Reihen. Man nennt $\sum_{n=0}^{\infty} b_n$ Umordnung von $\sum_{n=0}^{\infty} a_n$, falls eine Bijektion $\Gamma: \mathbb{N}_0 \to \mathbb{N}_0$ existiert mit $\forall n \in \mathbb{N}_0: b_n = a_{\Gamma(n)}$
\end{definition}

\begin{definition}
  Eine Reihe $\sum_{n=0}^{\infty} a_n$ heißt unbedingt konvergent, falls jede Umordnung $\sum_{n=0}^{\infty} b_n$ von $\sum_{n=0}^{\infty} a_n$ ebenfalls konvergiert und dieselbe Summe von $\sum_{n=0}^{\infty} a_n$ hat. Andernfalls heißt $\sum_{n=0}^{\infty} a_n$ bedingt konvergent.
\end{definition}

\begin{theorem}
  $\sum_{n=0}^{\infty}$ ist absolut konvergent $\iff$ sie ist unbedingt konvergent.
\end{theorem}