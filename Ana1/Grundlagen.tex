% -*- TeX-master: "../main" -*-

\chapter{Grundlagen}
\setcounter{section}{-1}

\section{Beispiele von logischen Schlussfolgerungen}

Natürliche Zahlen: \( \mathbb{N} = \left\{ 1, 2, \dots \right\} \)

$n$ ist gerade, wenn $k \in \mathbb{N}$ existiert mit $n = 2k$.

$n$ ist ungerade, wenn $k \in \mathbb{N}$ existiert mit $n = 2k-1$.

\begin{theorem}
  Sei $n \in \mathbb{N}$. $n$ ist genau dann gerade, wenn $n^2$ gerade ist.
\end{theorem}

\begin{proof}
  a. $n$ gerade $\Rightarrow$ $n^2$ gerade
  \begin{align*}
    n &= 2k \\
    n^2 &= (2k)^2 \\
    \, &= 2\underbrace{(2k^2)}_{\in \mathbb{N}} \\
    \, &\in \mathbb{N}
  \end{align*}

  b. $n^2$ gerade $\Rightarrow$ $n$ gerade: schwer zu beweisen

  c. $n$ ungerade $\Rightarrow$ $n^2$ ungerade
  \begin{align*}
    n &= 2k-1 \\
    n^2 &= (2k-1)^2 \\
    \, &= 4k^2-4k+1 \\
    \, &= 2\underbrace{(2k^2-2k+1}_{\in \mathbb{N}})-1 \\
    \, &\in \mathbb{N}
  \end{align*}

  Was hat c mit b zu tun?

  $p$: „$n$ ist gerade“; $q$: „$n^2$ ist gerade“.

  b. $p \Rightarrow q$; c. $\neg p \Rightarrow \neg q$ $\leftarrow$ Kontraposition zu b

  D.h. b ist genau dann wahr, wenn c wahr ist.

  c ist wahr $\Rightarrow$ b ist auch wahr.
\end{proof}

\begin{example}[Beweis durch Widerspruch]
  $\sqrt{2}$ ist irrational.

  \begin{proof}
    Annahme: $\sqrt{2}$ ist rational.

    Es sei \( A = \left\{ n \in \mathbb{N} \middle| \exists m \in \mathbb{Z}: \sqrt{2} = \frac{m}{n} \right\} \)

    $\sqrt{2}$ ist rational, wenn $A$ nicht leer ist. $A \subseteq \mathbb{N}$.

    Es sei $n_{*}: \forall n \in A: n \ge n_{*}$. Dann ist $\sqrt{2} = \frac{m}{n_{*}}$.

    \begin{align*}
      m-n_{*} &= \sqrt{2}n_{*} = \underbrace{\left( \sqrt{2}-1 \right)}_{< 1}n_{*} \\
      \, &< n_{*} \\
      m-n_{*} &\in \mathbb{N} \\
      \sqrt{2} &= \frac{m}{n_{*}} = \frac{m(m-n_{*})}{n(m-n_{*})} = \frac{m^2-mn_{*}}{n(m-n_{*})} \\
      \, &= \frac{2n_{*}^2-mn_{*}}{n(m-n_{*})} = \frac{2n_{*}-m}{m-n_{*}} \\
      \, &= \frac{\tilde{m}}{m-n_{*}}, \tilde{m} \in \mathbb{Z} \\
    \end{align*}
    Widerspruch: $m-n_{*} < n$
  \end{proof}

  Man kann damit auch zeigen, dass $\sqrt{3}$ irrational ist.

  Was ist mit $\sqrt{k}, k \in \mathbb{N}$?
\end{example}

\begin{theorem}
  Sei $k \in \mathbb{N}$. Dann ist entweder $\sqrt{k} \in \mathbb{N}$ oder $\sqrt{k}$ irrational.
\end{theorem}
\begin{proof}
  Annahme: $k \in \mathbb{Q} \setminus \mathbb{N}$

  Es sei \( A = \left\{ n \in \mathbb{N} \middle| \exists m \in \mathbb{Z}: \sqrt{k} = \frac{m}{n} \right\} \) (vgl. oben)

  $A$ hat ein kleinstes Element $n_{*}$ (vgl. oben)

  $k > 1 \implies \exists l \in \mathbb{N}: l < \sqrt{k} < l+1$

  \begin{align*}
    \sqrt{k} &= \frac{m}{n_{*}} \\
    \smash{\underbrace{m - ln_{*}}_{\in \mathbb{Z}}} &= \sqrt{k} n_{*} - ln_{*} \\
    \, &= \underbrace{(\sqrt{k} - l)}_{> 0} n_{*} > 0 \\
    \implies m - ln_{*} &\in \mathbb{N} \\
    m-ln_{*} &= \underbrace{(\sqrt{k} - l)}_{< 1} n < n \\
    \sqrt{k} = \frac{m}{n} &= \frac{m(m-ln_{*})}{n_{*}(m-ln_{*})} \\
    \, &= \frac{m^2 - lmn_*}{n_{*}(m-ln_{*})} \\
    \, &= \frac{kn^2 - lmn_{*}}{n_{*}(m-ln_{*}} \\
    \, &= \frac{\overbrace{kn - lm}^{\in \mathbb{Z}}}{m-ln_{*}} \\
    \implies m-ln_{*} &\in \mathbb{N}
  \end{align*}
  Widerspruch: $m-ln_{*} < n_{*}$
\end{proof}

\section{Aussagenlogik}

\begin{definition}
  Eine Aussage\index{Aussage} ist eine Behauptung, welche sprachlich oder durch eine Formel formuliert ist. Diese kann entweder wahr oder falsch sein. (Prinzip von ausgeschlossenen dritten)
\end{definition}
\begin{remark}
  Ein Beispiel beweist niemals eine Aussage. Ein Gegenbeispiel beweist hingegen, dass die Aussage falsch ist.
\end{remark}

\begin{definition}
  Es seien $p, q$ Aussagen.
  \begin{description}
  \item[Konjunktion\index{Aussage!Konjunktion}] $p \wedge q$\index[sym]{$\wedge$}
  \item[Disjunktion\index{Aussage!Disjunktion}] $p \vee q$\index[sym]{$\vee$}
  \item[Implikation\index{Aussage!Implikation}] $p \implies q$\index[sym]{$\implies$}
  \item[Äquivalenz\index{Aussage!Äquivalenz}] $p \iff q$\index[sym]{$\iff$}
  \item[Exklusives Oder\index{Aussage!Exklusives Oder}] $(p \vee q) \wedge (\neg p \vee \neg q)$
  \end{description}
\end{definition}

\begin{definition}
  Aussagenform $H(x)$: Aussage mit Variable\index{Aussage!Aussageform}
\end{definition}
\begin{example}
  \begin{align*}
    H_1(x) &:\iff (x^2 - 3x + 2 = 0) \\
    H_2(x) &:\iff (x = 1 \vee x = 2) \\
    H_1(x) &\iff H_2(x)
  \end{align*}
\end{example}

\subsection{Beweisstruktur}
Es sei $p \implies q$. $p$ heißt die vorraussetzung, $q$ heißt die Behauptung.

Ein Beweis hat die Struktur von $p \implies r_1 \implies r_2 \implies \dots \implies r_n \implies q$. $r_1, \dots, r_n$ sind bereits bekannte wahre Aussagen oder Axiome.

\begin{theorem}
  Regeln der Aussagenlogik:
  \begin{align*}
    A &\implies A \\
    (A \implies B) \wedge (B \implies C) &\implies (A \implies C) \tag{Transitivität\index{Aussage!Transitivität}} \\
    (A \wedge B) \wedge C &\iff A \wedge B \wedge C & (A \vee B) \vee C &\iff A \vee B \vee C \tag{Assoziativität\index{Aussage!Assoziativgesetz}} \\
    A \wedge B &\iff B \wedge A & A \vee B &\iff B \vee A \tag{Kommutativität\index{Aussage!Kommutativgesetz}} \\
    A \wedge (B \vee C) &\iff (A \wedge B) \vee (A \wedge C) & A \vee (B \wedge C) &\iff (A \vee B) \wedge (A \vee C) \tag{Distributivität\index{Aussage!Distributivgesetz}} \\
    (B \implies C) &\implies (A \wedge B \implies A \wedge C) \tag{Monotonie\index{Aussage!Monotonie}} \\
    \neg (A \wedge B) &\iff \neg A \vee \neg B & \neg (A \vee B) &\iff \neg A \wedge \neg B \tag{De Morgansche Regeln\index{Aussage!De Morgansche Regeln}} \\
    \neg (\neg A) &\iff A \tag{Doppelte Negation\index{Aussage!Doppelte Negation}}
  \end{align*}
\end{theorem}

\subsection{Mengen}
Nach Cantor ist eine Menge\index{Menge} $M$ eine Zusammenfassung bestimmter, wohlunterschiedener Objekte unserer Anschauung oder unseres Denkens (welche die Elemente von $M$ genannt werden) zu einem Ganzen.

\begin{example}
  \begin{align*}
    A &:= {M, A, T, H, E, M, A, T, I, K} \\
    \, &= {M, A, H, T, E, A, I, K} \\
    \, &= {T, H, E, M, A, T, I, K} \\
  \end{align*}
\end{example}

Man schreibt $x \in A$\index[sym]{$\in$}, wenn $A$ eine Menge ist und $x$ ein Element\index{Menge!Element} von $A$ ist. Ist $x$ kein Element von $A$, so schreibt man $x \not\in A$.

Ist $H(x)$ eine Aussage, die von einer Variable $x$ abhängig ist, dann gibt es eine Menge $A := \{x|H(x)\}$. $x \in A \iff H(x)$.

\begin{definition}
  \begin{description}
  \item[Gleichheit] Zwei Mengen $A$ und $B$ sind gleich, wenn sie dieselben Elemente enthalten.\index{Menge!Gleichheit}
  \item[Leere Menge] Die leere Menge $\emptyset := \{\}$ ist die eindeutige Menge, welche kein Element enthält.\index{Menge!Leere Menge}\index[sym]{$\emptyset$}
  \item[Teilmenge] $A \subseteq B \iff \forall x \in A: x \in B$\index{Menge!Teilmenge}\index[sym]{$\subseteq$}
  \item[Echte Teilmenge] $A \subsetneq B \iff A \subseteq B \wedge A \ne B$\index{Menge!Echte Teilmenge}\index[sym]{$\subsetneq$}
  \item[Disjunkte Mengen] $\forall x \in A: x \not\in B$\index{Menge!Disjunkte Mengen}
  \end{description}
\end{definition}
\begin{remark}
  $A = B \iff A \subseteq B \wedge B \subseteq A$
\end{remark}

\begin{definition}
  Operationen mit Mengen:
  \begin{description}
  \item[Durchschnitt] $A \cap B := \left\{ x \middle| x \in A \wedge x \in B \right\}$\index{Menge!Durchschnitt}\index[sym]{$\cap$}
  \item[Vereinigung] $A \cup B := \left\{ x \middle| x \in A \vee x \in B \right\}$\index{Menge!Vereinigung}\index[sym]{$\cup$}
  \item[Differenz] $A \setminus B := \left\{ x \in A \wedge x \not\in B \right\}$\index{Menge!Differenz}\index[sym]{$\setminus$}
  \item[Komplement] Für $A \subseteq M: A^C = A^C_M = M \setminus A$\index{Menge!Komplement}
  \end{description}
\end{definition}

\begin{theorem}
  Es seien $A$, $B$, $C$, $M$ Mengen.

  \begin{align*}
    A \cap B &= B \cap A & A \cup B &= B \cup A \tag{Transitivgesetz\index{Menge!Kommutativgesetz}} \\
    A \cap (B \cap C) &= (A \cap B) \cap C & A \cup (B \cup C) &= (A \cup B) \cup C) \tag{Assoziativgesetz\index{Menge!Assoziativgesetz}} \\
    A \cap (B \cup C) &= (A \cap B) \cup (A \cap C) & A \cup (B \cap C) &= (A \cup B) \cap (A \cup C) \tag{Distributivgesetz\index{Menge!Distributivgesetz}}
  \end{align*}
\end{theorem}
\begin{proof}
  \begin{align*}
    A \cap B &= \left\{ x \middle| x \in A \wedge x \in B \right\} \\
    \, &= \left\{ x \middle| x \in B \wedge x \in A \right\} \\
    \, &= B \cap A \\
    x \in A \cup (B \cap C) &\iff x \in A \vee (x \in B \wedge x \in C) \\
    \, &\iff (x \in A \vee x \in B) \wedge (x \in A \vee x \in C) \\
    \, &\iff x \in (A \cup B) \cap (A \cup C)
  \end{align*}
\end{proof}

\begin{definition}[Mengenfamilie\index{Menge!Mengenfamilie}]
  Sei $J \ne \emptyset$ eine Indexmenge\index{Menge!Indexmenge}. Die Mengenfamilie ist gegeben durch die Mengen $A_j$ für jedes $j \in J$.
\end{definition}

\begin{definition}
  Gegeben ist $\{A\}_{j \in J}, J \ne \emptyset$.
  \begin{align*}
    \bigcap_{j \in J} A_j &:= \left\{ x \middle| \forall j \in J: x \in A_j \right\} \\
    \bigcup_{j \in J} A_j &:= \left\{ x \middle| \exists j \in J: x \in A_j \right\}
  \end{align*}
\end{definition}

\begin{definition}[Quantoren\index{Menge!Quantor}]
  \begin{itemize}
  \item $\forall$: „Für alle“\index[sym]{$\forall$}
  \item $\exists$: „Es existiert“\index[sym]{$\exists$}
  \item $\exists!$: „Es existiert genau eins“\index[sym]{$\exists"!"$}
  \end{itemize}
\end{definition}

\begin{example}[Aussageformen]
  Es sei $H(x)$ eine Aussageform und $A$ eine Menge. Man schreibt z.B.
  \begin{itemize}
  \item $\forall x \in A: H(x)$
  \item $\exists x \in A: H(x)$
  \item $\exists! x \in A: H(x)$
  \end{itemize}

  Es sei $H(x, y)$ eine Aussageform und $A, B$ Mengen. Man schreibt z.B.
  \begin{itemize}
  \item $\forall x \in A\: \exists y \in B: H(x, y)$
  \item $\exists x \in A\: \forall y \in B: H(x, y)$
  \end{itemize}
\end{example}

\begin{remark}[Negation von Quantoren]
  \begin{align*}
    \neg (\forall x \in A: H(x)) &\iff \exists x \in A: \neg H(x) \\
    \neg (\exists x \in A: H(x)) &\iff \forall x \in A: \neg H(x)
  \end{align*}
\end{remark}

\begin{example}
  \begin{align*}
    \neg (\forall x \in A\: \exists y \in B: H(x, y)) &\iff \exists x \in A: \neg (\exists y \in B: H(x, y)) \\
    \, &= \exists x \in A\: \forall y \in B: \neg H(x, y) \\
    \neg (\exists x \in A\: \forall y \in B: H(x, y)) &\iff \forall x \in A\: \exists y \in B: \neg H(x, y)
  \end{align*}
\end{example}

\begin{definition}[Potenzmenge\index{Menge!Potenzmenge}]
  Ist $A$ eine Menge, so heißt $\mathscr{P}(A) := \left\{ N \middle| x \subseteq A \right\}$ Potenzmenge von A.\index[sym]{P(M)@$\mathscr{P}(M)$}

  Eine Teilmenge $\mathscr{A} \in \mathscr{P}(A)$ heißt Mengensystem\index{Menge!Mengensystem} über $A$.\index[sym]{A@$\mathscr{A}$}
\end{definition}
\begin{example}
  \begin{align*}
    \mathscr{P}(\emptyset) &= \left\{ \emptyset \right\} \\
    \mathscr{P}(\left\{ \emptyset \right\}) &= \left\{ \emptyset, \left\{ \emptyset \right\} \right\}
  \end{align*}
\end{example}

\begin{remark}[Russels „Paradox“\index{Menge!Russels Paradox}]
  Es sei $R := \left\{ M \middle| M \text{ ist eine Menge} \wedge M \not\in M \right\}$

  Ist $R$ eine Menge, so kann man fragen, ob $R \in R$ ist
  \begin{align*}
    R \not\in R \stackrel{\text{Def}}{\iff} R \in R \\
    R \in R \stackrel{\text{Def}}{\iff} R \not\in R
  \end{align*}

  Auflösung durch moderne Mengenlehre: $R$ definiert keine Menge, sondern eine Klasse.

  Ähnlich: $\mathscr{M} := \left\{ M | M \text{ ist eine Menge} \right\}$ ist eine Klasse und keine Menge.
\end{remark}

\subsection{Kartesisches Product}
\begin{definition}
  Es seien $A, B$ Mengen. Für $a \in A, b \in B$ ist $(a, b) := \left\{ a, \left\{ a, b \right\} \right\}$ ein grordnetes Paar (2-Tupel) aus $a$ und $b$.\index{Menge!Geordnetes Paar}\index{Menge!Tupel}

  Für $a_1, a_2 \in A, b_1, b_2 \in B$ ist $(a_1, b_1) = (a_2, b_2) \iff a_1 = b_1 \wedge a_2 = b_2$

  $A \times B := \left\{ (a, b) \middle| a \in A, b \in B \right\}$ heißt das kartesische Produkt von $A$ und $B$.\index{Menge!Kartesisches Produkt}\index[sym]{$\times$}
\end{definition}

\begin{definition}
  Eine Relation\index{Relation} $R := (A, B, G)$ besteht aus
  \begin{itemize}
  \item einer Definitionsmenge\index{Relation!Definitionsmenge} $A$,
  \item einer Zielmenge\index{Relation!Zielmenge} $B$ und
  \item einem Graph\index{Relation!Graph} $G \subseteq A \times B$ (man schreibt auch $G_R$)
  \end{itemize}

  Ist $(a, b) \in G$, so sagt man „$a$ ist $R$-verwandt zu $b$“ und man schreibt $a\, R\, b$.

  $R_1 = R_2 \iff A_1 = A_2 \wedge B_1 = B_2 \wedge R_1 = R_2$

  Die inverse Relation\index{Relation!Inverse Relation} ist definiert durch $R^{-1} := (B, A, G_{R^{-1}}), G_{R^{-1}} := \left\{ (b, a) \middle| (a, b) \in G_R \right\}$
\end{definition}

\begin{definition}
  Sei $R = (A, B, G)$ eine Relation.
  \begin{itemize}
  \item $R$ ist reflexiv\index{Relation!Reflexiv}, wenn $\forall a \in a: a\, R\, a$
  \item $R$ ist symmetrisch\index{Relation!Symmetrisch}, wenn $\forall a_1, a_2 \in A: a_1\, R\, a_2 \iff a_2\, R\, a_1$
  \item $R$ is transitiv\index{Relation!Transitiv}, wenn $\forall a_1, a_2, a_3 \in A: a_1\, R\, a_2 \wedge a_2\, R\, a_3 \implies a_1\, R\, a_3$
  \item Eine Äquivalenzrelation\index{Relation!Äquivalenzrelation} ist eine reflexive, symmetrische und transitive Relation (auf $A$).
  \item Ist $a_1\,R\,a_2$, so nennt man $a_1$ äquivalent\index{Relation!Äquivalenz} zu $a_2$ bezüglich $R$. Man schreibt $a_1 \sim_R a_2$.\index[sym]{$\sim_R$}
  \end{itemize}

  Sei $R$ eine Äquivalenzrelation auf $A$, dann ist $[a]_R := \left\{ b \in A \middle| a\,R\,b \right\}$ die Äquivalenzklasse.\index{Relation!Äquivalenzklasse}\index[sym]{$[a]_R$}
\end{definition}
\begin{remark}
  Sei $R$ Äquivalenzrelation, dann ist $\forall a \in A: [a]_R \ne \emptyset$

  $a_1, a_2 \in [a]_R \implies a_1 \sim_R a_2, a_2 \sim_R a_1$
\end{remark}
\begin{lemma}
  Sei $R$ eine Äquivalenzrelation auf $A \ne \emptyset$. Für $a_1, a_2 \in R$ ist entweder $[a_1]_R = [a_2]_R$ oder $[a_1]_R \cap [a_2]_R = \emptyset$.
\end{lemma}
\begin{proof}
  Da $[a_1]_R \ne \emptyset, [a_2]_R \ne \emptyset$, ist zu zeigen: $[a_1]_R \cap [a_2]_R = \emptyset \implies [a_1]_R = [a_2]_R$.

  Es sei $b \in [a_1]_R \cap [a_2]_R$.

  Sei $c \in [a_1]_R, c \sim_R a_1$ und $b \sim_R a_1 \implies a_1 \sim_R b \implies c \sim_R b$.

  Auch $b \in [a_1]_R: b \sim_R a_2$

  $\implies c \sim_R a_2 \implies c \in [a_2]_R \implies [a_1]_R \subseteq [a_2]_R$. Aus Symmetrie des Arguments: $[a_2]_R \subseteq [a_1]_R$
\end{proof}
\begin{corollary}
  Es sei $R$ eine Äquivalenzrelation auf $A \ne \emptyset$

  Dann sind $a_1, a_2 \in A$ entweder äquivalent oder sie gehören zu disjunkten Äquivalenzklassen.
\end{corollary}
\begin{remark}
  Sei $A \ne \emptyset$ eine Menge. Man definiert die Zerlegung\index{Relation!Zerlegung} $F = \left\{ A_j \right\}_{j \in J}, \forall j \in J: A_j \in A$ mit
  \begin{itemize}
  \item $\forall j \in J: A_j \ne \emptyset$
  \item $\forall j_1, j_2 \in J, j_1 \ne j_2: A_{j_1} \cap A_{j_2} = \emptyset$
  \item $\bigcup_{j \in J} A_j = A$
  \end{itemize}

  Sei $R$ ein Äquivalenzrelation auf $A$ und $F:= \left\{ [a]_R \middle| a \in A \right\}$ ist eine Zerlegung von $A$. Man schreibt $F = A / R$ oder $A /_{\sim} R$\index[sym]{$/$}\index[sym]{$/_{\sim}$}
\end{remark}

% = TODO: complement notes from Tuesday

\section{Funktionen}
\begin{definition}
  Eine Relation $R = (A, B, G)$ heißt eine Funktion (oder Abbildung), wenn $\forall a \in A\: \exists! b = B: (a, b) \in G$.

  Man setzt $f(a) := b$.

  $A$ heißt Definitionsbereich, $B$ heißt Wertebereich (Zielmenge).
\end{definition}
\begin{example}
  \begin{align*}
    f&: \mathbb{Z} \to R, n \mapsto 3n^2+7 \\
    g&: \mathbb{Z} \to \mathbb{Z}, n \mapsto 3n^2+7 \\
    h&: \left[0, \infty\right\} \to \mathbb{R}, n \mapsto x^2+3x+4 \\
    j&: \mathbb{R} \to \mathbb{R}, n \mapsto x^2+3x+4
  \end{align*}
\end{example}

\begin{definition}
  Ist $f: A \to B$ eine Funktion, dann ist
  \begin{itemize}
  \item $M \subseteq A: f(M) := \left\{ b \in B \middle| \exists x \in M: b = f(x) \right\}$ das Bild von $M$.
  \item $N \subseteq B: f^{-1}(N) := \left\{ a \in A \middle| f(a) \in N \right\}$ das Urbild von $N$
  \end{itemize}

  Ist $f: A \to B$ eine Funktion, dann gibt es ein $f^{-1}: \mathscr{P}(B) \to \mathscr{P}(A)$

  Ist $f: A \to B$ und $M \subseteq A$, dann ist $f|_M: M \to B$ eine Einschränkung von $f$ auf $M$.
\end{definition}

\begin{definition}
  Es sei $f: A \to B$, dann ist f
  \begin{itemize}
  \item injektiv, wenn $\forall a_1, a_2 \in A: f(a_1) = f(a_2) \implies a_1 = a_2$
  \item surjektiv, wenn $f(A) = B$
  \item bijektiv, wenn $f$ injektiv und surjektiv ist.
  \end{itemize}

  Wenn $f$ injektiv ist: dann ist man $\tilde{f}: A \to f(A), a \mapsto f(a)$ surjektiv.

  Wenn $f$ bijektiv ist, dann $\exists f^{-1}: B \to A, b \mapsto a$. Man schreibt $f^{-1}(b) := a$
\end{definition}

\section{Geordnete Mengen}
\begin{definition}
  Sei $A$ ein Menge und $R$ eine Relation auf $A$. Falls
  \begin{itemize}
  \item Reflexivität: $\forall a \in A: a \prec a$
  \item Transitivität: $\forall a_1, a_2, a_3 \in A: a_1 \prec a_2 \wedge a_2 \prec a_3 \implies a_1 \prec a_2$
  \item $\forall a_1, a_2 \in A: a_1 \prec a_2 \wedge a_2 \prec a_1 \implies a_1 = a_2$
  \end{itemize}

  Man schreibt $a_1 \prec a_2$ für $a_1 R a_2$. Dann heißt $(A, \prec)$ eine teilweise geordnete Menge.

  $T \subseteq A$ heißt geordnet, wenn es eine Kette gibt mit $a_1, a_2 \in T \implies a_1 \prec a_2 \vee a_2 \prec a_1$.
\end{definition}
\begin{example}
  $\mathscr{P}(A): M \subseteq N \iff M \prec N$
\end{example}