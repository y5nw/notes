% -*- TeX-master: "../main" -*-

\chapter{Grundlagen}
\setcounter{section}{-1}

\section{Beispiele von logischen Schlussfolgerungen}

Natürliche Zahlen: \( \mathbb{N} = \left\{ 1, 2, \dots \right\} \)

$n$ ist gerade, wenn $k \in \mathbb{N}$ existiert mit $n = 2k$.

$n$ ist ungerade, wenn $k \in \mathbb{N}$ existiert mit $n = 2k-1$.

\begin{theorem}
  Sei $n \in \mathbb{N}$. $n$ ist genau dann gerade, wenn $n^2$ gerade ist.
\end{theorem}

\begin{proof}
  a. $n$ gerade $\Rightarrow$ $n^2$ gerade
  \begin{align*}
    n &= 2k \\
    n^2 &= (2k)^2 \\
    \, &= 2\underbrace{(2k^2)}_{\in \mathbb{N}} \\
    \, &\in \mathbb{N}
  \end{align*}

  b. $n^2$ gerade $\Rightarrow$ $n$ gerade: schwer zu beweisen

  c. $n$ ungerade $\Rightarrow$ $n^2$ ungerade
  \begin{align*}
    n &= 2k-1 \\
    n^2 &= (2k-1)^2 \\
    \, &= 4k^2-4k+1 \\
    \, &= 2\underbrace{(2k^2-2k+1}_{\in \mathbb{N}})-1 \\
    \, &\in \mathbb{N}
  \end{align*}

  Was hat c mit b zu tun?

  $p$: „$n$ ist gerade“; $q$: „$n^2$ ist gerade“.

  b. $p \Rightarrow q$; c. $\neg p \Rightarrow \neg q$ $\leftarrow$ Kontraposition zu b

  D.h. b ist genau dann wahr, wenn c wahr ist.

  c ist wahr $\Rightarrow$ b ist auch wahr.
\end{proof}

\begin{example}[Beweis durch Widerspruch]
  $\sqrt{2}$ ist irrational.

  \begin{proof}
    Annahme: $\sqrt{2}$ ist rational.

    Es sei \( A = \left\{ n \in \mathbb{N} \middle| \exists m \in \mathbb{Z}: \sqrt{2} = \frac{m}{n} \right\} \)

    $\sqrt{2}$ ist rational, wenn $A$ nicht leer ist. $A \subseteq \mathbb{N}$.

    Es sei $n_{*}: \forall n \in A: n \ge n_{*}$. Dann ist $\sqrt{2} = \frac{m}{n_{*}}$.

    \begin{align*}
      m-n_{*} &= \sqrt{2}n_{*} = \underbrace{\left( \sqrt{2}-1 \right)}_{< 1}n_{*} \\
      \, &< n_{*} \\
      m-n_{*} &\in \mathbb{N} \\
      \sqrt{2} &= \frac{m}{n_{*}} = \frac{m(m-n_{*})}{n(m-n_{*})} = \frac{m^2-mn_{*}}{n(m-n_{*})} \\
      \, &= \frac{2n_{*}^2-mn_{*}}{n(m-n_{*})} = \frac{2n_{*}-m}{m-n_{*}} \\
      \, &= \frac{\tilde{m}}{m-n_{*}}, \tilde{m} \in \mathbb{Z} \\
    \end{align*}
    Widerspruch: $m-n_{*} < n$
  \end{proof}

  Man kann damit auch zeigen, dass $\sqrt{3}$ irrational ist.

  Was ist mit $\sqrt{k}, k \in \mathbb{N}$?
\end{example}

\begin{theorem}
  Sei $k \in \mathbb{N}$. Dann ist entweder $\sqrt{k} \in \mathbb{N}$ oder $\sqrt{k}$ irrational.
\end{theorem}
\begin{proof}
  Annahme: $k \in \mathbb{Q} \setminus \mathbb{N}$

  Es sei \( A = \left\{ n \in \mathbb{N} \middle| \exists m \in \mathbb{Z}: \sqrt{k} = \frac{m}{n} \right\} \) (vgl. oben)

  $A$ hat ein kleinstes Element $n_{*}$ (vgl. oben)

  $k > 1 \implies \exists l \in \mathbb{N}: l < \sqrt{k} < l+1$

  \begin{align*}
    \sqrt{k} &= \frac{m}{n_{*}} \\
    \smash{\underbrace{m - ln_{*}}_{\in \mathbb{Z}}} &= \sqrt{k} n_{*} - ln_{*} \\
    \, &= \underbrace{(\sqrt{k} - l)}_{> 0} n_{*} > 0 \\
    \implies m - ln_{*} &\in \mathbb{N} \\
    m-ln_{*} &= \underbrace{(\sqrt{k} - l)}_{< 1} n < n \\
    \sqrt{k} = \frac{m}{n} &= \frac{m(m-ln_{*})}{n_{*}(m-ln_{*})} \\
    \, &= \frac{m^2 - lmn_*}{n_{*}(m-ln_{*})} \\
    \, &= \frac{kn^2 - lmn_{*}}{n_{*}(m-ln_{*}} \\
    \, &= \frac{\overbrace{kn - lm}^{\in \mathbb{Z}}}{m-ln_{*}} \\
    \implies m-ln_{*} &\in \mathbb{A}
  \end{align*}
  Widerspruch: $m-ln_{*} < n_{*}$
\end{proof}

\section{Aussagenlogik}

\begin{definition}
  Eine Aussage\index{Aussage} ist eine Behauptung, welche sprachlich oder durch eine Formel formuliert ist. Diese kann entweder wahr oder falsch sein. (Prinzip von ausgeschlossenen dritten)
\end{definition}
\begin{remark}
  Ein Beispiel beweist niemals eine Aussage. Ein Gegenbeispiel beweist hingegen, dass die Aussage falsch ist.
\end{remark}

\begin{definition}
  Es seien $p, q$ Aussagen.
  \begin{description}
  \item[Konjunktion] $p \wedge q$
  \item[Disjunktion] $p \vee q$
  \item[Implikation] $p \implies q$
  \item[Äquivalenz] $p \iff q$
  \item[Exklusives Oder] $(p \vee q) \wedge (\neg p \vee \neg q)$
  \end{description}
\end{definition}

\begin{definition}
  Aussagenform $H(x)$: Aussage mit Variable
\end{definition}
\begin{example}
  \begin{align*}
    H_1(x) &:\iff (x^2 - 3x + 2 = 0) \\
    H_2(x) &:\iff (x = 1 \vee x = 2) \\
    H_1(x) &\iff H_2(x)
  \end{align*}
\end{example}

\subsection{Beweisstruktur}
Es sei $p \implies q$. $p$ heißt die vorraussetzung, $q$ heißt die Behauptung.

Ein Beweis hat die Struktur von $p \implies r_1 \implies r_2 \implies \dots \implies r_n \implies q$. $r_1, \dots, r_n$ sind bereits bekannte wahre Aussagen oder Axiome.

\begin{theorem}
  Regeln der Aussagenlogik:
  \begin{align*}
    A &\implies A \\
    (A \implies B) \wedge (B \implies C) &\implies (A \implies C) \tag{Transitivität} \\
    (A \wedge B) \wedge C &\iff A \wedge B \wedge C & (A \vee B) \vee C &\iff A \vee B \vee C \tag{Assoziativität} \\
    A \wedge B &\iff B \wedge A & A \vee B &\iff B \vee A \tag{Kommutativität} \\
    A \wedge (B \vee C) &\iff (A \wedge B) \vee (A \wedge C) & A \vee (B \wedge C) &\iff (A \vee B) \wedge (A \vee C) \tag{Distributivität} \\
    (B \implies C) &\implies (A \wedge B \implies A \wedge C) \tag{Monotonie} \\
    \neg (A \wedge B) &\iff \neg A \vee \neg B & \neg (A \vee B) &\iff \neg A \wedge \neg B \tag{Morgan'sche Regeln} \\
    \neg (\neg A) &\iff A \tag{Doppelte Negation}
  \end{align*}
\end{theorem}

\subsection{Mengen}
Nach Cantor ist eine Menge $M$ eine Zusammenfassung bestimmter, wohlunterschiedener Objekte unserer Anschauung oder unseres Denkens (welche die Elemente von $M$ genannt werden) zu einem Ganzen.

\begin{example}
  \begin{align*}
    A &:= {M, A, T, H, E, M, A, T, I, K} \\
    \, &= {M, A, H, T, E, A, I, K} \\
    \, &= {T, H, E, M, A, T, I, K} \\
  \end{align*}
\end{example}

Man schreibt $x \in A$, wenn $A$ eine Menge ist und $x$ ein Element von $A$ ist. Ist $x$ kein Element von $A$, so schreibt man $x \not\in A$.

Ist $H(x)$ eine Aussage, die von einer Variable $x$ abhängig ist, dann gibt es eine Menge $A := \{x|H(x)\}$. $x \in A \iff H(x)$.

\begin{definition}
  \begin{description}
  \item[Gleichheit] Zwei Mengen $A$ und $B$ sind gleich, wenn sie dieselben Elemente enthalten.
  \item[Leere Menge] Die leere Menge $\emptyset := \{\}$ ist die eindeutige Menge, welche kein Element enthält.
  \item[Teilmenge] $A \subseteq B \iff \forall x \in A: x \in B$
  \item[Echte Teilmenge] $A \subsetneq B \iff A \subseteq B \wedge A \ne B$
  \item[Disjunkte Mengen] $\forall x \in A: x \not\in B$
  \end{description}
\end{definition}
\begin{remark}
  $A = B \iff A \subseteq B \wedge B \subseteq A$
\end{remark}

\begin{definition}
  Operationen mit Mengen:
  \begin{description}
  \item[Durchschnitt] $A \cap B := \left\{ x \middle| x \in A \wedge x \in B \right\}$
  \item[Vereinigung] $A \cup B := \left\{ x \middle| x \in A \vee x \in B \right\}$
  \item[Differenz] $A \setminus B := \left\{ x \in A \wedge x \not\in B \right\}$
  \item[Komplement] Für $A \subseteq M: A^C = A^C_M = M \setminus A$
  \end{description}
\end{definition}