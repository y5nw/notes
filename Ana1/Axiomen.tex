% -*- TeX-master: "../main" -*-
\chapter{Die Axiome der reellen Zahlen}
Axiome der reelle Zahlen: Es gibt eine Menge $\mathbb{R}$, genannt „reelle Zahlen“, die 3 Gruppen von Axiomen erfüllt:
\begin{itemize}
\item algebraische Axiome
\item Anordnungsaxiome
\item Vollständigkeitsaxiom (!)
\end{itemize}

\section{Algebraische Axiome}
In $\mathbb{R}$ gibt es 2 Operationen:
\begin{itemize}
\item Addition: $a+b \in \mathbb{R}$
\item Multiplikation: $a \cdot b \in \mathbb{R}$
\end{itemize}

Es gilt folgende Axiome:
\begin{enumerate}
\item Assoziativität der Addition: $(a + b) + c = a + (b + c)$
\item Kommutativität der Addition: $a + b = b + a$
\item Neutrales Element der Addition: Es gibt genau eine Zahl, genannt „null“ und geschrieben 0, mit $\forall a \in \mathbb{R}: a + 0 = a$
\item Inverse der Addition: $\forall a \in \mathbb{R}\: \exists! b \in R: a + b = 0$, geschrieben $b = -a$
\item Assoziativität der Multiplikation: $(a \cdot b) \cdot c = a \cdot (b \cdot c)$
\item Kommutativität der Multiplikation: $a \cdot b = b \cdot a$
\item Neutrales Element der Multiplikation: Es gibt genau eine Zahl, genannt „eins“ und geschrieben 1, die von null verschieden ist, mit $\forall a \in \mathbb{R}, a \cdot 1 = a$
\item Inverse der Multiplikation: $\forall a \in \mathbb{R}\: \exists! b \in R: a \cdot b = 1$, geschrieben $b = a^{-1}$
\item Distributivgesetz: $a \cdot (b + c) = a \cdot b + a \cdot c$
\end{enumerate}

Jede Menge $\mathbb{K}$, die die o.g. Axiome erfüllt, heißt Körper.

\begin{definition}
  Notation:
  \begin{itemize}
  \item Negation: $a - b := a + (-b)$
  \item Quotient: $\frac{a}{b} := a \cdot b^{-1}$
  \end{itemize}
\end{definition}

\begin{theorem}
  Es gilt:
  \begin{itemize}
  \item $-(-a) = a$
  \item $(-a)+(-b) = -(a+b)$
  \item $(a^{-1})^{-1} = a$
  \item $a^{-1} b^{-1} = (ab)^{-1}$
  \item $a \cdot 0 = 0$
  \item $a(-b) = -(ab)$
  \item $a(b-c) = ab-ac$
  \item $ab = 0 \implies a = 0 \vee b = 0$
  \end{itemize}
\end{theorem}
\begin{proof}[zu $a \cdot 0 = 0$]
  \begin{align*}
    a \cdot 0 + a \cdot 0 &= a \cdot (0+0) \\
    \,&= a \cdot 0 \\
    \implies (a \cdot 0 + a \cdot 0) + (-a \cdot 0) &= a \cdot 0 + (-a \cdot 0) \\
    \,&= 0 \\
    (a \cdot 0 + a \cdot 0) + (-a \cdot 0) &= a \cdot 0 + 0 \\
    \,&= a \cdot 0 \\
    \implies a \cdot 0 &= 0
  \end{align*}
\end{proof}
\begin{proof}[zu $ab = 0 \implies a = 0 \vee b = 0$]
  \begin{align*}
    a \ne 0 \implies b &= 1 \cdot b \\
    \,&= (a^{-1}a)b \\
    \,&= a^{-1}(ab) \\
    \,&= a \cdot 0 \\
    \,&= 0
  \end{align*}
\end{proof}

\begin{theorem}
  \begin{align*}
    \frac{a}{b} + \frac{c}{d} &= \frac{ad+cb}{bd} \\
    \frac{a}{b} \cdot \frac{c}{d} &= \frac{ac}{bd} \\
    \frac{\frac{a}{b}}{\frac{c}{d}} &= \frac{ad}{bc}
  \end{align*}
\end{theorem}

\begin{proof}[Zur Eindeutigkeit von 0 und 1]
  Es seien 0 und $0'$ neutrale Elemente der Addition. Dann ist $0 = 0 + 0' = 0'$
\end{proof}
\begin{proof}[Zur Eindeutigkeit der Inverse]
  Es seien $a + b = 0, a + b' = 0$. Dann ist
  \begin{align*}
    b' &= b' + 0 \\
    \,&= b' + (a + b) \\
    \,&= (a + b') + b \\
    \,&= b
  \end{align*}
\end{proof}

\section{Die Anordnungsaxiome}
\[ \forall a, b \in \mathbb{R}: a = b \vee a \ne b \]

Ist $a \ne b$, so besteht eine Anordnung $<$, sodass genau eine der Relationen $a < b$ oder $b < a$ gilt.

D.h., für alle $a, b \in \mathbb{R}$ gilt genau eine der Aussagen $a < b, a = b, a > b$.

Diese Anordnung genügt den Axiomen
\begin{itemize}
\item Transitivität: $a < b \wedge b < c \implies a < c$
\item $a < b, c = \mathbb{R} \implies a + c < b + c$
\item $a < b, c > 0 \implies ac < bc$
\end{itemize}

\begin{definition}
  Notation:
  \begin{itemize}
  \item $a < b$: $a$ ist echt kleiner als $b$
  \item $a \le b \iff a < b \vee a = b$
  \item $a > b$: $a$ ist größer als $b$
  \item $a \ge b \iff a > b \vee a = b$
  \item $a$ ist positiv, wenn $a > 0$; $a$ ist negativ, wenn $a < 0$
  \item $a$ ist nicht negativ, wenn $a \ge 0$; $a$ ist nicht positiv, wenn $a \le 0$
  \end{itemize}
\end{definition}

\begin{example}[Beweis für $a < b \iff b - a > 0$]
  \begin{align*}
    a < b &\implies a-a < b-a \\
    \,&\implies 0 < b-a \\
    b-a > 0 &\implies b-a+a > a \\
    \,&\implies b > a
  \end{align*}
\end{example}

\begin{theorem}
  \begin{align*}
    a < b &\iff b-a > 0 \\
    a < 0 &\iff -a > 0 \\
    a > 0 &\iff -a < 0 \\
    a < b &\iff -b < -a \\
    a < b \wedge c < d &= a+c < b+d \\
    ab > 0 &\iff (a > 0 \wedge b > 0) \vee (a < 0 \wedge b < 0) \\
    ab < 0 &\iff (a > 0 \wedge b < 0) \vee (a < 0 \wedge b > 0) \\
    a \ne 0 &\iff a^2 > 0 \\
    a < b \wedge c < 0 &\iff ac > bc \\
    a > 0 &\iff \frac{1}{a} > 0 \\
    a^2 < b^2 \wedge a > 0 \wedge b > 0 &\implies a < b
  \end{align*}
\end{theorem}
\begin{proof}[Beweis für $a < 0 \iff b-a > 0$, etc.]
  Sei $a < b$, dann ist
  \begin{align*}
    a + (-a) &< b + (-a) \\
    0 &< b - a
  \end{align*}

  Man setzt oben $b = 0$ ein. Dann ist
  \begin{align*}
    0 &< 0 - a \\
    0 &< -a
  \end{align*}
\end{proof}
\begin{proof}[Beweis für $a < b \wedge c < d \iff a+c < b+d$]
  \begin{align*}
    a < b \wedge c + d &\iff a + c < b + c \wedge b + c < b + d \\
    \,&\iff a + c < b + d
  \end{align*}
\end{proof}
\begin{proof}[Beweis für $a, b > 0 \vee a, b < 0 \implies ab > 0$]
  \begin{align*}
    a, b > 0 &\iff ab > 0b \\
    a, b < 0 &\iff (-a), (-b) > 0 \\
    \,&\implies (-a)(-b) > 0 \\
    \,&\iff ab > 0
  \end{align*}
\end{proof}

\section{Das Vollständigkeitsaxiom}
\begin{axiom}[Vollständigkeitsaxiom]
  Jede nichtleere Teilmenge $M \subseteq \mathbb{R}$, welche nach oben beschränkt ist, besitzt eine kleinste obere Schrankt, ganannt das Supremum vom $M$, geschrieben $\operatorname{sup} M$.
\end{axiom}

\begin{definition}
  Sei $M \subseteq \mathbb{R}$, $M \ne \emptyset$
  \begin{itemize}
  \item $M$ heißt nach oben beschränkt, falls $k \in \mathbb{R}$ existiert mit $\forall x \in M: x \le k$. Jede solche Zahl heißt obere Schranke von $M$.
  \item $M$ heißt nach unten beschränkt, falls $k \in \mathbb{R}$ existiert mit $\forall x \in M: x \ge k$. Jede solche Zahl heißt untere Schranke von $M$.
  \item $M$ heißt beschränkt, falls $k \ge 0$ existiert mit $\forall x \in M: -k \le x \le k$.
  \end{itemize}
\end{definition}
\begin{definition}
  Eine Zahl $k \in \mathbb{R}$ heißt kleinste obere (bzw. größte untere) Schranke, falls
  \begin{itemize}
  \item Es eine obere (bzw. untere) Schranke ist und
  \item Es gibt keine kleinere obere (bzw. größere untere) Schranke für $M$
  \end{itemize}
\end{definition}
\begin{remark}
  $x \le k \iff -k \ge -x$. D.h.,
  \begin{itemize}
  \item $k$ ist eine obere Schranke für $M$ $\iff$ $-k$ ist eine untere Schranke für $-M = \left\{ x \middle| x \in M \right\}$.
  \item $k$ ist die kleinste obere Schranke für $M$ $\iff$ $-k$ ist die größte untere Schranke für $-M$.
  \end{itemize}

  Jede nichtleere nach unten beschränkte Menge $M \subseteq \mathbb{R}$ hat eine größte untere Schranke, genannt Infimum von $M$, geschrieben $\operatorname{inf} M$
\end{remark}
\begin{example}
  Es seien $M = [0, 1] = \left\{ x \middle| 0 \le x \le 1 \right\}, A = (0, 1) = \left\{ x \middle| 0 < x < 1 \right\}$.
  \begin{align*}
    \operatorname{sup} M &= 1 & \operatorname{inf} M &= 0 \\
    \operatorname{sup} A &= 1 & \operatorname{inf} A &= 0
  \end{align*}
\end{example}
\begin{notation}
  Sei $M \subseteq R, M \ne \emptyset$. Man schreibt
  \begin{itemize}
  \item $\operatorname{sup} M < \infty$ bzw. $\operatorname{inf} M > -\infty$, falls $M$ nach oben (bzw. unten) beschränkt ist
  \item $\operatorname{sup} M = \infty$ bzw. $\operatorname{inf} M = \infty$, falls $M$ nicht nach oben (bzw. unten) beschränkt ist
  \end{itemize}
\end{notation}
\begin{theorem}
  Sei $M \subseteq \mathbb{R}, M \ne \emptyset$.
  \begin{itemize}
  \item Ist $\operatorname{sup} M < \infty$, folgt $\forall \epsilon > 0\: \exists x \in M: (\operatorname{sup} M)-\varepsilon < x$.
  \item Ist $\operatorname{sup} M = \infty$, so gilt $\forall k \ge 0\: \exists x \in M: x \ge k$
  \end{itemize}
\end{theorem}
\begin{theorem}
  Sei $M \subseteq \mathbb{R}, M \ne \emptyset$.
  \begin{itemize}
  \item Ist $\operatorname{inf} M > -\infty$, folgt $\forall \varepsilon > 0\: \exists x \in M: x < (\operatorname{inf} M)+\varepsilon$
  \item Ist $\operatorname{inf} M = -\infty$, folgt $\forall k \ge 0\: \exists x \in M: x < -k$
  \end{itemize}
\end{theorem}
\begin{theorem}
  Sei $M \subseteq \mathbb{R}, M \ne \emptyset$.
  \begin{itemize}
  \item $m \in M$ heißt größtes Element (Maximum) von $M$, geschrieben $\operatorname{max} M$, wenn $\forall x \in M: x \le m$
  \item $m \in M$ heißt kleinstes Element (Minimum) von $M$, geschrieben $\operatorname{min} M$, wenn $\forall x \in M: x \ge m$
  \end{itemize}
\end{theorem}
\begin{example}
  Sei $M = \left[0, 1\right)$. Dann ist $\operatorname{sup} M = 1, \operatorname{inf} M = 0$
  \begin{itemize}
  \item $M$ hat kein Maximum.
  \item $M$ hat ein Minimum.
  \end{itemize}
\end{example}