% -*- TeX-master: "../main" -*-
\chapter{Folgen}

\section{Beschränkung}
\begin{definition}
  Eine Folge $a_n$ heißt nach oben (bzw. unten) beschränkt, falls $\exists k \in \mathbb{R}\: \forall n \in \mathbb{N}: a_n \le k$ bzw. $\exists k \in \mathbb{R}\: \forall n \in \mathbb{N}: a_n \ge k$. Eine Folge heißt beschränkt, falls $\exists k > 0\: \forall n \in \mathbb{N}: -k \le a_n \le k$.
\end{definition}

\begin{theorem}
  Jede konvergente Folge $a_n$ ist beschränkt.
\end{theorem}

\begin{theorem}
  Der Grenzwert einer konvergenten Folge ist eindeutig.
\end{theorem}

\begin{theorem}
  Seien $x_n, y_n$ konvergente reelle Folgen mit $x_n \to a, y_n \to b$ für $n \to \infty$, dann ist
  \begin{itemize}
  \item $x_n + y_n \to a+b$
  \item $x_n y_n \to ab$
  \item $\lambda x_n \to \lambda a$
  \item $b \ne 0 \implies \frac{x_n}{y_n} \to \frac{a}{b}$ ($\frac{x_n}{y_n}$ ist für fast alle $n$ definiert)
  \item $x_n \le y_n \text{ für fast alle $n$} \implies a \le b$
  \end{itemize}
\end{theorem}

\begin{theorem}
  Seien $a_n, b_n, c_n$ Folgen mit $a_n \to 0, (b_n - a_n) \to 0, a_n \le c_n \le b_n$, dann konvergiert $c_n$ gegen 0.
\end{theorem}

\section{Monotone Konvergenz}

\begin{definition}
  Eine reelle Folge $a_n$ heißt (monoton) wachsend bzw. streng monoton wachsend, wenn $\forall n \in \mathbb{N}: a_n \le a_{n+1} \text{ bzw. } a_n < a_{n+1}$. Sie heißt (monoton) fallend bzw. streng monoton fallend, wenn $\forall n \in \mathbb{N}: a_n \ge a_{n+1} \text{ bzw. } a_n > a_{n+1}$. Man nennt $a_n$ (streng) monoton, wenn $a_n$ (strend) wachsend oder (streng) fallend ist.
\end{definition}

\begin{theorem}
  Eine monoton wachsende (bzw. fallende) Folge $a_n$ konvergiert genau dann, wenn $a_n$ nach oben (bzw. unten) beschränkt ist.
\end{theorem}

\begin{definition}[Divergente Folgen]
  Eine reelle Folge $x_n$ strebt gegen $+\infty$, falls $\forall k \ge 0\: \exists N \in \mathbb{N}\: \forall n \ge N: x_n \ge k$.

  Eine reelle Folge $x_n$ strebt gegen $-\infty$, falls $-x_n$ gegen $+\infty$ strebt.
\end{definition}

\begin{example}
  $x_n$ ist divergent (gegen $\infty$)
\end{example}

\begin{theorem}
  $x_n \xrightarrow{n \to \infty} \infty \implies \frac{1}{x_n} \xrightarrow{n \to \infty} 0$

  Ist eine Nullfolge $x_n > 0$ für fast alle $n$, dann ist $\frac{1}{x_n} \xrightarrow{n \to \infty} \infty$
\end{theorem}

\begin{example}[$\frac{n}{2^n} \to 0$]
  \begin{align*}
    2^n &= (1+1)^n \\
    \,&= \sum_{j=0}^n \binom{n}{j} \\
    \,&\stackrel{n\ge2}{\ge} \binom{n}{2} \\
    \,&= \frac{n(n-1)}{2}
  \end{align*}
\end{example}

\section{Häufungswerte und Teilfolgen}
\begin{definition}[Umordnung]
  Sei $a_n$ eine reelle Folge. Die Umordnung ist gegeben durch eine Bijektion $\sigma: \mathbb{N} \to \mathbb{N}$. $b_n := a_{\sigma(n)}$ heißt Umordnung von $a_n$.
\end{definition}
\begin{definition}[Teilfolge]
  Sei $a_n$ eine reelle Folge und $\kappa: \mathbb{N} \to \mathbb{N}$ streng monoton wachsend. $b_n := a_{\kappa(n)}$ heißt eine Teilfolge von $a_n$.
\end{definition}

\begin{theorem}
  Für jede konvergente reelle Folge $a_n$ konvergiert jede Umordnung und jede Teilfolge gegen denselben Grenzwert.
\end{theorem}

\begin{definition}
  Sei $a_n$ reelle Folge. Eine relle Zahl $a$ heißt Häufungswert (oder Häufungspunkt) von $a_n$, falls $\forall \varepsilon > 0\: \forall L \in \mathbb{N}\: \exists n > L: a-\varepsilon \le a_n \le a+\varepsilon$
\end{definition}

\begin{example}
  Die Folge $a_n = (-1)^n$ hat die Häufungswerte 1 und -1.

  Die Folge $a_n = (-1)^n + \frac{1}{n}$ hat (auch) die Häufungswerte 1 und -1.

  Die Folge $a_n = n$ hat keinen Häufungswert.
\end{example}

\begin{theorem}
  Sei $a \in \mathbb{R}$ eine Häufungswert der Folge $a_n$. Genau dann existert eine Teilfolge von $a_n$, die gegen $a$ konvergiert.
\end{theorem}

\section{Größte und kleinste Häufungspunkte, limes superior und limes inferior}
\begin{definition}
  Es sei $(a_n)_n$ eine beschränkte Folge.
  $\limsup_{n \to \infty} a_n := \lim_{n \to \infty} \sup_{l \ge n} a_l$ heißt limes superior.
  $\liminf_{n \to \infty} a_n := \lim_{n \to \infty} \inf_{l \ge n} a_l$ heißt limes inferior.
\end{definition}
\begin{remark}
  Klar: $\liminf_{n \to \infty} a_n \le \limsup_{n \to \infty} a_n$
\end{remark}

\begin{lemma}[Charakterisierung von limes superior und limes inferior]
  Es sei $(a_n)_n$ eine beschränkte Folge. Dann ist
  \begin{align*}
    a^{*} = \limsup_{n \to \infty} a_n \iff \forall \varepsilon > 0:& a_n < a^{*} + \varepsilon \wedge a_n > a^{*} - \varepsilon \\
    \,&\text{für fast alle $n$}\\
    a_{*} = \liminf_{n \to \infty} a_n \iff \forall \varepsilon > 0:& a_n < a_{*} + \varepsilon \wedge a_n > a_{*} - \varepsilon \\
    \,&\text{für fast alle $n$}
  \end{align*}
\end{lemma}

\begin{lemma}
  Es sei $(a_n)_n$ eine beschränkte volle Folge und $H((a_n)_n)$ die Menge der Häufungswerte dieser Folge. Dannn ist \[ \limsup a_n, \liminf a_n \in H((a_n)_n) \]

  Daraus folgt insbesondere, dass \[ H((a_n)_n) \ne \emptyset \]

  Ferner gilt: \[ \forall x \in H((a_n)_n): \liminf a_n \le x \le \limsup a_n \]
\end{lemma}

\begin{notation}
  Ist $(a_n)_n$ nicht nach oben beschränkt, schreibt man $\lim \sup a_n = \infty$

  Ist $(a_n)_n$ nicht nach unten beschränkt, schreibt man $\lim \inf a_n = -\infty$
\end{notation}

\begin{corollary}
  Eine volle Folge $(a_n)_n$ konvergiert $\iff$ $(a_n)_n$ ist beschränkt und $\limsup a_n = \liminf a_n$
\end{corollary}

\begin{theorem}[Satz von Bolzano-Weierstraß]
  Jede beschränkte volle Folge $(a_n)_n$ besitzt mindestens einen Häufungspunkt.
\end{theorem}

\begin{corollary}
  Jede beschränkte reelle Folge $(a_n)$ hat eine konvergente Teilfolge.
\end{corollary}

\section{Konvergenzkriterium von Cauchy}
\begin{definition}[Cauchy-Folge]
  Eine volle Folge $(a_n)_n$ heißt Cauchy-Folge, wenn $\forall \varepsilon > 0\: \exists N \in \mathbb{N}\: \forall n,m \ge N: |a_n-a_m| < \varepsilon$
\end{definition}
\begin{lemma}
  Jede konvergente volle Folge $(a_n)_n$ ist eine Cauchy-Folge.
\end{lemma}
\begin{lemma}
  Jede konvergente Cauchy-Folge ist beschränkt.
\end{lemma}

\begin{lemma}
  Eine volle Cauchy-Folge $(a_n)_n$ konvergiert $\iff$ sie hat eine konvergente Teilfolge
\end{lemma}

\begin{theorem}
  Jede volle Folge konvergiert genau dann, wenn sie eine Cauchy-Folge ist.
\end{theorem}