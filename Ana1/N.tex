% -*- TeX-master: "../main" -*-
\chapter{Die natürliche Zahlen $\mathbb{N}$}
\begin{definition}
  Eine Teilmenge $A \in \mathbb{R}$ heißt induktiv, falls
  \begin{itemize}
  \item $1 \in A$
  \item $x \in A \implies x+1 \in A$
  \end{itemize}
\end{definition}
\begin{observation}
  Sei $J$ eine Indexmenge und $A_j, j \in J$ induktive Teilmengen von $\mathbb{R}$. Dann ist $A := \bigcap_{j \in J} A_j$ ist induktiv.
  \begin{proof}
    \begin{align*}
      x \in A &\iff \forall j \in J: x \in A_j \\
      \,&\implies \forall j \in J: x+1 \in A_j \\
      \,&\implies x+1 \in A
    \end{align*}
  \end{proof}
\end{observation}
\begin{definition}
  Sei $\mathfrak{J} := \left\{ A \subseteq R \middle| A \text{ ist induktiv} \right\}$. Man definiert $\mathbb{N}$ als die kleinster induktive Teilmenge von $\mathbb{R}$, d.h. $\mathbb{N} := \bigcap_{A \in \mathfrak{J}} A$
\end{definition}

\begin{theorem}[Induktionsprinzip]
  Ist $M \subseteq \mathbb{N}$ induktiv, dann ist $M = \mathbb{N}$.
\end{theorem}
\begin{proof}
  Nach Vorraussetzung ist $M \subseteq \mathbb{N}$. Aus Definition von $\mathbb{N}$ ist $\mathbb{N} \subseteq M \implies M = \mathbb{N}$.
\end{proof}